\begin{verbatim}

#include <stdio.h>
#include <stdlib.h>
#include <string.h>
#include <sys/types.h>
#include <sys/socket.h>
#include <netinet/in.h>
#include <arpa/inet.h>
#include <unistd.h>
#include <sys/mman.h>
#include <sys/stat.h>
#include <gnutls/gnutls.h>
#include <gnutls/x509.h>

/* A TLS client that loads the certificate and key.
 */

#define MAX_BUF 1024
#define SA struct sockaddr
#define MSG "GET / HTTP/1.0\r\n\r\n"

#define CERT_FILE "cert.pem"
#define KEY_FILE "key.pem"
#define CAFILE "ca.pem"

static int cert_callback(gnutls_session session,
                  const gnutls_datum* req_ca_rdn, int nreqs,
                  const gnutls_pk_algorithm* sign_algos, int sign_algos_length,
                  gnutls_retr_st * st);

gnutls_x509_crt crt;
gnutls_x509_privkey key;

/* Helper functions to load a certificate and key
 * files into memory. They use mmap for simplicity.
 */
static gnutls_datum mmap_file( const char* file)
{
int fd;
gnutls_datum mmaped_file = { NULL, 0 };
struct stat stat_st;
void* ptr;

   fd = open( file, 0);
   if (fd==-1) return mmaped_file;
   
   fstat( fd, &stat_st);
   
   if ((ptr=mmap( NULL, stat_st.st_size, PROT_READ, MAP_SHARED, fd, 0)) == MAP_FAILED)
      return mmaped_file;
   
   mmaped_file.data = ptr;
   mmaped_file.size = stat_st.st_size;
   
   return mmaped_file;
}

static void munmap_file( gnutls_datum data)
{
   munmap( data.data, data.size);
}

/* Load the certificate and the private key.
 */
static void load_keys( void)
{
int ret;
gnutls_datum data;

   data = mmap_file( CERT_FILE);
   if (data.data == NULL) {
      fprintf(stderr, "*** Error loading cert file.\n");
      exit(1);
   }
   gnutls_x509_crt_init( &crt);
   
   ret = gnutls_x509_crt_import( crt, &data, GNUTLS_X509_FMT_PEM);
   if (ret < 0) {
      fprintf(stderr, "*** Error loading key file: %s\n", gnutls_strerror(ret));
      exit(1);
   }

   munmap_file( data);

   data = mmap_file( KEY_FILE);
   if (data.data == NULL) {
      fprintf(stderr, "*** Error loading key file.\n");
      exit(1);
   }

   gnutls_x509_privkey_init( &key);
   
   ret = gnutls_x509_privkey_import( key, &data, GNUTLS_X509_FMT_PEM);
   if (ret < 0) {
      fprintf(stderr, "*** Error loading key file: %s\n", gnutls_strerror(ret));
      exit(1);
   }

   munmap_file( data);
   
}

int main()
{
   int ret, sd, ii;
   gnutls_session session;
   char buffer[MAX_BUF + 1];
   gnutls_certificate_credentials xcred;
   /* Allow connections to servers that have OpenPGP keys as well.
    */

   gnutls_global_init();

   load_keys();

   /* X509 stuff */
   gnutls_certificate_allocate_credentials(&xcred);

   /* sets the trusted cas file
    */
   gnutls_certificate_set_x509_trust_file(xcred, CAFILE, GNUTLS_X509_FMT_PEM);

   gnutls_certificate_client_set_retrieve_function( xcred, cert_callback);
   
   /* Initialize TLS session 
    */
   gnutls_init(&session, GNUTLS_CLIENT);

   /* Use default priorities */
   gnutls_set_default_priority(session);

   /* put the x509 credentials to the current session
    */
   gnutls_credentials_set(session, GNUTLS_CRD_CERTIFICATE, xcred);

   /* connect to the peer
    */
   sd = tcp_connect();

   gnutls_transport_set_ptr( session, (gnutls_transport_ptr)sd);

   /* Perform the TLS handshake
    */
   ret = gnutls_handshake( session);

   if (ret < 0) {
      fprintf(stderr, "*** Handshake failed\n");
      gnutls_perror(ret);
      goto end;
   } else {
      printf("- Handshake was completed\n");
   }

   gnutls_record_send( session, MSG, strlen(MSG));

   ret = gnutls_record_recv( session, buffer, MAX_BUF);
   if (ret == 0) {
      printf("- Peer has closed the TLS connection\n");
      goto end;
   } else if (ret < 0) {
      fprintf(stderr, "*** Error: %s\n", gnutls_strerror(ret));
      goto end;
   }

   printf("- Received %d bytes: ", ret);
   for (ii = 0; ii < ret; ii++) {
      fputc(buffer[ii], stdout);
   }
   fputs("\n", stdout);

   gnutls_bye( session, GNUTLS_SHUT_RDWR);

 end:

   tcp_close( sd);

   gnutls_deinit(session);

   gnutls_certificate_free_credentials(xcred);

   gnutls_global_deinit();

   return 0;
}



/* This callback should be associated with a session by calling
 * gnutls_certificate_client_set_retrieve_function( session, cert_callback),
 * before a handshake.
 */

static int cert_callback(gnutls_session session,
                  const gnutls_datum* req_ca_rdn, int nreqs,
                  const gnutls_pk_algorithm* sign_algos, int sign_algos_length,
                  gnutls_retr_st * st)
{
   char issuer_dn[256];
   int i, ret;
   size_t len;
   gnutls_certificate_type type;

   /* Print the server's trusted CAs
    */
   if (nreqs > 0)
      printf("- Server's trusted authorities:\n");
   else
      printf("- Server did not send us any trusted authorities names.\n");

   /* print the names (if any) */
   for (i = 0; i < nreqs; i++) {
      len = sizeof(issuer_dn);
      ret = gnutls_x509_rdn_get(&req_ca_rdn[i], issuer_dn, &len);
      if (ret >= 0) {
         printf("   [%d]: ", i);
         printf("%s\n", issuer_dn);
      }
   }

   /* Select a certificate and return it.
    * The certificate must be of any of the "sign algorithms"
    * supported by the server.
    */

   type = gnutls_certificate_type_get( session);
   if (type == GNUTLS_CRT_X509) {
      st->type = type;
      st->ncerts = 1;

      st->cert.x509 = &crt;
      st->key.x509 = key;

      st->deinit_all = 0;
   } else {
      return -1;
   }

   return 0;

}

\end{verbatim}
