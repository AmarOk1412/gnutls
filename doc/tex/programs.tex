\chapter{Included programs}

\section{The ``srptool'' program}
\label{srptool}

The ``srptool'' is a very simple program that emulates the programs in the 
\emph{Stanford SRP libraries}.
It is intended for use in places where you don't expect SRP
authentication to be the used for system users.
Traditionally \emph{libsrp} used two files. One called 'tpasswd' which holds usernames 
and verifiers, and 'tpasswd.conf' which holds generators and primes.
\par
How to use srptool:

\begin{itemize}


\item To create tpasswd.conf which holds the g and n values for
SRP protocol (generator and a large prime), run:
\begin{verbatim}
$ srptool --create-conf /etc/tpasswd.conf
\end{verbatim}


\item This command will create /etc/tpasswd and will add user 'test' (you will also
be prompted for a password). Verifiers are stored by default in the
way libsrp expects.
\begin{verbatim}
$ srptool --passwd /etc/tpasswd \ 
    --passwd-conf /etc/tpasswd.conf -u test
\end{verbatim}

\item This command will check against a password. If the password matches 
the one in /etc/tpasswd you will get an ok.
\begin{verbatim}
$ srptool --passwd /etc/tpasswd \ 
    --passwd-conf /etc/tpasswd.conf --verify -u test
\end{verbatim}

\end{itemize}



\section{The ``gnutls-cli-debug'' program}

This program was created to assist in debugging \gnutls{}, but it
might be useful to extract a \tls{} server's capabilities. 
It's purpose is to connect onto a \tls{} server, perform
some tests and print the server's capabilities. An example output is:

\begin{verbatim}
crystal:/cvs/gnutls/src$ ./gnutls-cli-debug localhost -p 5556
Resolving 'localhost'...
Connecting to '127.0.0.1:5556'...
Checking for TLS 1.0 support... yes
Checking for SSL 3.0 support... yes
Checking for version rollback bug in RSA PMS... no
Checking for version rollback bug in Client Hello... no
Checking whether we need to disable TLS 1.0... no
Checking whether the server ignores the RSA PMS version... no
Checking whether the server can accept Hello Extensions... yes
Checking whether the server can accept cipher suites not in SSL 3.0 spec... yes
Checking whether the server can accept a bogus TLS record version in the client hello... yes
Checking for certificate information...
- Certificate type: X.509
 - Got a certificate list of 1 certificates.

 - Certificate[0] info:
 # valid since: Sat Jul  7 13:18:00 EEST 2001
 # expires at: Sun Jul  7 13:18:00 EEST 2002
 # serial number: 01 
 # fingerprint: 43 ab a2 a7 d3 6a 28 02 60 73 b4 a5 c3 84 0a 3f 
 # version: #3
 # public key algorithm: RSA
 #   Modulus: 1024 bits
 # Subject's DN: C=GR,ST=Attiki,L=Athina,O=GNUTLS,OU=GNUTLS dev.,CN=localhost,EMAIL=root@localhost
 # Issuer's DN: C=GR,ST=Attiki,L=Athina,O=GNUTLS,OU=GNUTLS dev.,CN=GNUTLS TEST CA,EMAIL=gnutls-dev@gnupg.org

 
Checking for trusted CAs...
- Server's trusted authorities:
   [0]: C=GR,ST=Attiki,L=Athina,O=GNUTLS,OU=GNUTLS dev.,CN=GNUTLS TEST CA,EMAIL=gnutls-dev@gnupg.org
 
Checking whether the server understands TLS closure alerts... yes
Checking whether the server supports session resumption... no
Checking for export-grade ciphersuite support... no
Checking for ephemeral Diffie Hellman support... no
Checking for ephemeral Diffie Hellman prime size... N/A
Checking for AES cipher support... yes
Checking for 3DES cipher support... yes
Checking for ARCFOUR cipher support... yes
Checking for MD5 MAC support... yes
Checking for SHA1 MAC support... yes
Checking for max record size (TLS extension)... yes
Checking for OpenPGP authentication support (TLS extension)... no

\end{verbatim}

\section{The ``certtool'' program}

This is a program to generate X.509 certificates, certificate requests, and
private keys.

How to use certtool:

\begin{itemize}

\item To create a self signed certificate, use the command:
\begin{verbatim}
$ certtool --generate-privkey --outfile ca-key.pem
$ certtool --generate-self-signed --load-privkey ca-key.pem --outfile ca-cert.pem
\end{verbatim}
Note that a self-signed certificate usually belongs to a certificate
authority, that signs other certificates.

\item To create a private key, run:
\begin{verbatim}
$ certtool --generate-privkey --outfile key.pem
\end{verbatim}

\item To create a certificate request, run:
\begin{verbatim}
$ certtool --generate-request --load-privkey key.pem --outfile request.pem
\end{verbatim}

\item To generate a certificate using the previous request, use the command:
\begin{verbatim}
$ certtool --generate-certificate --load-request request.pem --outfile cert.pem \
   --load-ca-certificate ca-cert.pem --load-ca-privkey ca-key.pem
\end{verbatim}

\item To view the certificate information, use:
\begin{verbatim}
$ certtool --certificate-info --infile cert.pem
\end{verbatim}

\item To generate a PKCS \#12 structure using the previous key and certificate, use the command:
\begin{verbatim}
$ certtool --load-certificate cert.pem --load-privkey key.pem --to-p12 \
   --outder --outfile key.p12
\end{verbatim}


\end{itemize}

