\chapter{Authentication methods}
\par
The following authentication schemas are supported in \gnutls:
\begin{enumerate}
 \item Certificate authentication
 \item Anonymous authentication
 \item SRP authentication
\end{enumerate}

\section{Authentication using X.509 certificates}
This authentication method is part of the certificate authentication
method in \gnutls{}.
The X.509 protocols rely on a hierarchical trust model. In this trust model
Certification Authorities (CAs) are used to certify entities.
Usually more than one certification authorities exist, and certification
authorities may certify other authorities to issue certificates as well,
following a hierachical model. 
One needs to trust one or more CAs for his secure
communications. In that case only the certificates issued by the trusted
authorities are acceptable. 
\par
X.509 certificates contain the public parameters, 
of a public key algorithm, and the authority's signature, which proves the
authenticity of the parameters.
\par
The key exchange methods shown in \hyperref{figure}{figure }{}{fig:cert} are
available in X.509 authentication. 

\par
Note that \gnutls{} is not a generic purpose X.509 toolkit\footnote{Aegypten is such a toolkit. See 
\htmladdnormallink{http://www.gnupg.org/aegypten/}{http://www.gnupg.org/aegypten/}}. 
\gnutls{} only includes the required,
in order to use the TLS ciphersuites which require X.509 certificates.

\begin{figure}[hbtp]
\begin{tabular}{|l|p{9cm}|}
\hline
RSA & The RSA algorithm is used to encrypt a key and send it to the peer.
The certificate must allow the key to be used for encryption.
\\
\hline
DHE\_RSA & The RSA algorithm is used to sign Ephemeral Diffie Hellman
parameters which are send to the peer. The key in the certificate must allow
the key to be used for signing. Note that key exchange algorithms which use
Ephemeral Diffie Hellman parameters, offer perfect forward secrecy.
\\
\hline
DHE\_DSS & The DSS\footnote{DSS stands for Digital Signature Standard} algorithm is used to sign Ephemeral Diffie Hellman
parameters which are send to the peer. 
\\
\hline
\end{tabular}

\caption{Key exchange algorithms for OpenPGP and X.509 certificates.}
\label{fig:cert}

\end{figure}

\section{Authentication using OpenPGP keys}
This authentication method is part of the certificate authentication
method in \gnutls{}. All the key exchange methods shown in \hyperref{figure}{figure }{}{fig:cert} are
available in OpenPGP authentication. 

\section{Authentication using OpenPGP\index{OpenPGP keys} keys}
\label{sec:pgp}

This authentication method is part of the certificate authentication
method in \gnutls{}. All the key exchange methods shown in \hyperref{figure}{figure }{}{fig:cert} are
available in OpenPGP authentication. The \gnutls{}'s implementation is based on the
\cite{TLSPGP} proposal.

See \ref{pgp:trust} on page \pageref{pgp:trust} for more information 
about the OpenPGP trust model.




\section{Anonymous authentication}
The anonymous key exchange perform encryption but there is no indication of the 
identity of the peer. This kind of authentication is vulnerable to man in the middle attack, 
but this protocol can be used even if there is no prior communication or common trusted
parties with the peer. Unless really required, do not use anonymous authentication.
Available key exchange methods are shown in \hyperref{figure}{figure }{}{fig:anon}.

\begin{figure}[hbtp]
\begin{tabular}{|l|p{9cm}|}

\hline
ANON\_DH & This algorithm exchanges Diffie Hellman parameters. 
\\
\hline
\end{tabular}

\caption{Supported anonymous key exchange algorithms}
\label{fig:anon}

\end{figure}

\section{Authentication using SRP}
Authentication using the SRP\footnote{SRP stands for Secure Password Protocol and 
is described in RFC2945. The SRP key exchange is not a part of the \tlsI{} protocol}
is actually password authentication, since the two peers are identified by the knowledge of a password. 
This protocol also offers protection against off-line attacks, such as password 
file stealing. 
This is achieved since SRP does not use the plain password to perform authentication, but something called a 
verifier. The verifier is $g^{x}mod(n)$ and $x$ is a value calculated
from the username and the password. 
\par SRP is normaly used with a SHA based hash function, to calculate
the value of $x$. In \gnutls{} in addition to original SHA hash function,
a hash function based on blowfish crypt is also supported. The blowfish
crypt function has the property of variable complexity, thus the
verifier may resist future attacks based on computational power, by just increasing
the complexity of the function ---also called ``the cost''.
\par The advantage of SRP authentication, over other proposed secure password 
authentication schemas, is that SRP does not require the server to hold
the user's password. This kind of protection is similar to the one used traditionaly
in the \emph{UNIX} ``passwd'' file, where the contents of this file did not cause
harm to the system security if they were revealed.
\par
Available key exchange methods are shown in \hyperref{figure}{figure }{}{fig:srp}.

\begin{figure}[hbtp]
\begin{tabular}{|l|p{9cm}|}

\hline
SRP & Authentication using the SRP protocol. 
\\
\hline
\end{tabular}

\caption{Supported SRP key exchange algorithms}
\label{fig:srp}

\end{figure}

\subsection{The included ``srpcrypt'' program}
The ``gnutls-srpcrypt'' is a very simple program that emulates the programs in the libsrp 
found in \htmladdnormallink{http://srp.stanford.edu}{http://srp.stanford.edu}.
It is intended for use in places where you don't expect srp
authentication to be the performed to system users.

Traditionaly libsrp used two files. One called 'tpasswd' which holds usernames 
and verifiers, and 'tpasswd.conf' which holds generators and primes.
\par
How to use srpcrypt:

\begin{itemize}


\item To create tpasswd.conf which holds the g and n values for
SRP protocol (generator and a large prime), run:
\begin{verbatim}
$ gnutls-srpcrypt --create_conf /etc/tpasswd.conf
\end{verbatim}


\item This command will create /etc/tpasswd and will add user 'test' (you will also
be prompted for a password). Verifiers are stored by default in the
way libsrp expects (using a modified SHA()). However using this tool you 
may also specify blowfish srpcrypt instead of SHA (see --crypt parameter).
\begin{verbatim}
$ gnutls-srpcrypt --passwd /etc/tpasswd \ 
    --passwd_conf /etc/tpasswd.conf -u test
\end{verbatim}

\item This command will check against a password. If the password matches 
the one in /etc/tpasswd you will get an ok.
\begin{verbatim}
$ gnutls-srpcrypt --passwd /etc/tpasswd \ 
    --passwd_conf /etc/tpasswd.conf --verify -u test
\end{verbatim}

\end{itemize}
