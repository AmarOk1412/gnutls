\chapter{Authentication methods}

The \tls{} protocol provides confidentiality and encryption, but
also offers authentication, which is a prerequisite
for a secure connection. 
The available authentication methods in \gnutls{} are:
\begin{enumerate}
 \item Certificate authentication
 \item Anonymous authentication
 \item SRP authentication
\end{enumerate}

\section{Certificate authentication}

% x.509 section
\subsection*{Authentication using X.509\index{X.509 certificates} certificates}

X.509 certificates contain the public parameters, 
of a public key algorithm, and an authority's signature, which proves the
authenticity of the parameters.
See section \ref{x509:trust} on page \pageref{x509:trust} for more information
on X.509 protocols.

% openpgp section

\subsection*{Authentication using OpenPGP\index{OpenPGP!Keys} keys}
\label{sec:pgp}

OpenPGP keys also contain public parameters of a public key algorithm, and
signatures from several other parties. Depending on whether a signer is
trusted the key is considered trusted or not.
\gnutls{}'s OpenPGP authentication implementation is based on the
\cite{TLSPGP} proposal.

See \ref{pgp:trust} on page \pageref{pgp:trust} for more information 
about the OpenPGP trust model. For a more detailed introduction to OpenPGP
and GnuPG see \cite{GPGH}.

\subsection*{Using certificate authentication}

In \gnutls{} both the OpenPGP and X.509 certificates are part of the
certificate authentication and thus are handled using a common API.
\par
When using certificates the server is required
to have at least one certificate and private key pair. A client
may or may not have such a pair. The certificate and key pair
should be loaded, before any \tls{} session is initialized,
in a certificate credentials structure. This should be done by using
\printfunc{gnutls_certificate_set_x509_key_file}{gnutls\_certificate\_set\_x509\_key\_file}
or
\printfunc{gnutls_certificate_set_openpgp_key_file}{gnutls\_certificate\_set\_openpgp\_key\_file}
depending on the certificate type. 
In the X.509 case, the functions will also accept and use a certificate list 
that leads to a trusted authority. The certificate list must be ordered in such
way that every certificate certifies the one before it. The trusted authority's
certificate need not to be included, since the peer should possess it already.
\par
As an alternative, a callback may be used
so the server or the client specify the certificate and the key at the handshake time.
That callback can be set using the functions:
\begin{itemize}
\item \printfunc{gnutls_certificate_server_set_retrieve_function}{gnutls\_certificate\_server\_set\_retrieve\_function}
\item \printfunc{gnutls_certificate_client_set_retrieve_function}{gnutls\_certificate\_client\_set\_retrieve\_function}
\end{itemize}
Certificate verification is possible by loading the trusted authorities
into the credentials structure by using
\printfunc{gnutls_certificate_set_x509_trust_file}{gnutls\_certificate\_set\_x509\_trust\_file}
or
\printfunc{gnutls_certificate_set_openpgp_keyring_file}{gnutls\_certificate\_set\_openpgp\_keyring\_file}
for openpgp keys. Note however that the peer's certificate is not automatically verified,
you should call \printfunc{gnutls_certificate_verify_peers}{gnutls\_certificate\_verify\_peers},
after a successful handshake,
to verify the signatures of the certificate. An alternative way, which reports
a more detailed verification output, is to use
\printfunc{gnutls_certificate_get_peers}{gnutls\_certificate\_get\_peers} to obtain
the raw certificate of the peer and verify it using the functions discussed in
section \ref{x509:trust} on page \pageref{x509:trust}. 

\par
In a handshake, the negotiated cipher suite depends on the 
certificate's parameters, so not all key exchange methods will be available
with some certificates. \gnutls{} will disable ciphersuites that are not compatible with the key, or
the enabled authentication methods. For example keys marked as sign-only, will not be able to
access the plain RSA ciphersuites, but only the DHE\_RSA ones. It is
recommended not to use RSA keys for both signing and encryption. If possible
use the same key for the DHE\_RSA and RSA\_EXPORT ciphersuites, which use signing,
and a different key for the plain RSA ciphersuites, which use encryption.
All the key exchange methods shown in \hyperref{figure}{figure }{}{fig:cert} are
available in certificate authentication. 

Note that the DHE key exchange methods require Diffie Hellman parameters
to be generated and associated with a credentials structure. The RSA-EXPORT
method requires 512 bit RSA parameters, which should also be generated
and associated with the credentials structure. See the functions:
\begin{itemize}
\item \printfunc{gnutls_dh_params_generate2}{gnutls\_dh\_params\_generate2}
\item \printfunc{gnutls_certificate_set_dh_params}{gnutls\_certificate\_set\_dh\_params}
\item \printfunc{gnutls_rsa_params_generate2}{gnutls\_rsa\_params\_generate2}
\item \printfunc{gnutls_certificate_set_rsa_export_params}{gnutls\_certificate\_set\_rsa\_export\_params}
\end{itemize}


\begin{figure}[hbtp]
\index{Key exchange algorithms}
\begin{tabular}{|l|p{9cm}|}
\hline
RSA & The RSA algorithm is used to encrypt a key and send it to the peer.
The certificate must allow the key to be used for encryption.
\\
\hline
RSA\_EXPORT & The RSA algorithm is used to encrypt a key and send it to the peer.
In the EXPORT algorithm, the server signs temporary RSA parameters of 512
bits -- which are considered weak -- and sends them to the client.
\\
\hline
DHE\_RSA & The RSA algorithm is used to sign Ephemeral Diffie Hellman
parameters which are sent to the peer. The key in the certificate must allow
the key to be used for signing. Note that key exchange algorithms which use
Ephemeral Diffie Hellman parameters, offer perfect forward secrecy. That means
that even if the private key used for signing is compromised, it cannot be
used to reveal past session data.
\\
\hline
DHE\_DSS & The DSS algorithm is used to sign Ephemeral Diffie Hellman
parameters which are sent to the peer. The certificate must contain DSA
parameters to use this key exchange algorithm. DSS stands for Digital Signature
Standard.
\\
\hline
\end{tabular}

\caption{Key exchange algorithms for OpenPGP and X.509 certificates.}
\label{fig:cert}

\end{figure}




\section{Anonymous authentication\index{Anonymous authentication}}
The anonymous key exchange perform encryption but there is no indication of 
the identity of the peer. This kind of authentication is vulnerable to a
man in the middle attack, 
but this protocol can be used even if there is no prior communication and
trusted parties with the peer, or when full anonymity is required.
Unless really required, do not use anonymous authentication.
Available key exchange methods are shown in \hyperref{figure}{figure }{}{fig:anon}.
\par
Note that the key exchange methods for anonymous authentication
require Diffie Hellman parameters to be generated and associated with an
anonymous credentials structure. 

\begin{figure}[hbtp]
\begin{tabular}{|l|p{9cm}|}

\hline
ANON\_DH & This algorithm exchanges Diffie Hellman parameters. 
\\
\hline
\end{tabular}

\caption{Supported anonymous key exchange algorithms}
\label{fig:anon}

\end{figure}

\section{Authentication using SRP\index{SRP authentication}}

Authentication using the SRP\footnote{SRP stands for Secure Remote Password and 
is described in \cite{RFC2945}. The SRP key exchange is an extension to the \tlsI{} protocol}
protocol is actually password authentication. The two peers can be identified using a
single password, or there can be combinations where the client is 
authenticated using SRP and the server using a certificate.
\par
The advantage of SRP authentication, over other proposed secure password 
authentication schemas, is that SRP does not require the server to hold
the user's password. This kind of protection is similar to the one used traditionally
in the \emph{UNIX} ``passwd'' file, where the contents of this file did not cause
harm to the system security if they were revealed.
The SRP needs instead of the plain password something called a verifier, 
which is calculated using the user's password, and if stolen cannot
be used to impersonate the user. See \cite{TOMSRP} for a detailed description
of the SRP protocol, and for the Stanford SRP libraries.

\par
The implementation in \gnutls{} is based on paper \cite{TLSSRP}.
The available key exchange methods are shown in \hyperref{figure}{figure }{}{fig:srp}.

\begin{figure}[hbtp]
\begin{tabular}{|l|p{9cm}|}

\hline
SRP & Authentication using the SRP protocol. 
\\
\hline
SRP\_DSS & Client authentication using the SRP protocol. Server is 
authenticated using a certificate with DSA parameters.
\\
\hline
SRP\_RSA & Client authentication using the SRP protocol. Server is 
authenticated using a certificate with RSA parameters.
\\
\hline
\end{tabular}

\caption{Supported SRP key exchange algorithms}
\label{fig:srp}

\end{figure}

If clients supporting SRP know the username and password before the connection,
should initialize the client credentials and call the
function \printfunc{gnutls_srp_set_client_credentials}{gnutls\_srp\_set\_client\_credentials}.
Alternatively they could specify a callback function by using the
function \printfunc{gnutls_srp_set_client_credentials_function}{gnutls\_srp\_set\_client\_credentials\_function}.
This has the advantage that allows probing the server for SRP support.
In that case the callback function will be called twice per handshake.
The first time is before the ciphersuite is negotiated, and 
if the callback returns a negative error code, the callback will be 
called again if SRP has been negotiated. 
This uses a special TLS-SRP handshake idiom in order to avoid, in
interactive applications, to ask the user for SRP password and username 
if the server does not negotiate an SRP ciphersuite.
\par
In server side the default behaviour of \gnutls{} is to read the usernames 
and SRP verifiers from password files. These password files are the ones used
by the \emph{srp libraries} and can be specified using the
\printfunc{gnutls_srp_set_server_credentials_file}{gnutls\_srp\_set\_server\_credentials\_file}.
If a different password file format is to be used, then the 
function \printfunc{gnutls_srp_set_server_credentials_function}{gnutls\_srp\_set\_server\_credentials\_function},
should be called, in order to set an appropriate callback.
\par
Some helper functions such as
\begin{itemize}
\item \printfunc{gnutls_srp_verifier}{gnutls\_srp\_verifier}
\item \printfunc{gnutls_srp_base64_encode}{gnutls\_srp\_base64\_encode}
\item \printfunc{gnutls_srp_base64_decode}{gnutls\_srp\_base64\_decode}
\end{itemize}
are included in \gnutls{}, and may be used to generate, and maintain
SRP verifiers, and password files. 
A program to manipulate the required parameters 
for SRP authentication is also included. See section \ref{srptool} on 
page \pageref{srptool} for more information.



\section{Authentication and credentials}
In \gnutls{} every key exchange method is associated with a
credentials type. So in order to enable to enable a specific
method, the corresponding credentials type should be initialized
and set using \printfunc{gnutls_credentials_set}{gnutls\_credentials\_set}.
A mapping is shown in \hyperref{figure}{figure }{}{fig:kxcred}.

\begin{figure}[hbtp]
\begin{tabular}{|l|l|p{4.5cm}|}

\hline
\bf{Key exchange} & \bf{Client credentials} & \bf{Server credentials}
\\
\hline
\hline
KX\_RSA &&
\\
\cline{1-1}
KX\_DHE\_RSA & CRD\_CERTIFICATE & CRD\_CERTIFICATE
\\
\cline{1-1}
KX\_DHE\_DSS &&
\\
\cline{1-1}
KX\_RSA\_EXPORT &&
\\
\hline
KX\_SRP\_RSA & CRD\_SRP & CRD\_SRP
\\
\cline{1-1}
KX\_SRP\_DSS && CRD\_CERTIFICATE
\\
\hline
KX\_SRP & CRD\_SRP & CRD\_SRP
\\
\hline
KX\_ANON\_DH & CRD\_ANON & CRD\_ANON
\\
\hline
\end{tabular}

\caption{Key exchange algorithms and the corresponding credential types}
\label{fig:kxcred}

\end{figure}



\section{Parameters stored in credentials}

Several parameters such as the ones used for Diffie-Hellman authentication
are stored within the credentials structures, so all sessions can access
them. Those parameters are stored in structures such as {\bf gnutls\_dh\_params}
and {\bf gnutls\_rsa\_params}, and functions like 
\printfunc{gnutls_certificate_set_dh_params}{gnutls\_certificate\_set\_dh\_params}
and
\printfunc{gnutls_certificate_set_rsa_export_params}{gnutls\_certificate\_set\_rsa\_export\_params}
can be used to associate those parameters with the given credentials structure.
\par
Since those parameters need to be renewed from time to time and a 
global structure such as the credentials, may not be easy to modify
since it is accessible by all sessions, an alternative interface is
available using a callback function.
This can be set using the
\printfunc{gnutls_certificate_set_params_function}{gnutls\_certificate\_set\_params\_function}.
An example is shown below.

\begin{verbatim}
#include <gnutls.h>

gnutls_rsa_params rsa_params;
gnutls_dh_params dh_params;

/* This function will be called once a session requests DH
 * or RSA parameters. The parameters returned (if any) will
 * be used for the first handshake only.
 */
static int get_params( gnutls_session session, gnutls_params_type type,
        gnutls_params_st *st)
{
   if (type == GNUTLS_PARAMS_RSA_EXPORT)
      st->params.rsa_export = rsa_params;
   else if (type == GNUTLS_PARAMS_DH)
      st->params.dh = dh_params;
   else return -1;

   st->type = type;
   /* do not deinitialize those parameters.
    */
   st->deinit = 0;

   return 0;
}

int main()
{
   gnutls_certificate_credentials cert_cred;

   initialize_params();

   /* ...
    */

   gnutls_certificate_set_params_function( cert_cred, get_params);

}
\end{verbatim}
