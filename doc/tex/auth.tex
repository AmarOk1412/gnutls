\chapter{Authentication methods}
\par
The following authentication schemas are supported in \gnutls:
\begin{enumerate}
 \item Certificate authentication
 \item Anonymous authentication
 \item SRP authentication
\end{enumerate}

% x.509 section
\section{Authentication using X.509\index{X.509 certificates} certificates}

This authentication method is part of the certificate authentication
method in \gnutls{}.
The X.509 protocols rely on a hierarchical trust model. In this trust model
Certification Authorities (CAs) are used to certify entities.
Usually more than one certification authorities exist, and certification
authorities may certify other authorities to issue certificates as well,
following a hierachical model. 
One needs to trust one or more CAs for his secure
communications. In that case only the certificates issued by the trusted
authorities are acceptable. 
\par
X.509 certificates contain the public parameters, 
of a public key algorithm, and the authority's signature, which proves the
authenticity of the parameters.
\par
The key exchange methods shown in \hyperref{figure}{figure }{}{fig:cert} are
available in X.509 authentication. 

\par The use of X.509 certificates requires some functions which will 
assist in parsing them. \gnutls{} includes functions which extract 
parameters from given X.509 certificates. Some of them are:
\begin{itemize}
\item \printfunc{gnutls_x509_extract_certificate_dn}{gnutls\_x509\_extract\_certificate\_dn}
\item \printfunc{gnutls_x509_extract_certificate_serial}{gnutls\_x509\_extract\_certificate\_serial}
\item \printfunc{gnutls_x509_extract_certificate_subject_alt_name}{gnutls\_x509\_extract\_certificate\_subject\_alt\_name}
\end{itemize}

Given the complexity of the X.509 protocols we do not expect these limited 
functions to cover every need. Thus a function which exports X.509 certificates
to an XML form is provided. See 
\printfunc{gnutls_x509_get_certificate_xml}{gnutls\_x509\_get\_certificate\_xml}.

\par
Verifying certificate\index{Verifying certificate paths} paths is also important in X.509 authentication.
For this purpose you can use the
\printfunc{gnutls_x509_verify_certificate}{gnutls\_x509\_verify\_certificate}
function. A more generic one is also provided and can be used with all
of the certificate authentication methods, but is limited to a session. See the
\printfunc{gnutls_certificate_verify_peers}{gnutls\_certificate\_verify\_peers}
function.

\par
Note that \gnutls{} is not a generic purpose X.509 toolkit\footnote{Aegypten is such a toolkit. See 
\htmladdnormallink{http://www.gnupg.org/aegypten/}{http://www.gnupg.org/aegypten/}}. 
\gnutls{} only includes the required,
in order to use the TLS ciphersuites which require X.509 certificates.



\begin{figure}[hbtp]
\index{Key exchange algorithms}
\begin{tabular}{|l|p{9cm}|}
\hline
RSA & The RSA algorithm is used to encrypt a key and send it to the peer.
The certificate must allow the key to be used for encryption.
\\
\hline
RSA\_EXPORT & The RSA algorithm is used to encrypt a key and send it to the peer.
In the EXPORT algorithm, the server signs temporary RSA parameters of 512
bits -- which are considered weak -- and sends them to the client.
\\
\hline
DHE\_RSA & The RSA algorithm is used to sign Ephemeral Diffie Hellman
parameters which are sent to the peer. The key in the certificate must allow
the key to be used for signing. Note that key exchange algorithms which use
Ephemeral Diffie Hellman parameters, offer perfect forward secrecy.
\\
\hline
DHE\_DSS & The DSS\footnote{DSS stands for Digital Signature Standard} algorithm is used to sign Ephemeral Diffie Hellman
parameters which are sent to the peer. 
\\
\hline
\end{tabular}

\caption{Key exchange algorithms for OpenPGP and X.509 certificates.}
\label{fig:cert}

\end{figure}


% openpgp section

\section{Authentication using OpenPGP\index{OpenPGP keys} keys}
\label{sec:pgp}

This authentication method is part of the certificate authentication
method in \gnutls{}. All the key exchange methods shown in \hyperref{figure}{figure }{}{fig:cert} are
available in OpenPGP authentication. The \gnutls{}'s implementation is based on the
\cite{TLSPGP} proposal.

See \ref{pgp:trust} on page \pageref{pgp:trust} for more information 
about the OpenPGP trust model.




\section{Anonymous authentication\index{Anonymous authentication}}
The anonymous key exchange perform encryption but there is no indication of 
the identity of the peer. This kind of authentication is vulnerable to a
man in the middle attack, 
but this protocol can be used even if there is no prior communication or common trusted
parties with the peer. Unless really required, do not use anonymous authentication.
Available key exchange methods are shown in \hyperref{figure}{figure }{}{fig:anon}.

\begin{figure}[hbtp]
\begin{tabular}{|l|p{9cm}|}

\hline
ANON\_DH & This algorithm exchanges Diffie Hellman parameters. 
\\
\hline
\end{tabular}

\caption{Supported anonymous key exchange algorithms}
\label{fig:anon}

\end{figure}

\section{Authentication using SRP\index{SRP authentication}}
Authentication using the SRP\footnote{SRP stands for Secure Password Protocol and 
is described in RFC2945. The SRP key exchange is not a part of the \tlsI{} protocol}
is actually password authentication, since the two peers are identified by the knowledge of a password. 
This protocol also offers protection against off-line attacks, such as password 
file stealing. 
This is achieved since SRP does not use the plain password to perform authentication, but something called a 
verifier. The verifier is $g^{x}mod(n)$ and $x$ is a value calculated
from the user name and the password. 
\par SRP is normally used with a SHA based hash function, to calculate
the value of $x$. 
\par The advantage of SRP authentication, over other proposed secure password 
authentication schemas, is that SRP does not require the server to hold
the user's password. This kind of protection is similar to the one used traditionally
in the \emph{UNIX} ``passwd'' file, where the contents of this file did not cause
harm to the system security if they were revealed.
\par
Available key exchange methods are shown in \hyperref{figure}{figure }{}{fig:srp}.

\begin{figure}[hbtp]
\begin{tabular}{|l|p{9cm}|}

\hline
SRP & Authentication using the SRP protocol. 
\\
\hline
\end{tabular}

\caption{Supported SRP key exchange algorithms}
\label{fig:srp}

\end{figure}

\gnutls{} includes a program to manipulate the required for SRP
authentication. See section \ref{srpcrypt} on page \pageref{srpcrypt} for
more information.

