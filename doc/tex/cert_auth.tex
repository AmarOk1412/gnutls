\section{Certificate authentication}

% x.509 section
\subsection*{Authentication using X.509\index{X.509 certificates} certificates}

X.509 certificates contain the public parameters, 
of a public key algorithm, and an authority's signature, which proves the
authenticity of the parameters.
See section \ref{x509:trust} on page \pageref{x509:trust} for more information
on X.509 protocols.

% openpgp section

\subsection*{Authentication using OpenPGP\index{OpenPGP!Keys} keys}
\label{sec:pgp}

OpenPGP keys also contain public parameters of a public key algorithm, and
signatures from several other parties. Depending on whether a signer is
trusted the key is considered trusted or not.
\gnutls{}'s OpenPGP authentication implementation is based on the
\cite{TLSPGP} proposal.

See \ref{pgp:trust} on page \pageref{pgp:trust} for more information 
about the OpenPGP trust model. For a more detailed introduction to OpenPGP
and GnuPG see \cite{GPGH}.

\subsection*{Using certificate authentication}

In \gnutls{} both the OpenPGP and X.509 certificates are part of the
certificate authentication and thus are handled using a common API.
\par
When using certificates the server is required
to have at least one certificate and private key pair. A client
may or may not have such a pair. The certificate and key pair
should be loaded, before any \tls{} session is initialized,
in a certificate credentials structure. This should be done by using
\printfunc{gnutls_certificate_set_x509_key_file}{gnutls\_certificate\_set\_x509\_key\_file}
or
\printfunc{gnutls_certificate_set_openpgp_key_file}{gnutls\_certificate\_set\_openpgp\_key\_file}
depending on the certificate type. As an alternative, a callback may be used
so the server or the client set the certificate at the handshake time.
That callback can be set using
\printfunc{gnutls_certificate_server_set_retrieve_function}{gnutls\_certificate\_server\_set\_retrieve\_function}
or
\printfunc{gnutls_certificate_client_set_retrieve_function}{gnutls\_certificate\_client\_set\_retrieve\_function}
in case of a client.
\par
Certificate verification is possible by loading the trusted authorities
into the credentials structure by using
\printfunc{gnutls_certificate_set_x509_trust_file}{gnutls\_certificate\_set\_x509\_trust\_file}
or
\printfunc{gnutls_certificate_set_openpgp_keyring_file}{gnutls\_certificate\_set\_openpgp\_keyring\_file}
for openpgp keys. Note however that the peer's certificate is not automaticaly verified,
you should call \printfunc{gnutls_certificate_verify_peers}{gnutls\_certificate\_verify\_peers},
after a successful handshake,
to verify the signatures of the certificate. An alternative way, which reports
a more detailed verification output, is to use
\printfunc{gnutls_certificate_get_peers}{gnutls\_certificate\_get\_peers} to obtain
the raw certificate of the peer and verify it using the functions discussed in
section \ref{x509:trust} on page \pageref{x509:trust}. 

\par
In a handshake, the negotiated key exchange method depends on the 
certificate's parameters, so not all key exchange methods will be available
with some certificates. That is a certificate with DSA parameters will not
be able to use the RSA key exchange method.
All the key exchange methods shown in \hyperref{figure}{figure }{}{fig:cert} are
available in certificate authentication. 
Note that the DHE key exchange methods require Diffie Hellman parameters
to be generated and associated with a credentials structure. The RSA-EXPORT
method requires 512 bit RSA parameters, which should also be generated
and associated with the credentials structure. See the functions:
\begin{itemize}
\item \printfunc{gnutls_dh_params_generate2}{gnutls\_dh\_params\_generate2}
\item \printfunc{gnutls_certificate_set_dh_params}{gnutls\_certificate\_set\_dh\_params}
\item \printfunc{gnutls_rsa_params_generate2}{gnutls\_rsa\_params\_generate2}
\item \printfunc{gnutls_certificate_set_rsa_export_params}{gnutls\_certificate\_set\_rsa\_export\_params}
\end{itemize}

\begin{figure}[hbtp]
\index{Key exchange algorithms}
\begin{tabular}{|l|p{9cm}|}
\hline
RSA & The RSA algorithm is used to encrypt a key and send it to the peer.
The certificate must allow the key to be used for encryption.
\\
\hline
RSA\_EXPORT & The RSA algorithm is used to encrypt a key and send it to the peer.
In the EXPORT algorithm, the server signs temporary RSA parameters of 512
bits -- which are considered weak -- and sends them to the client.
\\
\hline
DHE\_RSA & The RSA algorithm is used to sign Ephemeral Diffie Hellman
parameters which are sent to the peer. The key in the certificate must allow
the key to be used for signing. Note that key exchange algorithms which use
Ephemeral Diffie Hellman parameters, offer perfect forward secrecy.
\\
\hline
DHE\_DSS & The DSS algorithm is used to sign Ephemeral Diffie Hellman
parameters which are sent to the peer. The certificate must contain DSA
parameters to use this key exchange algorithm. DSS stands for Digital Signature
Standard.
\\
\hline
\end{tabular}

\caption{Key exchange algorithms for OpenPGP and X.509 certificates.}
\label{fig:cert}

\end{figure}


