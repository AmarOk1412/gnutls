\section{Introduction}
\par
\gnutls{} is a portable library which implements the \tlsI{} and 
\sslIII{} protocols.
\tls{} stands for 'Transport Layer Security' and is the sucessor of \ssl{}, 
the Secure Sockets Layer protocol designed by Netscape. 

\tlsI{}\footnote{described in {\it RFC 2246}} is an Internet protocol,
defined by {IETF}\footnote{IETF or Internet Engineering Task Force 
is a large open international community of network
designers, operators, vendors, and researchers concerned with the evolution of 
the Internet architecture and the smooth operation of the Internet. It is open to any interested individual.}, 
that provides confidentiality, and authentication layers over any reliable
transport layer.

\par
\gnutls{} implements the above
protocols in a reentrant way. This allows multiple threads of
execution, without the need for critical sections and locks. See
\htmladdnormallink{http://www.gnutls.org/}{http://www.gnutls.org/}
and \htmladdnormallink{http://www.gnu.org/software/gnutls/}{http://www.gnu.org/software/gnutls/} 
for updated versions of the \gnutls{} software and this document.

\par
Currently \gnutls{} implements:
\begin{itemize}
 \item the \tlsI{} and \sslIII{} protocols, without any weak algorithms\footnote{
There are ciphersuites in \tlsI{} that are considered weak. These
ciphersuites are deliberately weak in order to be able to export encryption
software from some countries.}
 \item {\bf X.509} Public Key Infrastructure.
 \item {\bf OpenPGP} Public Key Infrastructure.
 \item {\bf SRP} for \tls{} authentication.
 \item \tls{} {\bf Extension mechanism}.
\end{itemize}
