\section{Error handling\index{Error handling}}
\par
In \gnutls{} most functions return an integer type as a result.
In almost all cases a zero or a positive number means success, and
a negative number indicates failure, or a situation that some
action has to be taken. Thus negative error codes may be fatal
or not. 
\par 
Fatal errors terminate the connection immediately and
further sends and receives will be disallowed. An example of
a fatal error code is GNUTLS\_E\_DECRYPTION\_FAILED. Non-fatal errors
may warn about something, ie a warning alert was received, or
indicate the some action has to be taken. This is the case with
the error code GNUTLS\_E\_REHANDSHAKE returned by 
\printfunc{gnutls_record_recv}{gnutls\_record\_recv}.
This error code indicates that the server requests a re-handshake. The client
may ignore this request, or may reply with an alert.
You can test if an error code is a fatal one by using the
\printfunc{gnutls_error_is_fatal}{gnutls\_error\_is\_fatal}.
\par
If any non fatal errors, that require an action, are to be returned by a
function, these error codes will be documented
in the function's reference. All the error codes are documented
in appendix \ref{ap:error_codes} on page \pageref{ap:error_codes}.


