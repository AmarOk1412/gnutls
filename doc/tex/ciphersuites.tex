\newpage
\section{TLS Cipher suites}
\par 
\tls 1.0 supports ciphersuites like {\bf TLS\_X509PKI\_DHE\_RSA\_WITH\_3DES\_CBC\_SHA}.
These ciphersuites contain three parameters:
\begin{itemize}
\item The key authentication method (X.509 PKI in the example)
\item The key exchange algorithm (DHE\_RSA in the example)
\item The Symmetric encryption algorithm and mode (3DES\_CBC in this
example)
\item The MAC\footnote{MAC stands for Message Authentication Code. It can
be described as a keyed hash algorithm. See RFC2104} algorithm used for authentication.
MAC\_SHA is used in the above example.
\end{itemize}


\begin{figure}[hbtp]
\begin{tabular}{|l|p{9cm}|}

\hline
RSA & The RSA algorithm is used to encrypt a key and send it to the peer.
The certificate must allow the key to be used for encryption.
\\
\hline
DHE\_RSA & The RSA algorithm is used to sign Ephemeral Diffie Hellman
parameters which are send to the peer. The key in the certificate must allow
the key to be used for signing 
\\
\hline
DHE\_DSS & The DSS\footnote{DSS stands for Digital Signature Standard} algorithm is used to sign Ephemeral Diffie Hellman
parameters which are send to the peer. Currently \gnutls does not support this ciphersuite.
\\
\hline
\end{tabular}

\caption{Supported X.509 key exchange algorithms}

\end{figure}

\begin{figure}[hbtp]
\begin{tabular}{|l|p{9cm}|}

\hline
ANON\_DH & This algorithm exchanges not signed diffie Hellman parameters. That way encryption may 
be performed but there is no indication of the identity of the peer. This kind of
authentication is vulnerable to man in the middle attack, but this protocol
can be used even if there is no prior communication or common trusted
parties with the peer.
Unless really required, do not use anonymous authentication.
\\
\hline
\end{tabular}

\caption{Supported anonymous key exchange algorithms}

\end{figure}


\begin{figure}[hbtp]
\begin{tabular}{|l|p{9cm}|}

\hline
SRP\footnote{SRP stands for Secure Password Protocol and is described in
RFC2945. The SRP key exchange is not a part of the TLS 1.0 protocol} & Authentication using the SRP
protocol. This protocol is actually password authentication, since the two
peers are identified by the knowledge of a password. However this protocol
also offers protection against off-line attacks (password file stealing
etc.). Use SRP authentication if password based authentication is required.
\\
\hline
\end{tabular}

\caption{Supported SRP key exchange algorithms}

\end{figure}




\addvspace{1.5cm}

\begin{figure}[hbtp]
\begin{tabular}{|l|p{9cm}|}

\hline
3DES\_CBC & 3DES\_CBC is the DES block cipher algorithm used with multiple (triple)
encryption (EDE). Has 64 bits block size and is used in CBC mode.
\\
\hline
ARCFOUR & ARCFOUR\footnote{A compatible cipher with RC4 of RSADSI} is a
fast stream cipher.
\\
\hline
AES\_CBC & AES or RIJNDAEL is the block cipher algorithm that replaces the old 
(and insecure\footnote{due to small key length}) DES algorithm. Has
128 bits block size and is used in CBC mode.
\\
\hline
TWOFISH\_CBC & TWOFISH is a block cipher algorithm by Counterpane. Has
128 bits block size and is used in CBC mode.
\\
\hline
\end{tabular}
\caption{Supported cipher algorithms}

\end{figure}



\addvspace{1.5cm}

\begin{figure}[hbtp]
\begin{tabular}{|l|p{9cm}|}

\hline
MAC\_MD5 & MD5 is a hash algorithm by Ron Rivest. Outputs 128 bits of data.
\\
\hline
MAC\_SHA & SHA is a hash algorithm by NSA. Outputs 160 bits of data.
\\
\hline
\end{tabular}
\caption{Supported MAC algorithms}

\end{figure}

