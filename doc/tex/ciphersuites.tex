\newpage
\section{TLS Cipher suites}
\par 
\tlsI supports ciphersuites like {\bf TLS\_X509PKI\_DHE\_RSA\_WITH\_3DES\_CBC\_SHA}.
These ciphersuites contain three parameters:
\begin{itemize}
\item The authentication method (X.509 PKI in the example)\footnote{
This field should be considered as a \gnutls extension. \tlsI only defines
the X.509 PKI authentication method. \gnutls supports other authentication
methods as well.}
\item The key exchange algorithm (DHE\_RSA in the example)
\item The Symmetric encryption algorithm and mode (3DES\_CBC in this
example)
\item The MAC\footnote{MAC stands for Message Authentication Code. It can
be described as a keyed hash algorithm. See RFC2104.} algorithm used for authentication.
MAC\_SHA is used in the above example.
\end{itemize}

The ciphersuite that will be used in the connection is negotiated at
the handshake procedure. However you must note that \tlsI does not always
negotiate the strongest available cipher suite. There are cases where
a man in the middle attacker could make the two entities negotiate
the least secure method they support. For that reason do not enable
ciphers and algorithms that you consider weak.

\addvspace{1.5cm}

