\newpage
\section{TLS Cipher suites}
\par 
\tls 1.0 supports ciphersuites like {\bf TLS\_DHE\_RSA\_WITH\_3DES\_CBC\_SHA}.
These ciphersuites contain three parameters:
\begin{itemize}
\item The key exchange algorithm (DHE\_RSA in the example)
\item The Symmetric encryption algorithm and mode (3DES\_CBC in this
example)
\item The MAC\footnote{MAC stands for Message Authentication Code. It can
be described as a keyed hash algorithm} algorithm used for authentication.
MAC-SHA is used in the above example.
\end{itemize}

\par The key exchange algorithms currently in the \tls protocol
are:

\par
\begin{tabular}{|l|p{9cm}|}

\hline
RSA & The RSA algorithm is used to encrypt a key and send it to the peer.
The certificate must allow the key to be used for encryption.
\\
\hline
DHE\_RSA & The RSA algorithm is used to sign Ephemeral Diffie Hellman
parameters which are send to the peer. The key in the certificate must allow
the key to be used for signing 
\\
\hline
DHE\_DSS & The DSS\footnote{DSS stands for Digital Signature Standard} algorithm is used to sign Ephemeral Diffie Hellman
parameters which are send to the peer. Currently \gnutls does not support this ciphersuite.
\\
\hline
DH\_DSS & Static Diffie Hellman parameters signed by a DSS certificate.
Currently \gnutls does not support this ciphersuite.
\\
\hline
DH\_RSA & Static Diffie Hellman parameters signed by an RSA certificate.
Currently \gnutls does not support this ciphersuite.
\\
\hline
DH\_ANON & Diffie Hellman parameters not signed. This key exchange method is
vulnerable to man in the middle attack.
\\
\hline
SRP\footnote{The SRP key exchange is not a part of the TLS 1.0 protocol} & Authentication using the SRP\footnote{SRP stands for Secure Remote Password protocol}
protocol
\\
\hline
\end{tabular}

\addvspace{1.5cm}
\par

 The Bulk Cipher algorithms used in \gnutls are: 
\par
\begin{tabular}{|l|p{9cm}|}

\hline
3DES\_CBC & 3DES\_CBC is the DES block cipher algorithm used with multiple (triple)
encryption (EDE). Has 64 bits block size and is used in CBC mode.
\\
\hline
ARCFOUR & ARCFOUR\footnote{A compatible cipher with RC4 of RSADSI} is a
fast stream cipher.
\\
\hline
AES\_CBC & AES or RIJNDAEL is the block cipher algorithm that replaces the old 
(and insecure\footnote{due to small key length}) DES algorithm. Has
128 bits block size and is used in CBC mode.
\\
\hline
TWOFISH\_CBC & TWOFISH is a block cipher algorithm by Counterpane. Has
128 bits block size and is used in CBC mode.
\\
\hline
\end{tabular}

\addvspace{1.5cm}
\par
 The MAC\footnote{see RFC2104} algorithms used in \gnutls are: 
\par

\begin{tabular}{|l|p{9cm}|}

\hline
MAC\_MD5 & MD5 is a hash algorithm by Ron Rivest. Outputs 128 bits of data.
\\
\hline
MAC\_SHA & SHA is a hash algorithm by NSA. Outputs 160 bits of data.
\\
\hline
\end{tabular}

