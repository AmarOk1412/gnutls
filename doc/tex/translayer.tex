\section{Transport Layer}
\par
\gnutls can be used above any transport layer. To do this you will only 
need to set up the 
\hyperref{gnutls\_transport\_set\_push\_function()}{gnutls\_transport\_set\_push\_function() (see Section }{
for more information)}{gnutls_transport_set_push_function} and
\hyperref{gnutls\_transport\_set\_pull\_function()}{gnutls\_transport\_set\_pull\_function() (see Section }{
for more information)}{gnutls_transport_set_pull_function}
functions. These functions will then be used by gnutls in order to send and receive data.
The functions specified should return -1 on error and probably set errno appropriately.
\gnutls supports EINTR and EAGAIN errno values (This means that appropriate
values will be returned to the caller of the gnutls function).
\par
By default (if the above functions are not called), gnutls will use
the berkeley sockets functions recv() and send(). In this case
gnutls will use some hacks in order for select() to work, thus
making easy to add {\emph TLS} support to existing servers.


