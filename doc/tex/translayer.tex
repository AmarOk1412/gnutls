\section{The transport layer}
\par
\gnutls{} can be used above any reliable transport layer. You may need to
use the functions:
\begin{itemize}
\item \printfunc{gnutls_transport_set_push_func}{gnutls\_transport\_set\_push\_func()}
\item \printfunc{gnutls_transport_set_pull_func}{gnutls\_transport\_set\_pull\_func()}
\end{itemize}
These functions accept a functions as a parameter. The given functions will 
be used by gnutls to send and receive data.
These functions should return -1 on error and should set errno appropriately.
\gnutls{} supports EINTR and EAGAIN errno values. These values are
usually used in non blocking IO and interrupted system calls.
The corresponding values (GNUTLS\_E\_INTERRUPTED, GNUTLS\_E\_AGAIN) 
will be returned to the caller of the gnutls function. \gnutls{} functions
can be resumed (called again), if any of these values is returned.
\par
By default, if none of the above functions are called, gnutls will use
the berkeley sockets functions \emph{recv()} and \emph{send()}. In this case
gnutls will use some hacks in order for \emph{select()} to work, thus
making easy to add \tls{} support to existing servers.
