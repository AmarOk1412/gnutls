\section{The transport layer}
\par
\gnutls{} is not limited to any transport layer, it 
can be used above any transport layer, as long as, it is a reliable 
one. A set of functions is provided and its purpose is to load
to gnutls the required callbacks to access the transport layer.

\begin{itemize}
\item \printfunc{gnutls_transport_set_push_func}{gnutls\_transport\_set\_push\_func}
\item \printfunc{gnutls_transport_set_pull_func}{gnutls\_transport\_set\_pull\_func}
\item \printfunc{gnutls_transport_set_ptr}{gnutls\_transport\_set\_ptr}
\end{itemize}

These functions accept a callback function as a parameter.
The callback functions should return -1 on error and should set errno 
appropriately.
\par
\gnutls{} currently only interprets the EINTR and EAGAIN errno values. 
These values are usually used in non blocking IO and interrupted system calls.
The corresponding values (GNUTLS\_E\_INTERRUPTED, GNUTLS\_E\_AGAIN) 
will be returned to the caller of the gnutls function. \gnutls{} functions
can be resumed (called again), if any of these values is returned.
\par
By default, if none of the above functions are called, gnutls will use
the Berkeley Sockets functions \emph{recv()} and \emph{send()}. In this case
gnutls will use some hacks in order for \emph{select()} to work, thus
making easy to add \tls{} support to existing TCP/IP servers.
