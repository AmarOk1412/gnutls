\section{Transport Layer}
\par
\gnutls{} can be used above any reliable transport layer. To do this you will only 
need to set up the 
\hyperref{gnutls\_transport\_set\_push\_func()}{gnutls\_transport\_set\_push\_func() (see Section }{)}{gnutls_transport_set_push_func} and
\hyperref{gnutls\_transport\_set\_pull\_func()}{gnutls\_transport\_set\_pull\_func() (see Section }{)}{gnutls_transport_set_pull_func}
functions. These functions will then be used by gnutls in order to send and receive data.
The functions specified should return -1 on error and should set errno appropriately.
\gnutls{} supports EINTR and EAGAIN errno values. These values are
usually used in non blocking IO and interrupted system calls.
The corresponding values (GNUTLS\_E\_INTERRUPTED, GNUTLS\_E\_AGAIN) 
will be returned to the caller of the gnutls function. \gnutls{} functions
can be resumed (called again), if any of these values is returned.
\par
By default, if none of the above functions are called, gnutls will use
the berkeley sockets functions \emph{recv()} and \emph{send()}. In this case
gnutls will use some hacks in order for \emph{select()} to work, thus
making easy to add \tls support to existing servers.


