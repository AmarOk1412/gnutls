\section{The transport layer}
\par
\tls{} is not limited to any transport layer, it 
can be used above any transport layer, as long as, it is a reliable 
one. A set of functions is provided and its purpose is to load
to gnutls the required callbacks to access the transport layer.

\begin{itemize}
\item \printfunc{gnutls_transport_set_push_function}{gnutls\_transport\_set\_push\_function}
\item \printfunc{gnutls_transport_set_pull_function}{gnutls\_transport\_set\_pull\_function}
\item \printfunc{gnutls_transport_set_ptr}{gnutls\_transport\_set\_ptr}
\end{itemize}

These functions accept a callback function as a parameter.
The callback functions should return the number of bytes written, or -1 on 
error and should set errno appropriately.
\par
\gnutls{} currently only interprets the EINTR and EAGAIN errno values and
returns the corresponding gnutls error codes GNUTLS\_E\_INTERRUPTED and
GNUTLS\_E\_AGAIN.
These values are usually returned by interrupted system calls, or 
when non blocking IO is used. All \gnutls{} functions
can be resumed (called again), if any of these error codes is returned.
The error codes above refer to the system call, not the \gnutls{} function,
since signals do not interrupt gnutls' functions.

\par
By default, if the transport functions are not set, \gnutls{} will use
the Berkeley Sockets functions. In this case
gnutls will use some hacks in order for \emph{select()} to work, thus
making easy to add \tls{} support to existing TCP/IP servers.
