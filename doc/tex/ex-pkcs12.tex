\begin{verbatim}

#include <stdio.h>
#include <stdlib.h>
#include <gnutls/gnutls.h>
#include <gnutls/pkcs12.h>

#define OUTFILE "out.p12"

/* This function will write a pkcs12 structure into a file.
 * cert: is a DER encoded certificate
 * pkcs8_key: is a PKCS #8 encrypted key
 * password: is the password used to encrypt the PKCS #12 packet.
 */
int write_pkcs12(const gnutls_datum * cert, const gnutls_datum * pkcs8_key,
                 const char *password)
{
   gnutls_pkcs12 pkcs12;
   int ret, bag_index;
   gnutls_pkcs12_bag bag, key_bag;
   char pkcs12_struct[10 * 1024];
   int pkcs12_struct_size;
   FILE *fd;

   /* A good idea might be to use gnutls_x509_privkey_get_key_id()
    * to obtain a unique ID.
    */
   gnutls_datum key_id = { "\x00\x00\x00\x01", 4 };

   gnutls_global_init();
   gnutls_global_set_log_level(2);

   /* Firstly we create two helper bags, which hold the certificate,
    * and the (encrypted) key.
    */

   gnutls_pkcs12_bag_init(&bag);
   gnutls_pkcs12_bag_init(&key_bag);

   ret = gnutls_pkcs12_bag_set_data(bag, GNUTLS_BAG_CERTIFICATE, cert);
   if (ret < 0) {
      fprintf(stderr, "ret: %s\n", gnutls_strerror(ret));
      exit(1);
   }

   /* ret now holds the bag's index.
    */
   bag_index = ret;

   /* Associate a friendly name with the given certificate. Used
    * by browsers.
    */
   gnutls_pkcs12_bag_set_friendly_name(bag, bag_index, "My name");

   /* Associate the certificate with the key using a unique key
    * ID.
    */
   gnutls_pkcs12_bag_set_key_id(bag, bag_index, &key_id);

   gnutls_pkcs12_bag_encrypt(bag, password, 0);

   /* Now the key.
    */

   ret = gnutls_pkcs12_bag_set_data(key_bag,
                                    GNUTLS_BAG_PKCS8_ENCRYPTED_KEY,
                                    &pkcs8_key);
   if (ret < 0) {
      fprintf(stderr, "ret: %s\n", gnutls_strerror(ret));
      exit(1);
   }

   /* Note that since the PKCS #8 key is encrypted we don't
    * bother encrypting the bag.
    */
   bag_index = ret;

   gnutls_pkcs12_bag_set_friendly_name(key_bag, bag_index, "My name");

   gnutls_pkcs12_bag_set_key_id(key_bag, bag_index, &key_id);


   /* The bags were filled. Now create the PKCS #12 structure.
    */
   gnutls_pkcs12_init(&pkcs12);

   /* Insert the two bags in the PKCS #12 structure.
    */

   gnutls_pkcs12_set_bag(pkcs12, bag);
   gnutls_pkcs12_set_bag(pkcs12, key_bag);


   /* Generate a message authentication code for the PKCS #12
    * structure.
    */
   gnutls_pkcs12_generate_mac(pkcs12, password);

   pkcs12_struct_size = sizeof(pkcs12_struct);
   ret =
       gnutls_pkcs12_export(pkcs12, GNUTLS_X509_FMT_DER, pkcs12_struct,
                            &pkcs12_struct_size);
   if (ret < 0) {
      fprintf(stderr, "ret: %s\n", gnutls_strerror(size));
      exit(1);
   }

   fd = fopen(OUTFILE, "w");
   if (fd == NULL) {
      fprintf(stderr, "cannot open file\n");
      exit(1);
   }
   fwrite(pkcs12_struct, 1, pkcs12_struct_size, fd);
   fclose(fd);

   gnutls_pkcs12_bag_deinit(bag);
   gnutls_pkcs12_bag_deinit(key_bag);
   gnutls_pkcs12_deinit(pkcs12);

}

\end{verbatim}
