%\section{Preparation\footnote{This section is heavily based on the `libksba' documentation}}
\section{Preparation}

To use \gnutls{}, you have to perform some changes to your sources and
your build system. The necessary changes are explained in the following
subsections.

\subsection*{Headers}

All the data types and functions of the \gnutls{} library are defined in
the header file `gnutls/gnutls.h'. This must be included in all programs that
make use of the \gnutls{} library.
\par
The extra functionality of the \gnutlse{} library is available by
including the header file `gnutls/extra.h' in your programs.

\subsection*{Version check}
It is often desirable to check that the version of `gnutls' used is indeed
one which fits all requirements.  Even with binary compatibility new
features may have been introduced but due to problem with the dynamic
linker an old version is actually used.  So you may want to check that
the version is okay right after program startup.
See the function \printfunc{gnutls_check_version}{gnutls\_check\_version}


\subsection*{Building the source}

If you want to compile a source file including the `gnutls/gnutls.h' header
file, you must make sure that the compiler can find it in the
directory hierarchy.  This is accomplished by adding the path to the
directory in which the header file is located to the compilers include
file search path (via the -I option).

However, the path to the include file is determined at the time the
source is configured.  To solve this problem, \gnutls{} ships with two small
helper programs \command{libgnutls-config} and \command{libgnutls-extra-config}
that knows about the path to the
include file and other configuration options.  The options that need
to be added to the compiler invocation at compile time are output by
the \option{--cflags} option to \option{libgnutls-config}.  The following
example shows how it can be used at the command line:

\begin{verbatim}
gcc -c foo.c `libgnutls-config --cflags`
\end{verbatim}

Adding the output of \command{libgnutls-config --cflags} to the compilers
command line will ensure that the compiler can find the \gnutls{} header
file.

A similar problem occurs when linking the program with the library.
Again, the compiler has to find the library files.  For this to work,
the path to the library files has to be added to the library search
path (via the -L option).  For this, the option
\option{--libs} to \command{libgnutls-config} can be used.  For
convenience, this option also outputs all other options that are
required to link the program with the \gnutls{} libararies.
The example shows how to link `foo.o'
with the \gnutls{} libraries to a program \emph{foo}.

\begin{verbatim}
gcc -o foo foo.o `libgnutls-config --libs`
\end{verbatim}

Of course you can also combine both examples to a single command by
specifying both options to `libgnutls-config':

\begin{verbatim}
gcc -o foo foo.c `libgnutls-config --cflags --libs`
\end{verbatim}


\section{Multi-threaded applications}

Although the \gnutls{} library is thread safe by design, some parts of the crypto
backend, such as the random generator, are not. Since \emph{libgcrypt 1.1.92}
there was an automatic detection of the thread library used by the
application, so most applications wouldn't need to do any changes to
ensure thread-safety. Due to the unportability of the automatic thread
detection, this was removed from later releases of \emph{libgcrypt}, so
applications have now to register callback functions to ensure proper locking
in sensitive parts of \emph{libgcrypt}. 
\par
There are helper macros to help you properly initialize the libraries.
Examples are shown below.
\begin{itemize}

\item POSIX threads
\begin{verbatim}
#include <gnutls.h>
#include <gcrypt.h>
#include <errno.h>
#include <pthread.h>
GCRY_THREAD_OPTION_PTHREAD_IMPL;

int main() 
{
   /* The order matters.
    */
   gcry_control (GCRYCTL_SET_THREAD_CBS, &gcry_threads_pthread);
   gnutls_global_init();
}
\end{verbatim}

\item GNU PTH threads
\begin{verbatim}
#include <gnutls.h>
#include <gcrypt.h>
#include <errno.h>
#include <pth.h>
GCRY_THREAD_OPTION_PTH_IMPL;

int main() 
{
   gcry_control (GCRYCTL_SET_THREAD_CBS, &gcry_threads_pth);
   gnutls_global_init();
}
\end{verbatim}

\item Other thread packages
\begin{verbatim}
/* The gcry_thread_cbs structure must have been
 * initialized.
 */
static struct gcry_thread_cbs gcry_threads_other = { ... };

int main()
{
   gcry_control (GCRYCTL_SET_THREAD_CBS, &gcry_threads_other);
}
\end{verbatim}
\end{itemize}

