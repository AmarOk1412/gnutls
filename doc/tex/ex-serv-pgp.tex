\begin{verbatim}

#include <stdio.h>
#include <stdlib.h>
#include <errno.h>
#include <sys/types.h>
#include <sys/socket.h>
#include <netinet/in.h>
#include <arpa/inet.h>
#include <string.h>
#include <unistd.h>
#include <gnutls/gnutls.h>
/* Must be linked against gnutls-extra.
 */
#include <gnutls/extra.h>

#define KEYFILE "secret.asc"
#define CERTFILE "public.asc"
#define RINGFILE "ring.gpg"

/* This is a sample TLS 1.0-OpenPGP echo server.
 */


#define SA struct sockaddr
#define SOCKET_ERR(err,s) if(err==-1) {perror(s);return(1);}
#define MAX_BUF 1024
#define PORT 5556               /* listen to 5556 port */
#define DH_BITS 1024

/* These are global */
gnutls_certificate_credentials_t cred;
const int cert_type_priority[2] = { GNUTLS_CRT_OPENPGP, 0 };
gnutls_dh_params_t dh_params;

/* Defined in a previous example */
extern int generate_dh_params( void);
extern gnutls_session_t initialize_tls_session( void);

int main()
{
   int err, listen_sd, i;
   int sd, ret;
   struct sockaddr_in sa_serv;
   struct sockaddr_in sa_cli;
   int client_len;
   char topbuf[512];
   gnutls_session_t session;
   char buffer[MAX_BUF + 1];
   int optval = 1;
   char name[256];

   strcpy(name, "Echo Server");

   /* this must be called once in the program
    */
   gnutls_global_init();

   gnutls_certificate_allocate_credentials( &cred);
   gnutls_certificate_set_openpgp_keyring_file( cred, RINGFILE);

   gnutls_certificate_set_openpgp_key_file( cred, CERTFILE, KEYFILE);

   generate_dh_params();
   
   gnutls_certificate_set_dh_params( cred, dh_params);

   /* Socket operations
    */
   listen_sd = socket(AF_INET, SOCK_STREAM, 0);
   SOCKET_ERR(listen_sd, "socket");

   memset(&sa_serv, '\0', sizeof(sa_serv));
   sa_serv.sin_family = AF_INET;
   sa_serv.sin_addr.s_addr = INADDR_ANY;
   sa_serv.sin_port = htons(PORT);  /* Server Port number */

   setsockopt(listen_sd, SOL_SOCKET, SO_REUSEADDR, &optval, sizeof(int));

   err = bind(listen_sd, (SA *) & sa_serv, sizeof(sa_serv));
   SOCKET_ERR(err, "bind");
   err = listen(listen_sd, 1024);
   SOCKET_ERR(err, "listen");

   printf("%s ready. Listening to port '%d'.\n\n", name, PORT);

   client_len = sizeof(sa_cli);
   for (;;) {
      session = initialize_tls_session();
      gnutls_certificate_type_set_priority(session, cert_type_priority);

      sd = accept(listen_sd, (SA *) & sa_cli, &client_len);

      printf("- connection from %s, port %d\n",
             inet_ntop(AF_INET, &sa_cli.sin_addr, topbuf,
                       sizeof(topbuf)), ntohs(sa_cli.sin_port));

      gnutls_transport_set_ptr( session, (gnutls_transport_ptr_t)sd);
      ret = gnutls_handshake( session);
      if (ret < 0) {
         close(sd);
         gnutls_deinit(session);
         fprintf(stderr, "*** Handshake has failed (%s)\n\n",
                 gnutls_strerror(ret));
         continue;
      }
      printf("- Handshake was completed\n");

      /* see the Getting peer's information example */
      /* print_info(session); */

      i = 0;
      for (;;) {
         bzero(buffer, MAX_BUF + 1);
         ret = gnutls_record_recv( session, buffer, MAX_BUF);

         if (ret == 0) {
            printf
                ("\n- Peer has closed the GNUTLS connection\n");
            break;
         } else if (ret < 0) {
            fprintf(stderr,
                    "\n*** Received corrupted data(%d). Closing the connection.\n\n",
                    ret);
            break;
         } else if (ret > 0) {
            /* echo data back to the client
             */
            gnutls_record_send( session, buffer,
                         strlen(buffer));
         }
      }
      printf("\n");
      gnutls_bye( session, GNUTLS_SHUT_WR); /* do not wait for
                                 * the peer to close the connection.
                                 */

      close(sd);
      gnutls_deinit(session);

   }
   close(listen_sd);

   gnutls_certificate_free_credentials( cred);

   gnutls_global_deinit();

   return 0;

}

\end{verbatim}
