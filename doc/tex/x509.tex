\section{Authentication using X.509 certificates}

This authentication method is part of the certificate authentication
method in \gnutls{}.
The X.509 protocols rely on a hierarchical trust model. In this trust model
Certification Authorities (CAs) are used to certify entities.
Usually more than one certification authorities exist, and certification
authorities may certify other authorities to issue certificates as well,
following a hierachical model. 
One needs to trust one or more CAs for his secure
communications. In that case only the certificates issued by the trusted
authorities are acceptable. 
\par
X.509 certificates contain the public parameters, 
of a public key algorithm, and the authority's signature, which proves the
authenticity of the parameters.
\par
The key exchange methods shown in \hyperref{figure}{figure }{}{fig:cert} are
available in X.509 authentication. 

\par The use of X.509 certificates requires some functions which will 
assist in parsing them. \gnutls{} includes functions which extract 
parameters from given X.509 certificates. Some of them are:
\begin{itemize}
\item \printfunc{gnutls_x509_extract_certificate_dn}{gnutls\_x509\_extract\_certificate\_dn}
\item \printfunc{gnutls_x509_extract_certificate_serial}{gnutls\_x509\_extract\_certificate\_serial}
\item \printfunc{gnutls_x509_extract_certificate_subject_alt_name}{gnutls\_x509\_extract\_certificate\_subject\_alt\_name}
\end{itemize}

Given the complexity of the X.509 protocols we do not expect these limited 
functions to cover every need. Thus a function which exports X.509 certificates
to an XML form is provided. See 
\printfunc{gnutls_x509_get_certificate_xml}{gnutls\_x509\_get\_certificate\_xml}.

\par
Verifying certificate paths is also important in X.509 authentication.
For this purpose you can use the
\printfunc{gnutls_x509_verify_certificate}{gnutls\_x509\_verify\_certificate}
function. A more generic one is also provided and can be used with all
of the certificate authentication methods, but is limited to a session. See the
\printfunc{gnutls_certificate_verify_peers}{gnutls\_certificate\_verify\_peers}
function.

\par
Note that \gnutls{} is not a generic purpose X.509 toolkit\footnote{Aegypten is such a toolkit. See 
\htmladdnormallink{http://www.gnupg.org/aegypten/}{http://www.gnupg.org/aegypten/}}. 
\gnutls{} only includes the required,
in order to use the TLS ciphersuites which require X.509 certificates.

