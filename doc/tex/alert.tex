\section{The alert protocol}

The Alert protocol is there to allow signals to be sent between peers.
These signals are mostly used to inform the peer about the cause of
a protocol failure. Some of these signals are used internally by the
protocol and the application protocol does not have to cope with them
(see \emph{GNUTLS\_A\_CLOSE\_NOTIFY}), and others refer to the
application protocol solely (see \emph{GNUTLS\_A\_USER\_CANCELLED}).
An alert signal includes a level indication which may be either
fatal or warning. Fatal alerts always terminate the current connection,
and prevent future renegotiations using the current session ID.

\par The alert messages are protected by the record protocol, thus
the information that it's included does not leak. You must take
extreme care for the alert information not to leak, to a possible attacker
(via public logfiles etc).

\par
\begin{itemize}
\item \printfunc{gnutls_alert_send}{gnutls\_alert\_send()}:
to send an alert signal.
\item \printfunc{gnutls_alert_send_appropriate}{gnutls\_alert\_send\_appropriate()}:
to send an alert signal that depends on a given gnutls error number.
\item \printfunc{gnutls_alert_get}{gnutls\_alert\_get()}:
returns the last received alert.
\item \printfunc{gnutls_alert_get_name}{gnutls\_alert\_get\_name()}:
returns the name (in a character array) of the given alert.
\end{itemize}

