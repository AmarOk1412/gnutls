\begin{verbatim}

#include <stdio.h>
#include <stdlib.h>
#include <errno.h>
#include <sys/types.h>
#include <sys/socket.h>
#include <netinet/in.h>
#include <arpa/inet.h>
#include <string.h>
#include <unistd.h>
#include <gnutls.h>

#define KEYFILE "key.pem"
#define CERTFILE "cert.pem"
#define CAFILE "ca.pem"
#define CRLFILE NULL

#define SRP_PASSWD "tpasswd"
#define SRP_PASSWD_CONF "tpasswd.conf"


/* This is a sample TCP echo server.
 */


#define SA struct sockaddr
#define ERR(err,s) if(err==-1) {perror(s);return(1);}
#define MAX_BUF 1024
#define PORT 5556               /* listen to 5556 port */

/* These are global */
SRP_SERVER_CREDENTIALS srp_cred;
X509PKI_SERVER_CREDENTIALS x509_cred;

GNUTLS_STATE initialize_state()
{
   GNUTLS_STATE state;
   int ret;
   const int protocol_priority[] = { GNUTLS_TLS1, GNUTLS_SSL3, 0 };
   const int kx_priority[] = { GNUTLS_KX_X509PKI_RSA, GNUTLS_KX_X509PKI_DHE_RSA, GNUTLS_KX_SRP, 0 };
   const int cipher_priority[] = { GNUTLS_CIPHER_RIJNDAEL_CBC, GNUTLS_CIPHER_3DES_CBC, 0};
   const int comp_priority[] = { GNUTLS_COMP_ZLIB, GNUTLS_COMP_NULL, 0 };
   const int mac_priority[] = { GNUTLS_MAC_SHA, GNUTLS_MAC_MD5, 0 };

   gnutls_init(&state, GNUTLS_SERVER);

   /* in order to support session resuming:
    */
   if ((ret = gnutls_db_set_name(state, "gnutls-rsm.db")) < 0)
      fprintf(stderr, "*** DB error (%d)\n\n", ret);

   gnutls_protocol_set_priority(state, protocol_priority);
   gnutls_cipher_set_priority(state, cipher_priority);
   gnutls_compression_set_priority(state, comp_priority);
   gnutls_kx_set_priority(state, kx_priority);
   gnutls_mac_set_priority(state, mac_priority);

   gnutls_cred_set(state, GNUTLS_SRP, srp_cred);
   gnutls_cred_set(state, GNUTLS_X509PKI, x509_cred);

   /* request client certificate if any.
    */
   gnutls_x509pki_server_set_cert_request( state, GNUTLS_CERT_REQUEST);
   
   return state;
}

void print_info(GNUTLS_STATE state)
{
   const char *tmp;
   unsigned char sesid[32];
   int sesid_size, i;

   /* print session_id specific data */
   gnutls_session_get_id(state, sesid, &sesid_size);
   printf("\n- Session ID: ");
   for (i = 0; i < sesid_size; i++)
      printf("%.2X", sesid[i]);
   printf("\n");

   /* print srp specific data */
   if (gnutls_get_auth_type(state) == GNUTLS_SRP) {
         printf("\n- User '%s' connected\n",
                gnutls_srp_server_get_username( state));
   }

   /* print state information */
   tmp = gnutls_protocol_get_name(gnutls_protocol_get_version(state));
   printf("- Version: %s\n", tmp);

   tmp = gnutls_kx_get_name(gnutls_kx_get_algo(state));
   printf("- Key Exchange: %s\n", tmp);

   tmp =
       gnutls_compression_get_name
       (gnutls_compression_get_algo(state));
   printf("- Compression: %s\n", tmp);

   tmp = gnutls_cipher_get_name(gnutls_cipher_get_algo(state));
   printf("- Cipher: %s\n", tmp);

   tmp = gnutls_mac_get_name(gnutls_mac_get_algo(state));
   printf("- MAC: %s\n", tmp);

}



int main()
{
   int err, listen_sd, i;
   int sd, ret;
   struct sockaddr_in sa_serv;
   struct sockaddr_in sa_cli;
   int client_len;
   char topbuf[512];
   GNUTLS_STATE state;
   char buffer[MAX_BUF + 1];
   int optval = 1;
   int http = 0;
   char name[256];

   strcpy(name, "Echo Server");

   /* this must be called once in the program
    */
   if (gnutls_global_init() < 0) {
      fprintf(stderr, "global state initialization error\n");
      exit(1);
   }
   if (gnutls_x509pki_allocate_server_sc(&x509_cred, 1) < 0) {
      fprintf(stderr, "memory error\n");
      exit(1);
   }
   if (gnutls_x509pki_set_server_trust_file(x509_cred, CAFILE, CRLFILE) < 0) {
      fprintf(stderr, "X509 PARSE ERROR\nDid you have ca.pem?\n");
      exit(1);
   }
   if (gnutls_x509pki_set_server_key_file(x509_cred, CERTFILE, KEYFILE) < 0) {
      fprintf(stderr, "X509 PARSE ERROR\nDid you have key.pem and cert.pem?\n");
      exit(1);
   }
   /* SRP_PASSWD a password file (created with the included crypt utility) 
    * Read README.crypt prior to using SRP.
    */
   gnutls_srp_allocateserver_sc(&srp_cred);
   gnutls_srp_set_server_cred(srp_cred, SRP_PASSWD, SRP_PASSWD_CONF);


   /* Socket operations
    */
   listen_sd = socket(AF_INET, SOCK_STREAM, 0);
   ERR(listen_sd, "socket");

   memset(&sa_serv, '\0', sizeof(sa_serv));
   sa_serv.sin_family = AF_INET;
   sa_serv.sin_addr.s_addr = INADDR_ANY;
   sa_serv.sin_port = htons(PORT);  /* Server Port number */

   setsockopt(listen_sd, SOL_SOCKET, SO_REUSEADDR, &optval, sizeof(int));

   err = bind(listen_sd, (SA *) & sa_serv, sizeof(sa_serv));
   ERR(err, "bind");
   err = listen(listen_sd, 1024);
   ERR(err, "listen");

   printf("%s ready. Listening to port '%d'.\n\n", name, PORT);

   client_len = sizeof(sa_cli);
   for (;;) {
      state = initialize_state();

      sd = accept(listen_sd, (SA *) & sa_cli, &client_len);

      printf("- connection from %s, port %d\n",
             inet_ntop(AF_INET, &sa_cli.sin_addr, topbuf,
                       sizeof(topbuf)), ntohs(sa_cli.sin_port));

      gnutls_transport_set_ptr( state, sd);
      ret = gnutls_handshake( state);
      if (ret < 0) {
         close(sd);
         gnutls_deinit(state);
         fprintf(stderr, "*** Handshake has failed (%s)\n\n",
                 gnutls_strerror(ret));
         continue;
      }
      printf("- Handshake was completed\n");

      print_info(state);

      i = 0;
      for (;;) {
         bzero(buffer, MAX_BUF + 1);
         ret = gnutls_read( state, buffer, MAX_BUF);

         if (gnutls_error_is_fatal(ret) == 1 || ret == 0) {
            if (ret == 0) {
               printf
                   ("\n- Peer has closed the GNUTLS connection\n");
               break;
            } else {
               fprintf(stderr,
                       "\n*** Received corrupted data(%d). Closing the connection.\n\n",
                       ret);
               break;
            }

         }
         if (ret > 0) {
            /* echo data back to the client
             */
            gnutls_write( state, buffer,
                         strlen(buffer));
         }
         if (ret == GNUTLS_E_WARNING_ALERT_RECEIVED || ret == GNUTLS_E_FATAL_ALERT_RECEIVED) {
            ret = gnutls_alert_get_last(state);
            printf("* Received alert '%d'.\n", ret);
         }
      }
      printf("\n");
      gnutls_bye( state, 1); /* do not wait for
                                 * the peer to close the connection.
                                 */

      close(sd);
      gnutls_deinit(state);

   }
   close(listen_sd);

   gnutls_x509pki_free_server_sc(x509_cred);
   gnutls_srp_free_server_sc(srp_cred);

   gnutls_global_deinit();

   return 0;

}

\end{verbatim}
