\chapter{Patents}
\label{ap:patents}
Patents in algorithms is a disputable topic, although some countries have 
decided to approve such patents. In order for \gnutls{} to be free in all
countries, we try not to include patented algorithms, which could turn the 
library being non free.

\section{TLS patent}
\index{Patents!TLS}
A patent which we couln't avoid was a patent by Netscape Communications 
Corporation on the Secure Sockets Layer (\ssl{}) work that \tlsI{} is based on.
Fortunately Netscape has provided a statement that allows royalty free 
adoption and use of the \ssl{} protocol. Below is the quote of Netscape 
Communications' statement in RFC2246\cite{RFC2246}.

\begin{verbatim}
    Intellectual Property Rights

    Secure Sockets Layer

    The United States Patent and Trademark Office ("the PTO")
    recently issued U.S. Patent No. 5,657,390 ("the SSL Patent")  to
    Netscape for inventions described as Secure Sockets Layers
    ("SSL"). The IETF is currently considering adopting SSL as a
    transport protocol with security features.  Netscape encourages
    the royalty-free adoption and use of the SSL protocol upon the
    following terms and conditions:

      * If you already have a valid SSL Ref license today which
        includes source code from Netscape, an additional patent
        license under the SSL patent is not required.

      * If you don't have an SSL Ref license, you may have a royalty
        free license to build implementations covered by the SSL
        Patent Claims or the IETF TLS specification provided that you
        do not to assert any patent rights against Netscape or other
        companies for the implementation of SSL or the IETF TLS
        recommendation.
\end{verbatim}

\section{SRP patent}
\index{Patents!SRP}
A patent application was filed by Stanford University on the SRP algorithm.
The letters\footnote{found in \htmladdnormallink{http://www.ietf.org/ietf/IPR/PHOENIX-SRP-RFC2945.txt}{http://www.ietf.org/ietf/IPR/WU-SRP}} below were sent to IETF.

\begin{verbatim}
Received April 26, 2000
Kirsten Leute <kirsten.leute@stanford.edu>

Stanford University has a U.S. patent pending for the Secure Remote
Password (SRP) authentication and key-exchange system.  To encourage
widespread use of strong cryptographic authentication technologies,
Stanford University is granting royalty-free licenses for SRP when used in
its implicit server authenticating mode, such as implementations based on
RFC 2945.  Details will soon be available at
(http://otl.stanford.edu/industry/resources/rts.html).

Stanford University will also offer non-exclusive licenses in a
nondiscriminatory manner for use of SRP in its bi-directional
authenticating mode (SRP-Z) under reasonable terms and conditions.

Please contact me with any questions regarding the licensing of SRP.

Sincerely,

Kirsten Leute
Associate
(650) 725-9407
Fax: (650) 725-7295
kirsten.leute@stanford.edu

======================================================================================
Received December 22, 2000
From: Thomas Wu <tjw@CS.Stanford.EDU>

The SRP Authentication and Key Exchange System, as specified in
RFC 2945, is available royalty-free worldwide for commercial and
non-commercial use.

Extended variants of SRP, such as those based on SRP-Z, may require
a license, which Stanford will grant on a non-exclusive basis, under
reasonable and non-discriminatory terms.

For questions about SRP, please contact me or visit
http://otl.stanford.edu/

Tom Wu
tjw@CS.Stanford.EDU

\end{verbatim}

\par
\gnutls{} uses the SRP algorithm as described in RFC 2945, which
is available royalty-free by Stanford, so the above patent does
not cause any harm. However 
the US patent 6226383 held by Phoenix, known as the SPEKE patent
may apply to the SRP algorithm according to Phoenix.
See \htmladdnormallink{http://www.ietf.org/ietf/IPR/PHOENIX-SRP-RFC2945.txt}{http://www.ietf.org/ietf/IPR/PHOENIX-SRP-RFC2945.txt}. 
Also the EKE patents (US 5241599 and US 5440635), held by Lucent,
may also apply to the SRP algorithm. See
\htmladdnormallink{http://www.ietf.org/ietf/IPR/LUCENT-SRP}{http://www.ietf.org/ietf/IPR/LUCENT-SRP}.
