\section{Authentication using SRP\index{SRP authentication}}

Authentication using the SRP\footnote{SRP stands for Secure Password Protocol and 
is described in \cite{RFC2945}. The SRP key exchange is not a part of the \tlsI{} protocol}
is actually password authentication, since the two peers are identified by the knowledge of a password. 
This protocol also offers protection against off-line attacks, such as password 
file stealing. 
This is achieved since SRP does not use the plain password to perform authentication, 
but something called a verifier, which is calculated using the user's password.
\par The advantage of SRP authentication, over other proposed secure password 
authentication schemas, is that SRP does not require the server to hold
the user's password. This kind of protection is similar to the one used traditionally
in the \emph{UNIX} ``passwd'' file, where the contents of this file did not cause
harm to the system security if they were revealed.
\par
The implementation in \gnutls{} is based on paper \cite{TLSSRP}.
Available key exchange methods are shown in \hyperref{figure}{figure }{}{fig:srp}.

\begin{figure}[hbtp]
\begin{tabular}{|l|p{9cm}|}

\hline
SRP & Authentication using the SRP protocol. 
\\
\hline
SRP\_DSS & Client authentication using the SRP protocol. Server is 
authenticated using a certificate with DSA parameters.
\\
\hline
SRP\_RSA & Client authentication using the SRP protocol. Server is 
authenticated using a certificate with RSA parameters.
\\
\hline
\end{tabular}

\caption{Supported SRP key exchange algorithms}
\label{fig:srp}

\end{figure}

The default behaviour of \gnutls{} is to read the usernames and
SRP verifiers from password files. These password files are the ones used
by \emph{libsrp} and can be specified using the
\printfunc{gnutls_srp_set_server_credentials_file}{gnutls\_srp\_set\_server\_credentials\_file}.
If a different password file format is to be used, then the 
function \printfunc{gnutls_srp_set_server_credentials_function}{gnutls\_srp\_set\_server\_credentials\_function},
should be called, in order to set an appropriate callback.

Some helper functions such as
\begin{itemize}
\item \printfunc{gnutls_srp_verifier}{gnutls\_srp\_verifier}
\item \printfunc{gnutls_srp_base64_encode}{gnutls\_srp\_base64\_encode}
\item \printfunc{gnutls_srp_base64_decode}{gnutls\_srp\_base64\_decode}
\end{itemize}
are included in \gnutls{}, and may be used to generate, and maintain
SRP verifiers, and password files. A program to manipulate the required parameters 
for SRP authentication is also included. See section \ref{srpcrypt} on 
page \pageref{srpcrypt} for more information.

