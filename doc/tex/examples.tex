\chapter{How to use GNUTLS\index{Example programs} in applications}
\section{Client examples}
This section contains examples of \tls{} and \ssl{} clients, using \gnutls{}. 

\subsection{Simple client example with X.509 certificate support}
Let's assume now that we want to create a client which communicates
with servers using the X.509 authentication schema. The following client
is a very simple \tls{} client, it does not support session resuming nor
any other fancy features.
\begin{verbatim}

#include <stdio.h>
#include <stdlib.h>
#include <sys/types.h>
#include <sys/socket.h>
#include <netinet/in.h>
#include <arpa/inet.h>
#include <gnutls.h>

#define MAX_BUF 1024
#define CRLFILE "crl.pem"
#define CAFILE "ca.pem"
#define SA struct sockaddr
#define MSG "GET / HTTP/1.0\r\n\r\n"

int main()
{
   const char *PORT = "443";
   const char *SERVER = "127.0.0.1";
   int err, ret;
   int sd, ii;
   struct sockaddr_in sa;
   GNUTLS_STATE state;
   char buffer[MAX_BUF + 1];
   X509PKI_CLIENT_CREDENTIALS xcred;
   const int protocol_priority[] = { GNUTLS_TLS1, GNUTLS_SSL3, 0 };
   const int kx_priority[] = { GNUTLS_KX_X509PKI_RSA, 0 };
   const int cipher_priority[] = { GNUTLS_CIPHER_3DES_CBC, GNUTLS_CIPHER_ARCFOUR, 0};
   const int comp_priority[] = { GNUTLS_COMP_ZLIB, GNUTLS_COMP_NULL, 0 };
   const int mac_priority[] = { GNUTLS_MAC_SHA, GNUTLS_MAC_MD5, 0 };


   if (gnutls_global_init() < 0) {
      fprintf(stderr, "global state initialization error\n");
      exit(1);
   }
   /* X509 stuff */
   if (gnutls_x509pki_allocate_client_sc(&xcred, 0) < 0) {  /* no client private key */
      fprintf(stderr, "memory error\n");
      exit(1);
   }
   /* set's the trusted cas file
    */
   gnutls_x509pki_set_client_trust_file(xcred, CAFILE, CRLFILE);

   /* connects to server 
    */
   sd = socket(AF_INET, SOCK_STREAM, 0);

   memset(&sa, '\0', sizeof(sa));
   sa.sin_family = AF_INET;
   sa.sin_port = htons(atoi(PORT));
   inet_pton(AF_INET, SERVER, &sa.sin_addr);

   err = connect(sd, (SA *) & sa, sizeof(sa));
   if (err < 0) {
      fprintf(stderr, "Connect error\n");
      exit(1);
   }
   /* Initialize TLS state 
    */
   gnutls_init(&state, GNUTLS_CLIENT);

   /* allow both SSL3 and TLS1
    */
   gnutls_protocol_set_priority(state, protocol_priority);

   /* allow only ARCFOUR and 3DES ciphers
    * (3DES has the highest priority)
    */
   gnutls_cipher_set_priority(state, cipher_priority);

   /* only allow null compression
    */
   gnutls_compression_set_priority(state, comp_priority);

   /* use GNUTLS_KX_X509PKI_RSA
    */
   gnutls_kx_set_priority(state, kx_priority);

   /* allow the usage of both SHA and MD5
    */
   gnutls_mac_set_priority(state, mac_priority);


   /* put the x509 credentials to the current state
    */
   gnutls_set_cred(state, GNUTLS_X509PKI, xcred);


   gnutls_transport_set_ptr( state, sd);
   /* Perform the TLS handshake
    */
   ret = gnutls_handshake( state);

   if (ret < 0) {
      fprintf(stderr, "*** Handshake failed\n");
      gnutls_perror(ret);
      goto end;
   } else {
      printf("- Handshake was completed\n");
   }

   gnutls_write( state, MSG, strlen(MSG));

   ret = gnutls_read( state, buffer, MAX_BUF);
   if (gnutls_is_fatal_error(ret) == 1 || ret == 0) {
      if (ret == 0) {
         printf("- Peer has closed the GNUTLS connection\n");
         goto end;
      } else {
         fprintf(stderr, "*** Received corrupted data(%d) - server has terminated the connection abnormally\n",
                 ret);
         goto end;
      }
   } else {
      if (ret == GNUTLS_E_WARNING_ALERT_RECEIVED || ret == GNUTLS_E_FATAL_ALERT_RECEIVED)
         printf("* Received alert [%d]\n", gnutls_get_last_alert(state));
      if (ret == GNUTLS_E_REHANDSHAKE)
         printf("* Received HelloRequest message (server asked to rehandshake)\n");
         gnutls_send_appropriate_alert( state, ret); /* we don't want rehandshake */
   }

   if (ret > 0) {
      printf("- Received %d bytes: ", ret);
      for (ii = 0; ii < ret; ii++) {
         fputc(buffer[ii], stdout);
      }
      fputs("\n", stdout);
   }
   gnutls_bye( state, GNUTLS_SHUT_RDWR);

 end:

   shutdown(sd, SHUT_RDWR);     /* no more receptions */
   close(sd);

   gnutls_deinit(state);

   gnutls_x509pki_free_client_sc(xcred);

   gnutls_global_deinit();

   return 0;
}

\end{verbatim}


\subsection{Getting peer's information}
\par The above example was the simplest form of a client, it didn't even check
the result of the peer's certificate verification function. The lack of
this check may result to an unauthenticated connection.
The following function does check the peer's
X.509 certificate, and prints some information about the current state.
\par
This function should be called after a successful
\printfunc{gnutls_handshake}{gnutls\_handshake}

\begin{verbatim}

#define PRINTX(x,y) if (y[0]!=0) printf(" -   %s %s\n", x, y)
#define PRINT_DN(X) PRINTX( "CN:", X.common_name); \
        PRINTX( "OU:", X.organizational_unit_name); \
        PRINTX( "O:", X.organization); \
        PRINTX( "L:", X.locality_name); \
        PRINTX( "S:", X.state_or_province_name); \
        PRINTX( "C:", X.country); \
        PRINTX( "E:", X.email)

/* This function will print some details of the
 * given state.
 */
int print_info(GNUTLS_STATE state)
{
   const char *tmp;
   GNUTLS_CredType cred;
   gnutls_x509_dn dn;
   const gnutls_datum *cert_list;
   int status;
   int cert_list_size = 0;
   GNUTLS_KXAlgorithm kx;
   time_t expiret = gnutls_certificate_expiration_time_peers(state);
   time_t activet = gnutls_certificate_activation_time_peers(state);

   /* print the key exchange's algorithm name
    */
   kx = gnutls_kx_get(state);
   tmp = gnutls_kx_get_name(kx);
   printf("- Key Exchange: %s\n", tmp);

   cred = gnutls_auth_get_type(state);
   switch (cred) {
   case GNUTLS_CRD_ANON:
      printf("- Anonymous DH using prime of %d bits\n",
             gnutls_dh_get_bits(state));
      break;
   case GNUTLS_CRD_CERTIFICATE:
      /* in case of certificate authentication
       */
      cert_list = gnutls_certificate_get_peers(state, &cert_list_size);
      status = gnutls_certificate_verify_peers(state);
      
      if ( status < 0) {
         if ( status == GNUTLS_E_NO_CERTIFICATE_FOUND)
            printf("- Peer did not send any X509 Certificate.\n");
         else
            printf("- Could not verify certificate\n");
      } else {

         if ( status & GNUTLS_CERT_INVALID)
            printf("- Peer's certificate is invalid\n");
         if ( status & GNUTLS_CERT_CORRUPTED)
            printf("- Peer's certificate is corrupted.\n");
         if ( status & GNUTLS_CERT_REVOKED)
            printf("- Peer's certificate is revoked\n");

         if ( status & GNUTLS_CERT_NOT_TRUSTED)
            printf("- Peer's certificate is not trusted\n");
         else
            printf("- Peer's certificate is trusted\n");
      }

      /* Check if we have been using ephemeral Diffie Hellman.
       */
      if (kx == GNUTLS_KX_DHE_RSA || kx == GNUTLS_KX_DHE_DSS) {
         printf("\n- Ephemeral DH using prime of %d bits\n",
                gnutls_dh_get_bits(state));
      }

      /* if the certificate list is available, then
       * print some information about it.
       */
      if (cert_list_size > 0 && gnutls_cert_type_get(state) == GNUTLS_CRT_X509) {
         char digest[20];
         char serial[40];
         int digest_size = sizeof(digest), i;
         int serial_size = sizeof(serial);
         char printable[120];
         char *print;
         int algo, bits;

         printf(" - Certificate info:\n");

         printf(" - Certificate is valid since: %s", ctime( &activet));
         printf(" - Certificate expires: %s", ctime( &expiret));

         /* Print the fingerprint of the certificate
          */
         if (gnutls_x509_fingerprint(GNUTLS_DIG_MD5, &cert_list[0], digest, &digest_size) >= 0) {
            print = printable;
            for (i = 0; i < digest_size; i++) {
               sprintf(print, "%.2x ", (unsigned char) digest[i]);
               print += 3;
            }
            printf(" - Certificate fingerprint: %s\n", printable);
         }

         /* Print the serial number of the certificate.
          */
         if (gnutls_x509_extract_certificate_serial(&cert_list[0], serial, &serial_size) >= 0) {
            print = printable;
            for (i = 0; i < serial_size; i++) {
               sprintf(print, "%.2x ", (unsigned char) serial[i]);
               print += 3;
            }
            printf(" - Certificate serial number: %s\n", printable);
         }

         /* Extract some of the public key algorithm's parameters
          */
         algo = gnutls_x509_extract_certificate_pk_algorithm( &cert_list[0], &bits);
         printf("Certificate public key: ");

         if (algo==GNUTLS_PK_RSA) {
            printf("RSA\n");
            printf(" Modulus: %d bits\n", bits);
         } else if (algo==GNUTLS_PK_DSA) {
            printf("DSA\n");
            printf(" Exponent: %d bits\n", bits);
         } else {
            printf("UNKNOWN\n");
         }

         /* Print the version of the X.509 
          * certificate.
          */
         printf(" - Certificate version: #%d\n", gnutls_x509_extract_certificate_version(&cert_list[0]));

         gnutls_x509_extract_certificate_dn(&cert_list[0], &dn);
         PRINT_DN(dn);

         gnutls_x509_extract_certificate_issuer_dn(&cert_list[0], &dn);
         printf(" - Certificate Issuer's info:\n");
         PRINT_DN(dn);

      }
   }

   tmp = gnutls_protocol_get_name(gnutls_protocol_get_version(state));
   printf("- Protocol: %s\n", tmp);

   tmp = gnutls_cert_type_get_name( gnutls_cert_type_get(state));
   printf("- Certificate Type: %s\n", tmp);

   tmp = gnutls_compression_get_name(gnutls_compression_get(state));
   printf("- Compression: %s\n", tmp);

   tmp = gnutls_cipher_get_name(gnutls_cipher_get(state));
   printf("- Cipher: %s\n", tmp);

   tmp = gnutls_mac_get_name(gnutls_mac_get(state));
   printf("- MAC: %s\n", tmp);

   return 0;
}

\end{verbatim}


\subsection{Client with Resume capability example}
\label{resume-example}
This is the same client as above, but here we add support for session
resumption.
\begin{verbatim}

#include <stdio.h>
#include <stdlib.h>
#include <sys/types.h>
#include <sys/socket.h>
#include <netinet/in.h>
#include <arpa/inet.h>
#include <gnutls.h>

#define MAX_BUF 1024
#define CRLFILE "crl.pem"
#define CAFILE "ca.pem"
#define SA struct sockaddr
#define MSG "GET / HTTP/1.0\r\n\r\n"

const int protocol_priority[] = { GNUTLS_TLS1, GNUTLS_SSL3, 0 };
const int kx_priority[] = { GNUTLS_KX_RSA, GNUTLS_KX_DHE_RSA, 0 };
const int cipher_priority[] = { GNUTLS_CIPHER_3DES_CBC, GNUTLS_CIPHER_ARCFOUR, 0};
const int comp_priority[] = { GNUTLS_COMP_ZLIB, GNUTLS_COMP_NULL, 0 };
const int mac_priority[] = { GNUTLS_MAC_SHA, GNUTLS_MAC_MD5, 0 };

int main()
{
   const char *PORT = "443";
   const char *SERVER = "127.0.0.1";
   int err, ret;
   int sd, ii, alert;
   struct sockaddr_in sa;
   GNUTLS_STATE state;
   char buffer[MAX_BUF + 1];
   GNUTLS_CERTIFICATE_CLIENT_CREDENTIALS xcred;
   /* variables used in session resuming */
   int t;
   char *session;
   int session_size;

   if (gnutls_global_init() < 0) {
      fprintf(stderr, "global state initialization error\n");
      exit(1);
   }
   /* X509 stuff */
   if (gnutls_certificate_allocate_client_sc(&xcred) < 0) {
      fprintf(stderr, "memory error\n");
      exit(1);
   }
   gnutls_certificate_set_x509_trust_file(xcred, CAFILE, GNUTLS_X509_FMT_PEM);

   for (t = 0; t < 2; t++) {    /* connect 2 times to the server */

      sd = socket(AF_INET, SOCK_STREAM, 0);
      memset(&sa, '\0', sizeof(sa));
      sa.sin_family = AF_INET;
      sa.sin_port = htons(atoi(PORT));
      inet_pton(AF_INET, SERVER, &sa.sin_addr);

      err = connect(sd, (SA *) & sa, sizeof(sa));
      if (err < 0) {
         fprintf(stderr, "Connect error");
         exit(1);
      }
      gnutls_init(&state, GNUTLS_CLIENT);

      gnutls_protocol_set_priority(state, protocol_priority);
      gnutls_cipher_set_priority(state, cipher_priority);
      gnutls_compression_set_priority(state, comp_priority);
      gnutls_kx_set_priority(state, kx_priority);
      gnutls_mac_set_priority(state, mac_priority);

      gnutls_cred_set(state, GNUTLS_CRD_CERTIFICATE, xcred);

      if (t > 0) { /* if this is not the first time we connect */
         gnutls_session_set_data(state, session, session_size);
         free(session);
      }
      
      gnutls_transport_set_ptr( state, sd);

      /* Perform the TLS handshake
       */
      ret = gnutls_handshake( state);

      if (ret < 0) {
         fprintf(stderr, "*** Handshake failed\n");
         gnutls_perror(ret);
         goto end;
      } else {
         printf("- Handshake was completed\n");
      }

      if (t == 0) { /* the first time we connect */
         /* get the session data size */
         gnutls_session_get_data(state, NULL, &session_size);
         session = malloc(session_size);

         /* put session data to the session variable */
         gnutls_session_get_data(state, session, &session_size);

      } else { /* the second time we connect */

         /* check if we actually resumed the previous session */
         if (gnutls_session_resumed( state) == 0) {
            printf("- Previous session was resumed\n");
         } else {
            fprintf(stderr, "*** Previous session was NOT resumed\n");
         }
      }

      /* This function was defined in a previous example
       */
      print_info(state);

      gnutls_record_send( state, MSG, strlen(MSG));

      ret = gnutls_record_recv( state, buffer, MAX_BUF);
      if (gnutls_error_is_fatal(ret) == 1 || ret == 0) {
         if (ret == 0) {
            printf("- Peer has closed the GNUTLS connection\n");
            goto end;
         } else {
            fprintf(stderr, "*** Received corrupted data(%d) - server has terminated the connection abnormally\n",
                    ret);
            goto end;
         }
      } else {
         if (ret == GNUTLS_E_WARNING_ALERT_RECEIVED || ret == GNUTLS_E_FATAL_ALERT_RECEIVED)
            alert = gnutls_alert_get(state);
            printf("* Received alert [%d]: %s\n", alert, gnutls_alert_get_name(alert));
         if (ret == GNUTLS_E_REHANDSHAKE) {
            printf("* Received HelloRequest message (server asked to rehandshake)\n");
            gnutls_alert_send_appropriate( state, ret); /* we don't want rehandshake */
         }
      }

      if (ret > 0) {
         printf("- Received %d bytes: ", ret);
         for (ii = 0; ii < ret; ii++) {
            fputc(buffer[ii], stdout);
         }
         fputs("\n", stdout);
      }
      gnutls_bye( state, GNUTLS_SHUT_RDWR);

    end:

      shutdown(sd, SHUT_RDWR);  /* no more receptions */
      close(sd);

      gnutls_deinit(state);

   }  /* for() */

   gnutls_certificate_free_client_sc(xcred);

   gnutls_global_deinit();

   return 0;
}

\end{verbatim}


\subsection{Client with Resume capability example II}
\label{resume-example2}
This is also a client with resume capability, but also demonstrates
the use of session IDs.
\begin{verbatim}

#include <stdio.h>
#include <stdlib.h>
#include <sys/types.h>
#include <sys/socket.h>
#include <netinet/in.h>
#include <arpa/inet.h>
#include <gnutls.h>

#define MAX_BUF 1024
#define CRLFILE "crl.pem"
#define CAFILE "ca.pem"
#define SA struct sockaddr
#define MSG "GET / HTTP/1.0\r\n\r\n"

int main()
{
   const char *PORT = "443";
   const char *SERVER = "127.0.0.1";
   int err, ret;
   int sd, ii;
   struct sockaddr_in sa;
   GNUTLS_STATE state;
   char buffer[MAX_BUF + 1];
   X509PKI_CLIENT_CREDENTIALS xcred;
   /* variables used in session resuming */
   int t;
   char *session;
   char *session_id;
   int session_size;
   int session_id_size;
   char *tmp_session_id;
   int tmp_session_id_size;

   if (gnutls_global_init() < 0) {
      fprintf(stderr, "global state initialization error\n");
      exit(1);
   }
   /* X509 stuff */
   if (gnutls_allocate_x509_client_sc(&xcred, 0) < 0) {  /* no client private key */
      fprintf(stderr, "memory error\n");
      exit(1);
   }
   gnutls_set_x509_client_trust(xcred, CAFILE, CRLFILE);

   for (t = 0; t < 2; t++) {    /* connect 2 times to the server */

      sd = socket(AF_INET, SOCK_STREAM, 0);
      memset(&sa, '\0', sizeof(sa));
      sa.sin_family = AF_INET;
      sa.sin_port = htons(atoi(PORT));
      inet_pton(AF_INET, SERVER, &sa.sin_addr);

      err = connect(sd, (SA *) & sa, sizeof(sa));
      if (err < 0) {
         fprintf(stderr, "Connect error");
         exit(1);
      }
      gnutls_init(&state, GNUTLS_CLIENT);
      gnutls_set_protocol_priority(state, GNUTLS_TLS1, GNUTLS_SSL3, 0);
      gnutls_set_cipher_priority(state, GNUTLS_3DES_CBC, GNUTLS_ARCFOUR, 0);
      gnutls_set_compression_priority(state, GNUTLS_NULL_COMPRESSION, 0);
      gnutls_set_kx_priority(state, GNUTLS_KX_RSA, 0);
      gnutls_set_mac_priority(state, GNUTLS_MAC_SHA, GNUTLS_MAC_MD5, 0);


      gnutls_set_cred(state, GNUTLS_X509PKI, xcred);

      if (t > 0) { /* if this is not the first time we connect */
         gnutls_set_current_session(state, session, session_size);
         free(session);
      }
      /* Perform the TLS handshake
       */
      ret = gnutls_handshake(sd, state);

      if (ret < 0) {
         fprintf(stderr, "*** Handshake failed\n");
         gnutls_perror(ret);
         goto end;
      } else {
         printf("- Handshake was completed\n");
      }

      if (t == 0) { /* the first time we connect */
         /* get the session data size */
         gnutls_get_current_session(state, NULL, &session_size);
         session = malloc(session_size);

         /* put session data to the session variable */
         gnutls_get_current_session(state, session, &session_size);

         /* keep the current session ID. This is only needed
          * in order to check if the server actually resumed this
          * connection.
          */
         gnutls_get_current_session_id(state, NULL, &session_id_size);
         session_id = malloc(session_id_size);
         gnutls_get_current_session_id(state, session_id, &session_id_size);

      } else { /* the second time we connect */

         /* check if we actually resumed the previous session */
         gnutls_get_current_session_id(state, NULL, &tmp_session_id_size);
         tmp_session_id = malloc(tmp_session_id_size);
         gnutls_get_current_session_id(state, tmp_session_id, &tmp_session_id_size);

         if (memcmp(tmp_session_id, session_id, session_id_size) == 0) {
            printf("- Previous session was resumed\n");
         } else {
            fprintf(stderr, "*** Previous session was NOT resumed\n");
         }
         free(tmp_session_id);
         free(session_id);
      }

      /* This function was defined in a previous example
       */
      print_info(state);

      gnutls_write(sd, state, MSG, strlen(MSG));

      ret = gnutls_read(sd, state, buffer, MAX_BUF);
      if (gnutls_is_fatal_error(ret) == 1 || ret == 0) {
         if (ret == 0) {
            printf("- Peer has closed the GNUTLS connection\n");
            goto end;
         } else {
            fprintf(stderr, "*** Received corrupted data(%d) - server has terminated the connection abnormally\n",
                    ret);
            goto end;
         }
      } else {
         if (ret == GNUTLS_E_WARNING_ALERT_RECEIVED || ret == GNUTLS_E_FATAL_ALERT_RECEIVED)
            printf("* Received alert [%d]\n", gnutls_get_last_alert(state));
         if (ret == GNUTLS_E_GOT_HELLO_REQUEST)
            printf("* Received HelloRequest message (server asked to rehandshake)\n");
      }

      if (ret > 0) {
         printf("- Received %d bytes: ", ret);
         for (ii = 0; ii < ret; ii++) {
            fputc(buffer[ii], stdout);
         }
         fputs("\n", stdout);
      }
      gnutls_bye(sd, state, 0);

    end:

      shutdown(sd, SHUT_RDWR);  /* no more receptions */
      close(sd);

      gnutls_deinit(state);

   }  /* for() */

   gnutls_free_x509_client_sc(xcred);

   gnutls_global_deinit();

   return 0;
}

\end{verbatim}


\subsection{Simple client example with SRP authentication}
Although {\bf SRP} is not part of the \tls{} standard, \gnutls{} implements
{\it David Taylor's} proposal\footnote{This is work in progress.}  for using the SRP algorithm
within the \tls{} handshake protocol. The following client
is a very simple SRP-TLS client which connects to a server 
and authenticates using {\it username} and {\it password}.

\begin{verbatim}

#include <stdio.h>
#include <stdlib.h>
#include <sys/types.h>
#include <sys/socket.h>
#include <netinet/in.h>
#include <arpa/inet.h>
#include <gnutls/gnutls.h>
#include <gnutls/extra.h>

#define MAX_BUF 1024
#define USERNAME "user"
#define PASSWORD "pass"
#define SA struct sockaddr
#define MSG "GET / HTTP/1.0\r\n\r\n"

const int protocol_priority[] = { GNUTLS_TLS1, GNUTLS_SSL3, 0 };
const int kx_priority[] = { GNUTLS_KX_SRP, 0 };
const int cipher_priority[] = { GNUTLS_CIPHER_3DES_CBC, GNUTLS_CIPHER_ARCFOUR\_128, 0};
const int comp_priority[] = { GNUTLS_COMP_NULL, 0 };
const int mac_priority[] = { GNUTLS_MAC_SHA, GNUTLS_MAC_MD5, 0 };

int main()
{
   const char *PORT = "443";
   const char *SERVER = "127.0.0.1";
   int err, ret;
   int sd, ii;
   struct sockaddr_in sa;
   gnutls_session state;
   char buffer[MAX_BUF + 1];
   gnutls_srp_client_credentials xcred;

   if (gnutls_global_init() < 0) {
      fprintf(stderr, "global state initialization error\n");
      exit(1);
   }

   /* now enable the gnutls-extra library which contains the
    * SRP stuff. */
   if (gnutls_global_init_extra() < 0) {
      fprintf(stderr, "global state initialization error\n");
      exit(1);
   }

   if (gnutls_srp_allocate_client_cred(&xcred) < 0) {
      fprintf(stderr, "memory error\n");
      exit(1);
   }
   gnutls_srp_set_client_cred(xcred, USERNAME, PASSWORD);

   /* connects to server 
    */
   sd = socket(AF_INET, SOCK_STREAM, 0);

   memset(&sa, '\0', sizeof(sa));
   sa.sin_family = AF_INET;
   sa.sin_port = htons(atoi(PORT));
   inet_pton(AF_INET, SERVER, &sa.sin_addr);

   err = connect(sd, (SA *) & sa, sizeof(sa));
   if (err < 0) {
      fprintf(stderr, "Connect error\n");
      exit(1);
   }
   /* Initialize TLS state 
    */
   gnutls_session_init(&state, GNUTLS_CLIENT);

   /* allow both SSL3 and TLS1
    */
   gnutls_protocol_set_priority(state, protocol_priority);
 
   /* allow only ARCFOUR and 3DES ciphers
    * (3DES has the highest priority)
    */
    gnutls_cipher_set_priority(state, cipher_priority);

   /* only allow null compression
    */
   gnutls_compression_set_priority(state, comp_priority);
 
   /* use GNUTLS_KX_SRP
    */
   gnutls_kx_set_priority(state, kx_priority);
 
   /* allow the usage of both SHA and MD5
    */
   gnutls_mac_set_priority(state, mac_priority);


   /* put the SRP credentials to the current state
    */
   gnutls_cred_set(state, GNUTLS_CRD_SRP, xcred);

   gnutls_transport_set_ptr( state, sd);

   /* Perform the TLS handshake
    */
   ret = gnutls_handshake( state);

   if (ret < 0) {
      fprintf(stderr, "*** Handshake failed\n");
      gnutls_perror(ret);
      goto end;
   } else {
      printf("- Handshake was completed\n");
   }

   gnutls_record_send( state, MSG, strlen(MSG));

   ret = gnutls_record_recv( state, buffer, MAX_BUF);
   if (gnutls_error_is_fatal(ret) == 1 || ret == 0) {
      if (ret == 0) {
         printf("- Peer has closed the GNUTLS connection\n");
         goto end;
      } else {
         fprintf(stderr, "*** Received corrupted data(%d) - server has terminated the connection abnormally\n",
                 ret);
         goto end;
      }
   } else {
      if (ret == GNUTLS_E_WARNING_ALERT_RECEIVED || ret == GNUTLS_E_FATAL_ALERT_RECEIVED)
         printf("* Received alert [%d]\n", gnutls_alert_get(state));
      if (ret == GNUTLS_E_REHANDSHAKE)
         printf("* Received HelloRequest message (server asked to rehandshake)\n");
   }

   if (ret > 0) {
      printf("- Received %d bytes: ", ret);
      for (ii = 0; ii < ret; ii++) {
         fputc(buffer[ii], stdout);
      }
      fputs("\n", stdout);
   }
   gnutls_bye( state, 0);

 end:

   shutdown(sd, SHUT_RDWR);     /* no more receptions */
   close(sd);

   gnutls_session_deinit(state);

   gnutls_srp_free_client_cred(xcred);

   gnutls_global_deinit();

   return 0;
}

\end{verbatim}


\section{Server examples}
This section contains examples of \tls{} and \ssl{} servers, using \gnutls{}.

\subsection{Echo Server with X.509 and SRP authentication}
The following example is a server which supports both {\bf SRP} and {\bf X.509} authentication.
This server also supports {\it session resuming}.
\begin{verbatim}

#include <stdio.h>
#include <stdlib.h>
#include <errno.h>
#include <sys/types.h>
#include <sys/socket.h>
#include <netinet/in.h>
#include <arpa/inet.h>
#include <string.h>
#include <unistd.h>
#include <gnutls.h>

#define KEYFILE "key.pem"
#define CERTFILE "cert.pem"
#define CAFILE "ca.pem"
#define CRLFILE NULL

#define SRP_PASSWD "tpasswd"
#define SRP_PASSWD_CONF "tpasswd.conf"


/* This is a sample TCP echo server.
 */


#define SA struct sockaddr
#define ERR(err,s) if(err==-1) {perror(s);return(1);}
#define MAX_BUF 1024
#define PORT 5556               /* listen to 5556 port */

/* These are global */
GNUTLS_SRP_SERVER_CREDENTIALS srp_cred;
GNUTLS_CERTIFICATE_SERVER_CREDENTIALS x509_cred;

GNUTLS_STATE initialize_state()
{
   GNUTLS_STATE state;
   int ret;
   const int protocol_priority[] = { GNUTLS_TLS1, GNUTLS_SSL3, 0 };
   const int kx_priority[] = { GNUTLS_KX_RSA, GNUTLS_KX_DHE_RSA, GNUTLS_KX_SRP, 0 };
   const int cipher_priority[] = { GNUTLS_CIPHER_RIJNDAEL_CBC, GNUTLS_CIPHER_3DES_CBC, 0};
   const int comp_priority[] = { GNUTLS_COMP_ZLIB, GNUTLS_COMP_NULL, 0 };
   const int mac_priority[] = { GNUTLS_MAC_SHA, GNUTLS_MAC_MD5, 0 };

   gnutls_init(&state, GNUTLS_SERVER);

   /* in order to support session resuming:
    */
   if ((ret = gnutls_db_set_name(state, "gnutls-rsm.db")) < 0)
      fprintf(stderr, "*** DB error (%d)\n\n", ret);

   gnutls_protocol_set_priority(state, protocol_priority);
   gnutls_cipher_set_priority(state, cipher_priority);
   gnutls_compression_set_priority(state, comp_priority);
   gnutls_kx_set_priority(state, kx_priority);
   gnutls_mac_set_priority(state, mac_priority);

   gnutls_cred_set(state, GNUTLS_CRD_SRP, srp_cred);
   gnutls_cred_set(state, GNUTLS_CRD_CERTIFICATE, x509_cred);

   /* request client certificate if any.
    */
   gnutls_certificate_server_set_request( state, GNUTLS_CERT_REQUEST);
   
   return state;
}

void print_info(GNUTLS_STATE state)
{
   const char *tmp;
   unsigned char sesid[32];
   int sesid_size, i;

   /* print session_id specific data */
   gnutls_session_get_id(state, sesid, &sesid_size);
   printf("\n- Session ID: ");
   for (i = 0; i < sesid_size; i++)
      printf("%.2X", sesid[i]);
   printf("\n");

   /* print srp specific data */
   if (gnutls_auth_get_type(state) == GNUTLS_CRD_SRP) {
         printf("\n- User '%s' connected\n",
                gnutls_srp_server_get_username( state));
   }

   /* print state information */
   tmp = gnutls_protocol_get_name(gnutls_protocol_get_version(state));
   printf("- Version: %s\n", tmp);

   tmp = gnutls_kx_get_name(gnutls_kx_get(state));
   printf("- Key Exchange: %s\n", tmp);

   tmp =
       gnutls_compression_get_name
       (gnutls_compression_get(state));
   printf("- Compression: %s\n", tmp);

   tmp = gnutls_cipher_get_name(gnutls_cipher_get(state));
   printf("- Cipher: %s\n", tmp);

   tmp = gnutls_mac_get_name(gnutls_mac_get(state));
   printf("- MAC: %s\n", tmp);

}



int main()
{
   int err, listen_sd, i;
   int sd, ret;
   struct sockaddr_in sa_serv;
   struct sockaddr_in sa_cli;
   int client_len;
   char topbuf[512];
   GNUTLS_STATE state;
   char buffer[MAX_BUF + 1];
   int optval = 1;
   int http = 0;
   char name[256];

   strcpy(name, "Echo Server");

   /* this must be called once in the program
    */
   if (gnutls_global_init() < 0) {
      fprintf(stderr, "global state initialization error\n");
      exit(1);
   }
   if (gnutls_certificate_allocate_server_sc(&x509_cred) < 0) {
      fprintf(stderr, "memory error\n");
      exit(1);
   }
   if (gnutls_certificate_set_x509_trust_file(x509_cred, CAFILE, CRLFILE) < 0) {
      fprintf(stderr, "X509 PARSE ERROR\nDid you have ca.pem?\n");
      exit(1);
   }
   if (gnutls_certificate_set_x509_key_file(x509_cred, CERTFILE, KEYFILE) < 0) {
      fprintf(stderr, "X509 PARSE ERROR\nDid you have key.pem and cert.pem?\n");
      exit(1);
   }
   /* SRP_PASSWD a password file (created with the included crypt utility) 
    * Read README.crypt prior to using SRP.
    */
   gnutls_srp_allocate_server_sc(&srp_cred);
   gnutls_srp_set_server_cred_file(srp_cred, SRP_PASSWD, SRP_PASSWD_CONF);


   /* Socket operations
    */
   listen_sd = socket(AF_INET, SOCK_STREAM, 0);
   ERR(listen_sd, "socket");

   memset(&sa_serv, '\0', sizeof(sa_serv));
   sa_serv.sin_family = AF_INET;
   sa_serv.sin_addr.s_addr = INADDR_ANY;
   sa_serv.sin_port = htons(PORT);  /* Server Port number */

   setsockopt(listen_sd, SOL_SOCKET, SO_REUSEADDR, &optval, sizeof(int));

   err = bind(listen_sd, (SA *) & sa_serv, sizeof(sa_serv));
   ERR(err, "bind");
   err = listen(listen_sd, 1024);
   ERR(err, "listen");

   printf("%s ready. Listening to port '%d'.\n\n", name, PORT);

   client_len = sizeof(sa_cli);
   for (;;) {
      state = initialize_state();

      sd = accept(listen_sd, (SA *) & sa_cli, &client_len);

      printf("- connection from %s, port %d\n",
             inet_ntop(AF_INET, &sa_cli.sin_addr, topbuf,
                       sizeof(topbuf)), ntohs(sa_cli.sin_port));

      gnutls_transport_set_ptr( state, sd);
      ret = gnutls_handshake( state);
      if (ret < 0) {
         close(sd);
         gnutls_deinit(state);
         fprintf(stderr, "*** Handshake has failed (%s)\n\n",
                 gnutls_strerror(ret));
         continue;
      }
      printf("- Handshake was completed\n");

      print_info(state);

      i = 0;
      for (;;) {
         bzero(buffer, MAX_BUF + 1);
         ret = gnutls_read( state, buffer, MAX_BUF);

         if (gnutls_error_is_fatal(ret) == 1 || ret == 0) {
            if (ret == 0) {
               printf
                   ("\n- Peer has closed the GNUTLS connection\n");
               break;
            } else {
               fprintf(stderr,
                       "\n*** Received corrupted data(%d). Closing the connection.\n\n",
                       ret);
               break;
            }

         }
         if (ret > 0) {
            /* echo data back to the client
             */
            gnutls_write( state, buffer,
                         strlen(buffer));
         }
         if (ret == GNUTLS_E_WARNING_ALERT_RECEIVED || ret == GNUTLS_E_FATAL_ALERT_RECEIVED) {
            ret = gnutls_alert_get_last(state);
            printf("* Received alert '%d'.\n", ret);
         }
      }
      printf("\n");
      gnutls_bye( state, 1); /* do not wait for
                                 * the peer to close the connection.
                                 */

      close(sd);
      gnutls_deinit(state);

   }
   close(listen_sd);

   gnutls_certificate_free_server_sc(x509_cred);
   gnutls_srp_free_server_sc(srp_cred);

   gnutls_global_deinit();

   return 0;

}

\end{verbatim}


\section{Compatibility with the OpenSSL\index{OpenSSL} library}

To ease \gnutls{}' integration with existing applications, a compatibility 
layer with the widely used OpenSSL library is included in the \emph{gnutls-openssl}
library. This compatibility layer is not complete and it is not 
intended to completely reimplement the OpenSSL API with \gnutls{}.
It only provides source-level compatibility. There is currently no
attempt to make it binary-compatible with OpenSSL.

Prototypes for the compatibility functions are found in the 
``gnutls/openssl.h'' header file.

Current limitations imposed by the compatibility layer include:

\begin{itemize}

\item Error handling is not thread safe.

\end{itemize}


