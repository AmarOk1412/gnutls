\chapter{How to use \gnutls{}\index{Example programs} in applications}

%\section{Preparation\footnote{This section is heavily based on the `libksba' documentation}}
\section{Preparation}

To use \gnutls{}, you have to perform some changes to your sources and
your build system. The necessary changes are explained in the following
subsections.

\subsection*{Headers}

All the data types and functions of the \gnutls{} library are defined in
the header file `gnutls/gnutls.h'. This must be included in all programs that
make use of the \gnutls{} library.
\par
The extra functionality of the \gnutlse{} library is available by
including the header file `gnutls/extra.h' in your programs.

\subsection*{Version check}
It is often desirable to check that the version of `gnutls' used is indeed
one which fits all requirements.  Even with binary compatibility new
features may have been introduced but due to problem with the dynamic
linker an old version is actually used.  So you may want to check that
the version is okay right after program startup.
See the function \printfunc{gnutls_check_version}{gnutls\_check\_version}


\subsection*{Building the source}

If you want to compile a source file including the `gnutls/gnutls.h' header
file, you must make sure that the compiler can find it in the
directory hierarchy.  This is accomplished by adding the path to the
directory in which the header file is located to the compilers include
file search path (via the -I option).

However, the path to the include file is determined at the time the
source is configured.  To solve this problem, \gnutls{} ships with two small
helper programs \command{libgnutls-config} and \command{libgnutls-extra-config}
that knows about the path to the
include file and other configuration options.  The options that need
to be added to the compiler invocation at compile time are output by
the \option{--cflags} option to \option{libgnutls-config}.  The following
example shows how it can be used at the command line:

\begin{verbatim}
gcc -c foo.c `libgnutls-config --cflags`
\end{verbatim}

Adding the output of \command{libgnutls-config --cflags} to the compilers
command line will ensure that the compiler can find the \gnutls{} header
file.

A similar problem occurs when linking the program with the library.
Again, the compiler has to find the library files.  For this to work,
the path to the library files has to be added to the library search
path (via the -L option).  For this, the option
\option{--libs} to \command{libgnutls-config} can be used.  For
convenience, this option also outputs all other options that are
required to link the program with the \gnutls{} libararies.
The example shows how to link `foo.o'
with the \gnutls{} libraries to a program \emph{foo}.

\begin{verbatim}
gcc -o foo foo.o `libgnutls-config --libs`
\end{verbatim}

Of course you can also combine both examples to a single command by
specifying both options to `libgnutls-config':

\begin{verbatim}
gcc -o foo foo.c `libgnutls-config --cflags --libs`
\end{verbatim}



\label{examples}
\section{Client examples}
This section contains examples of \tls{} and \ssl{} clients, using \gnutls{}. 
Note that these examples contain little or no error checking.

\subsection{Simple client example with X.509 certificate support}
Let's assume now that we want to create a TCP client which communicates
with servers that use X.509 or OpenPGP certificate authentication. The following client
is a very simple \tls{} client, it does not support session resuming, not
even certificate verification. The TCP functions defined in this example
are used in most of the other examples below, without redefining them.
\begin{verbatim}

#include <stdio.h>
#include <stdlib.h>
#include <sys/types.h>
#include <sys/socket.h>
#include <netinet/in.h>
#include <arpa/inet.h>
#include <unistd.h>
#include <gnutls/gnutls.h>

/* A very basic TLS client.
 */

#define MAX_BUF 1024
#define CRLFILE "crl.pem"
#define CAFILE "ca.pem"
#define SA struct sockaddr
#define MSG "GET / HTTP/1.0\r\n\r\n"

int main()
{
   const char *PORT = "443";
   const char *SERVER = "127.0.0.1";
   int err, ret;
   int sd, ii;
   struct sockaddr_in sa;
   gnutls_session session;
   char buffer[MAX_BUF + 1];
   gnutls_certificate_client_credentials xcred;
   const int protocol_priority[] = { GNUTLS_TLS1, GNUTLS_SSL3, 0 };
   const int kx_priority[] = { GNUTLS_KX_RSA, 0 };
   const int cipher_priority[] = { GNUTLS_CIPHER_3DES_CBC, GNUTLS_CIPHER_ARCFOUR_128, 0};
   const int comp_priority[] = { GNUTLS_COMP_NULL, 0 };
   const int mac_priority[] = { GNUTLS_MAC_SHA, GNUTLS_MAC_MD5, 0 };


   gnutls_global_init();

   /* X509 stuff */
   gnutls_certificate_allocate_credentials(&xcred);

   /* set's the trusted cas file
    */
   gnutls_certificate_set_x509_trust_file(xcred, CAFILE, GNUTLS_X509_FMT_PEM);

   /* connects to server 
    */
   sd = socket(AF_INET, SOCK_STREAM, 0);

   memset(&sa, '\0', sizeof(sa));
   sa.sin_family = AF_INET;
   sa.sin_port = htons(atoi(PORT));
   inet_pton(AF_INET, SERVER, &sa.sin_addr);

   err = connect(sd, (SA *) & sa, sizeof(sa));
   if (err < 0) {
      fprintf(stderr, "Connect error\n");
      exit(1);
   }
   /* Initialize TLS session 
    */
   gnutls_init(&session, GNUTLS_CLIENT);

   /* allow both SSL3 and TLS1
    */
   gnutls_protocol_set_priority(session, protocol_priority);

   /* allow only ARCFOUR and 3DES ciphers
    * (3DES has the highest priority)
    */
   gnutls_cipher_set_priority(session, cipher_priority);

   /* only allow null compression
    */
   gnutls_compression_set_priority(session, comp_priority);

   /* use GNUTLS_KX_RSA
    */
   gnutls_kx_set_priority(session, kx_priority);

   /* allow the usage of both SHA and MD5
    */
   gnutls_mac_set_priority(session, mac_priority);


   /* put the x509 credentials to the current session
    */
   gnutls_credentials_set(session, GNUTLS_CRD_CERTIFICATE, xcred);


   gnutls_transport_set_ptr( session, sd);
   /* Perform the TLS handshake
    */
   ret = gnutls_handshake( session);

   if (ret < 0) {
      fprintf(stderr, "*** Handshake failed\n");
      gnutls_perror(ret);
      goto end;
   } else {
      printf("- Handshake was completed\n");
   }

   gnutls_record_send( session, MSG, strlen(MSG));

   ret = gnutls_record_recv( session, buffer, MAX_BUF);
   if (ret == 0) {
      printf("- Peer has closed the TLS connection\n");
      goto end;
   } else if (ret < 0) {
      fprintf(stderr, "*** Received corrupted data(%d) - server has terminated the connection abnormally\n",
              ret);
      goto end;
   } else if (ret > 0) {
      printf("- Received %d bytes: ", ret);
      for (ii = 0; ii < ret; ii++) {
         fputc(buffer[ii], stdout);
      }
      fputs("\n", stdout);
   }
   gnutls_bye( session, GNUTLS_SHUT_RDWR);

 end:

   shutdown(sd, SHUT_RDWR);     /* no more receptions */
   close(sd);

   gnutls_deinit(session);

   gnutls_certificate_free_credentials(xcred);

   gnutls_global_deinit();

   return 0;
}

\end{verbatim}


\subsection{Obtaining session information}
Most of the times it is desirable to know the security properties of
the current established session. This includes the underlying ciphers and
the protocols involved. That is the purpose of the following function.
Note that this function will print meaningful values only if
called after a successful \printfunc{gnutls_handshake}{gnutls\_handshake}

\begin{verbatim}

#include <stdio.h>
#include <stdlib.h>
#include <gnutls/gnutls.h>
#include <gnutls/x509.h>

extern void print_x509_certificate_info(gnutls_session);

/* This function will print some details of the
 * given session.
 */
int print_info(gnutls_session session)
{
   const char *tmp;
   gnutls_credentials_type cred;
   gnutls_kx_algorithm kx;

   /* print the key exchange's algorithm name
    */
   kx = gnutls_kx_get(session);
   tmp = gnutls_kx_get_name(kx);
   printf("- Key Exchange: %s\n", tmp);

   /* Check the authentication type used and switch
    * to the appropriate.
    */
   cred = gnutls_auth_get_type(session);
   switch (cred) {
   case GNUTLS_CRD_ANON:       /* anonymous authentication */

      printf("- Anonymous DH using prime of %d bits\n",
             gnutls_dh_get_prime_bits(session));
      break;

   case GNUTLS_CRD_CERTIFICATE:        /* certificate authentication */
      
      /* Check if we have been using ephemeral Diffie Hellman.
       */
      if (kx == GNUTLS_KX_DHE_RSA || kx == GNUTLS_KX_DHE_DSS) {
         printf("\n- Ephemeral DH using prime of %d bits\n",
                gnutls_dh_get_prime_bits(session));
      }

      /* if the certificate list is available, then
       * print some information about it.
       */
      print_x509_certificate_info(session);

   } /* switch */

   /* print the protocol's name (ie TLS 1.0) 
    */
   tmp = gnutls_protocol_get_name(gnutls_protocol_get_version(session));
   printf("- Protocol: %s\n", tmp);

   /* print the certificate type of the peer.
    * ie X.509
    */
   tmp = gnutls_certificate_type_get_name(
      gnutls_certificate_type_get(session));

   printf("- Certificate Type: %s\n", tmp);

   /* print the compression algorithm (if any)
    */
   tmp = gnutls_compression_get_name( gnutls_compression_get(session));
   printf("- Compression: %s\n", tmp);

   /* print the name of the cipher used.
    * ie 3DES.
    */
   tmp = gnutls_cipher_get_name(gnutls_cipher_get(session));
   printf("- Cipher: %s\n", tmp);

   /* Print the MAC algorithms name.
    * ie SHA1
    */
   tmp = gnutls_mac_get_name(gnutls_mac_get(session));
   printf("- MAC: %s\n", tmp);

   return 0;
}

\end{verbatim}


\subsection{Verifying peer's certificate}
A \tls{} session is not secure just after the handshake procedure has finished.
It must be considered secure, only after the peer's certificate and identity have been
verified. That is, you have to verify the signature in peer's 
certificate, the hostname in the certificate, and expiration dates.
Just after this step you should treat the connection as being a secure one.
The following function is an example on how to verify the peer's certificate chain.
This is an advanced case. Things in a TLS session may be simplified by using
\printfunc{gnutls_certificate_verify_peers2}{gnutls\_certificate\_verify\_peers2}.

\index{Verifying certificate chains}
\label{ex:verify-chain}

\begin{verbatim}

#include <stdio.h>
#include <gnutls/gnutls.h>
#include <gnutls/x509.h>

/* All the available CRLs
 */
extern gnutls_x509_crl* crl_list;
extern int crl_list_size;

/* All the available trusted CAs
 */
extern gnutls_x509_crt* ca_list;
extern int ca_list_size;

static void verify_cert2(gnutls_x509_crt crt,
	gnutls_x509_crt issuer, gnutls_x509_crl * crl_list, int crl_list_size);
static void verify_last_cert(gnutls_x509_crt crt,
   gnutls_x509_crt *ca_list, int ca_list_size,
   gnutls_x509_crl * crl_list, int crl_list_size);


/* This function will try to verify the peer's certificate chain, and
 * also check if the hostname matches, and the activation, expiration dates.
 */
void verify_certificate_chain( gnutls_session session, const char* hostname,
   const gnutls_datum* cert_chain, int cert_chain_length)
{
   int i, ret;
   gnutls_x509_crt cert[cert_chain_length];

   /* Import all the certificates in the chain to
    * native certificate format.
    */
   for (i=0;i<cert_chain_length;i++) {
      gnutls_x509_crt_init(&cert[i]);
      gnutls_x509_crt_import( cert[i], &cert_chain[i], GNUTLS_X509_FMT_DER);
   }

   /* Now verify the certificates against their issuers
    * in the chain.
    */   
   for (i=1;i<cert_chain_length;i++) {
      verify_cert2( cert[i-1], cert[i], crl_list, crl_list_size);
   }

   /* Here we must verify the last certificate in the chain against
    * our trusted CA list.
    */
   verify_last_cert( cert[cert_chain_length-1], 
      ca_list, ca_list_size, crl_list, crl_list_size);

   /* Check if the name in the first certificate matches our destination!
    */
   if ( !gnutls_x509_crt_check_hostname( cert[0], hostname)) {
      printf("The certificate's owner does not match hostname '%s'\n", hostname);
   }

   for (i=0;i<cert_chain_length;i++)
      gnutls_x509_crt_deinit( cert[i]);

   return;
}


/* Verifies a certificate against an other certificate
 * which is supposed to be it's issuer. Also checks the
 * crl_list if the certificate is revoked.
 */
static void verify_cert2(gnutls_x509_crt crt,
   gnutls_x509_crt issuer, gnutls_x509_crl * crl_list, int crl_list_size)
{ 
   unsigned int output;
   int ret;
   time_t now = time(0);
   size_t name_size;
   char name[64];

   /* Print information about the certificates to
    * be checked.
    */
   name_size = sizeof(name);
   gnutls_x509_crt_get_dn( crt, name, &name_size);

   fprintf(stderr, "\nCertificate: %s\n", name);

   name_size = sizeof(name);
   gnutls_x509_crt_get_issuer_dn(crt, name, &name_size);

   fprintf(stderr, "Issued by: %s\n", name);

   /* Get the DN of the issuer cert.
    */
   name_size = sizeof(name);
   gnutls_x509_crt_get_dn(issuer, name, &name_size);

   fprintf(stderr, "Checking against: %s\n", name);

   /* Do the actual verification.
    */
   gnutls_x509_crt_verify(crt, &issuer, 1, 0, &output);

   if (output & GNUTLS_CERT_INVALID) {
      fprintf(stderr, "Not trusted");

      if (output & GNUTLS_CERT_SIGNER_NOT_CA)
         fprintf(stderr, ": Issuer is not a CA\n");
      else
         fprintf(stderr, "\n");
   } else
      fprintf(stderr, "Trusted\n");


    /* Now check the expiration dates.
     */
    if (gnutls_x509_crt_get_activation_time(crt) > now)
	fprintf(stderr, "Not yet activated\n");

    if (gnutls_x509_crt_get_expiration_time(crt) < now)
	fprintf(stderr, "Expired\n");

    /* Check if the certificate is revoked.
     */
    ret = gnutls_x509_crt_check_revocation(crt, crl_list, crl_list_size);
    if (ret == 1) {		/* revoked */
	fprintf(stderr, "Revoked\n");
    }
}


/* Verifies a certificate against the trusted CA list.
 * Also checks the crl_list if the certificate is revoked.
 */
static void verify_last_cert(gnutls_x509_crt crt,
   gnutls_x509_crt *ca_list, int ca_list_size,
   gnutls_x509_crl * crl_list, int crl_list_size)
{ 
   unsigned int output;
   int ret;
   time_t now = time(0);
   size_t name_size;
   char name[64];

   /* Print information about the certificates to
    * be checked.
    */
   name_size = sizeof(name);
   gnutls_x509_crt_get_dn( crt, name, &name_size);

   fprintf(stderr, "\nCertificate: %s\n", name);

   name_size = sizeof(name);
   gnutls_x509_crt_get_issuer_dn(crt, name, &name_size);

   fprintf(stderr, "Issued by: %s\n", name);

   /* Do the actual verification.
    */
   gnutls_x509_crt_verify(crt, ca_list, ca_list_size, 0, &output);

   if (output & GNUTLS_CERT_INVALID) {
      fprintf(stderr, "Not trusted");

      if (output & GNUTLS_CERT_SIGNER_NOT_CA)
         fprintf(stderr, ": Issuer is not a CA\n");
      else
         fprintf(stderr, "\n");
   } else
      fprintf(stderr, "Trusted\n");


    /* Now check the expiration dates.
     */
    if (gnutls_x509_crt_get_activation_time(crt) > now)
	fprintf(stderr, "Not yet activated\n");

    if (gnutls_x509_crt_get_expiration_time(crt) < now)
	fprintf(stderr, "Expired\n");

    /* Check if the certificate is revoked.
     */
    ret = gnutls_x509_crt_check_revocation(crt, crl_list, crl_list_size);
    if (ret == 1) {		/* revoked */
	fprintf(stderr, "Revoked\n");
    }
}

\end{verbatim}


\subsection{Using a callback to select the certificate to use}
There are cases where a client holds several certificate and key pairs,
and may not want to load all of them in the credentials structure.
The following example demonstrates the use of the certificate selection callback.
\par

\begin{verbatim}

#include <stdio.h>
#include <stdlib.h>
#include <string.h>
#include <sys/types.h>
#include <sys/socket.h>
#include <netinet/in.h>
#include <arpa/inet.h>
#include <unistd.h>
#include <sys/mman.h>
#include <sys/stat.h>
#include <gnutls/gnutls.h>
#include <gnutls/x509.h>

/* A TLS client that loads the certificate and key.
 */

#define MAX_BUF 1024
#define SA struct sockaddr
#define MSG "GET / HTTP/1.0\r\n\r\n"

#define CERT_FILE "cert.pem"
#define KEY_FILE "key.pem"
#define CAFILE "ca.pem"

static int cert_callback(gnutls_session session,
                  const gnutls_datum* req_ca_rdn, int nreqs,
                  gnutls_retr_st * st);

gnutls_x509_crt crt;
gnutls_x509_privkey key;

/* Helper functions to load a certificate and key
 * files into memory. They use mmap for simplicity.
 */
static gnutls_datum mmap_file( const char* file)
{
int fd;
gnutls_datum mmaped_file = { NULL, 0 };
struct stat stat_st;
void* ptr;

   fd = open( file, 0);
   if (fd==-1) return mmaped_file;
   
   fstat( fd, &stat_st);
   
   if ((ptr=mmap( NULL, stat_st.st_size, PROT_READ, MAP_SHARED, fd, 0)) == MAP_FAILED)
      return mmaped_file;
   
   mmaped_file.data = ptr;
   mmaped_file.size = stat_st.st_size;
   
   return mmaped_file;
}

static void munmap_file( gnutls_datum data)
{
   munmap( data.data, data.size);
}

/* Load the certificate and the private key.
 */
static void load_keys( void)
{
int ret;
gnutls_datum data;

   data = mmap_file( CERT_FILE);
   if (data.data == NULL) {
      fprintf(stderr, "*** Error loading cert file.\n");
      exit(1);
   }
   gnutls_x509_crt_init( &crt);
   
   ret = gnutls_x509_crt_import( crt, &data, GNUTLS_X509_FMT_PEM);
   if (ret < 0) {
      fprintf(stderr, "*** Error loading key file: %s\n", gnutls_strerror(ret));
      exit(1);
   }

   munmap_file( data);

   data = mmap_file( KEY_FILE);
   if (data.data == NULL) {
      fprintf(stderr, "*** Error loading key file.\n");
      exit(1);
   }

   gnutls_x509_privkey_init( &key);
   
   ret = gnutls_x509_privkey_import( key, &data, GNUTLS_X509_FMT_PEM);
   if (ret < 0) {
      fprintf(stderr, "*** Error loading key file: %s\n", gnutls_strerror(ret));
      exit(1);
   }

   munmap_file( data);
   
}

int main()
{
   int ret, sd, ii;
   gnutls_session session;
   char buffer[MAX_BUF + 1];
   gnutls_certificate_credentials xcred;
   /* Allow connections to servers that have OpenPGP keys as well.
    */

   gnutls_global_init();

   load_keys();

   /* X509 stuff */
   gnutls_certificate_allocate_credentials(&xcred);

   /* sets the trusted cas file
    */
   gnutls_certificate_set_x509_trust_file(xcred, CAFILE, GNUTLS_X509_FMT_PEM);

   gnutls_certificate_client_set_retrieve_function( xcred, cert_callback);
   
   /* Initialize TLS session 
    */
   gnutls_init(&session, GNUTLS_CLIENT);

   /* Use default priorities */
   gnutls_set_default_priority(session);

   /* put the x509 credentials to the current session
    */
   gnutls_credentials_set(session, GNUTLS_CRD_CERTIFICATE, xcred);

   /* connect to the peer
    */
   sd = tcp_connect();

   gnutls_transport_set_ptr( session, (gnutls_transport_ptr)sd);

   /* Perform the TLS handshake
    */
   ret = gnutls_handshake( session);

   if (ret < 0) {
      fprintf(stderr, "*** Handshake failed\n");
      gnutls_perror(ret);
      goto end;
   } else {
      printf("- Handshake was completed\n");
   }

   gnutls_record_send( session, MSG, strlen(MSG));

   ret = gnutls_record_recv( session, buffer, MAX_BUF);
   if (ret == 0) {
      printf("- Peer has closed the TLS connection\n");
      goto end;
   } else if (ret < 0) {
      fprintf(stderr, "*** Error: %s\n", gnutls_strerror(ret));
      goto end;
   }

   printf("- Received %d bytes: ", ret);
   for (ii = 0; ii < ret; ii++) {
      fputc(buffer[ii], stdout);
   }
   fputs("\n", stdout);

   gnutls_bye( session, GNUTLS_SHUT_RDWR);

 end:

   tcp_close( sd);

   gnutls_deinit(session);

   gnutls_certificate_free_credentials(xcred);

   gnutls_global_deinit();

   return 0;
}



/* This callback should be associated with a session by calling
 * gnutls_certificate_client_set_retrieve_function( session, cert_callback),
 * before a handshake.
 */

static int cert_callback(gnutls_session session,
                  const gnutls_datum* req_ca_rdn, int nreqs,
                  gnutls_retr_st * st)
{
   char issuer_dn[256];
   int i, ret;
   size_t len;
   gnutls_certificate_type type;

   /* Print the server's trusted CAs
    */
   if (nreqs > 0)
      printf("- Server's trusted authorities:\n");
   else
      printf("- Server did not send us any trusted authorities names.\n");

   /* print the names (if any) */
   for (i = 0; i < nreqs; i++) {
      len = sizeof(issuer_dn);
      ret = gnutls_x509_rdn_get(&req_ca_rdn[i], issuer_dn, &len);
      if (ret >= 0) {
         printf("   [%d]: ", i);
         printf("%s\n", issuer_dn);
      }
   }

   /* Select a certificate and return it.
    */

   type = gnutls_certificate_type_get( session);
   if (type == GNUTLS_CRT_X509) {
      st->type = type;
      st->ncerts = 1;

      st->cert.x509 = &crt;
      st->key.x509 = key;

      st->deinit_all = 0;
   } else {
      return -1;
   }

   return 0;

}

\end{verbatim}



\subsection{Client with Resume capability example}
\label{resume-example}
This is a modification of the simple client example. Here we demonstrate
the use of session resumption. The client tries to connect once using
\tls{}, close the connection and then try to establish a new connection
using the previously negotiated data.
\begin{verbatim}

#include <stdio.h>
#include <stdlib.h>
#include <gnutls/gnutls.h>

/* Those functions are defined in other examples.
 */
extern void check_alert(gnutls_session session, int ret);
extern int tcp_connect( void);
extern void tcp_close( int sd);

#define MAX_BUF 1024
#define CRLFILE "crl.pem"
#define CAFILE "ca.pem"
#define SA struct sockaddr
#define MSG "GET / HTTP/1.0\r\n\r\n"

int main()
{
   int ret;
   int sd, ii, alert;
   gnutls_session session;
   char buffer[MAX_BUF + 1];
   gnutls_certificate_credentials xcred;

   /* variables used in session resuming 
    */
   int t;
   char *session_data;
   size_t session_data_size;

   gnutls_global_init();

   /* X509 stuff */
   gnutls_certificate_allocate_credentials(&xcred);

   gnutls_certificate_set_x509_trust_file(xcred, CAFILE, GNUTLS_X509_FMT_PEM);

   for (t = 0; t < 2; t++) {    /* connect 2 times to the server */

      sd = tcp_connect();

      gnutls_init(&session, GNUTLS_CLIENT);

      gnutls_set_default_priority(session);

      gnutls_credentials_set(session, GNUTLS_CRD_CERTIFICATE, xcred);

      if (t > 0) { /* if this is not the first time we connect */
         gnutls_session_set_data(session, session_data, session_data_size);
         free(session_data);
      }
      
      gnutls_transport_set_ptr( session, (gnutls_transport_ptr)sd);

      /* Perform the TLS handshake
       */
      ret = gnutls_handshake( session);

      if (ret < 0) {
         fprintf(stderr, "*** Handshake failed\n");
         gnutls_perror(ret);
         goto end;
      } else {
         printf("- Handshake was completed\n");
      }

      if (t == 0) { /* the first time we connect */
         /* get the session data size */
         gnutls_session_get_data(session, NULL, &session_data_size);
         session_data = malloc(session_data_size);

         /* put session data to the session variable */
         gnutls_session_get_data(session, session_data, &session_data_size);

      } else { /* the second time we connect */

         /* check if we actually resumed the previous session */
         if (gnutls_session_is_resumed( session) != 0) {
            printf("- Previous session was resumed\n");
         } else {
            fprintf(stderr, "*** Previous session was NOT resumed\n");
         }
      }

      /* This function was defined in a previous example
       */
      /* print_info(session); */

      gnutls_record_send( session, MSG, strlen(MSG));

      ret = gnutls_record_recv( session, buffer, MAX_BUF);
      if (ret == 0) {
         printf("- Peer has closed the TLS connection\n");
         goto end;
      } else if (ret < 0) {
         fprintf(stderr, "*** Error: %s\n", gnutls_strerror(ret));
         goto end;
      }

      printf("- Received %d bytes: ", ret);
      for (ii = 0; ii < ret; ii++) {
         fputc(buffer[ii], stdout);
      }
      fputs("\n", stdout);

      gnutls_bye( session, GNUTLS_SHUT_RDWR);

    end:

      tcp_close(sd);

      gnutls_deinit(session);

   }  /* for() */

   gnutls_certificate_free_credentials(xcred);

   gnutls_global_deinit();

   return 0;
}

\end{verbatim}


\subsection{Simple client example with SRP authentication}
The following client
is a very simple SRP \tls{} client which connects to a server 
and authenticates using a {\it username} and a {\it password}. The
server may authenticate itself using a certificate, and in that case it
has to be verified.
\begin{verbatim}

#include <stdio.h>
#include <stdlib.h>
#include <gnutls/gnutls.h>
#include <gnutls/extra.h>

/* Those functions are defined in other examples.
 */
extern void check_alert(gnutls_session session, int ret);
extern int tcp_connect( void);
void tcp_close( int sd);

#define MAX_BUF 1024
#define USERNAME "user"
#define PASSWORD "pass"
#define SA struct sockaddr
#define MSG "GET / HTTP/1.0\r\n\r\n"

const int kx_priority[] = { GNUTLS_KX_SRP, 0 };

int main()
{
   int ret;
   int sd, ii;
   gnutls_session session;
   char buffer[MAX_BUF + 1];
   gnutls_srp_client_credentials xcred;

   if (gnutls_global_init() < 0) {
      fprintf(stderr, "global state initialization error\n");
      exit(1);
   }

   /* now enable the gnutls-extra library which contains the
    * SRP stuff. */
   if (gnutls_global_init_extra() < 0) {
      fprintf(stderr, "global state initialization error\n");
      exit(1);
   }

   if (gnutls_srp_allocate_client_credentials(&xcred) < 0) {
      fprintf(stderr, "memory error\n");
      exit(1);
   }
   gnutls_srp_set_client_credentials(xcred, USERNAME, PASSWORD);

   /* connects to server 
    */
   sd = tcp_connect();

   /* Initialize TLS session 
    */
   gnutls_init(&session, GNUTLS_CLIENT);


   /* Set the priorities.
    */
   gnutls_set_default_priority(session);
 
   /* use GNUTLS_KX_SRP
    */
   gnutls_kx_set_priority(session, kx_priority);
 

   /* put the SRP credentials to the current session
    */
   gnutls_credentials_set(session, GNUTLS_CRD_SRP, xcred);

   gnutls_transport_set_ptr( session, (gnutls_transport_ptr)sd);

   /* Perform the TLS handshake
    */
   ret = gnutls_handshake( session);

   if (ret < 0) {
      fprintf(stderr, "*** Handshake failed\n");
      gnutls_perror(ret);
      goto end;
   } else {
      printf("- Handshake was completed\n");
   }

   gnutls_record_send( session, MSG, strlen(MSG));

   ret = gnutls_record_recv( session, buffer, MAX_BUF);
   if (gnutls_error_is_fatal(ret) == 1 || ret == 0) {
      if (ret == 0) {
         printf("- Peer has closed the GNUTLS connection\n");
         goto end;
      } else {
         fprintf(stderr, "*** Error: %s\n", gnutls_strerror(ret));
         goto end;
      }
   } else
      check_alert( session, ret);

   if (ret > 0) {
      printf("- Received %d bytes: ", ret);
      for (ii = 0; ii < ret; ii++) {
         fputc(buffer[ii], stdout);
      }
      fputs("\n", stdout);
   }
   gnutls_bye( session, 0);

 end:

   tcp_close( sd);

   gnutls_deinit(session);

   gnutls_srp_free_client_credentials(xcred);

   gnutls_global_deinit();

   return 0;
}

\end{verbatim}


\section{Server examples}
This section contains examples of \tls{} and \ssl{} servers, using \gnutls{}.

\subsection{Echo Server with X.509 authentication}
This example is a very simple echo server which supports {\bf X.509} authentication,
using the RSA ciphersuites.
\begin{verbatim}

#include <stdio.h>
#include <stdlib.h>
#include <errno.h>
#include <sys/types.h>
#include <sys/socket.h>
#include <netinet/in.h>
#include <arpa/inet.h>
#include <string.h>
#include <unistd.h>
#include <gnutls/gnutls.h>

#define KEYFILE "key.pem"
#define CERTFILE "cert.pem"
#define CAFILE "ca.pem"
#define CRLFILE "crl.pem"

/* This is a sample TLS 1.0 echo server.
 */


#define SA struct sockaddr
#define SOCKET_ERR(err,s) if(err==-1) {perror(s);return(1);}
#define MAX_BUF 1024
#define PORT 5556               /* listen to 5556 port */
#define DH_BITS 1024

/* These are global */
gnutls_certificate_credentials x509_cred;

gnutls_session initialize_tls_session()
{
   gnutls_session session;

   gnutls_init(&session, GNUTLS_SERVER);

   /* avoid calling all the priority functions, since the defaults
    * are adequate.
    */
   gnutls_set_default_priority( session);   

   gnutls_credentials_set(session, GNUTLS_CRD_CERTIFICATE, x509_cred);

   /* request client certificate if any.
    */
   gnutls_certificate_server_set_request( session, GNUTLS_CERT_REQUEST);

   gnutls_dh_set_prime_bits( session, DH_BITS);

   /* some broken clients may require this in order to connect. 
    * This may weaken security though.
    */
   /* gnutls_handshake_set_rsa_pms_check( session, 1); */

   
   return session;
}

gnutls_dh_params dh_params;

static int generate_dh_params(void) {

   /* Generate Diffie Hellman parameters - for use with DHE
    * kx algorithms. These should be discarded and regenerated
    * once a day, once a week or once a month. Depends on the
    * security requirements.
    */
   gnutls_dh_params_init( &dh_params);
   gnutls_dh_params_generate2( dh_params, DH_BITS);
   
   return 0;
}

int main()
{
   int err, listen_sd, i;
   int sd, ret;
   struct sockaddr_in sa_serv;
   struct sockaddr_in sa_cli;
   int client_len;
   char topbuf[512];
   gnutls_session session;
   char buffer[MAX_BUF + 1];
   int optval = 1;
   char name[256];

   strcpy(name, "Echo Server");

   /* this must be called once in the program
    */
   gnutls_global_init();

   gnutls_certificate_allocate_credentials(&x509_cred);
   gnutls_certificate_set_x509_trust_file(x509_cred, CAFILE, 
      GNUTLS_X509_FMT_PEM);

   gnutls_certificate_set_x509_crl_file(x509_cred, CRLFILE, 
      GNUTLS_X509_FMT_PEM);

   gnutls_certificate_set_x509_key_file(x509_cred, CERTFILE, KEYFILE, 
      GNUTLS_X509_FMT_PEM);

   generate_dh_params();
   
   gnutls_certificate_set_dh_params( x509_cred, dh_params);

   /* Socket operations
    */
   listen_sd = socket(AF_INET, SOCK_STREAM, 0);
   SOCKET_ERR(listen_sd, "socket");

   memset(&sa_serv, '\0', sizeof(sa_serv));
   sa_serv.sin_family = AF_INET;
   sa_serv.sin_addr.s_addr = INADDR_ANY;
   sa_serv.sin_port = htons(PORT);  /* Server Port number */

   setsockopt(listen_sd, SOL_SOCKET, SO_REUSEADDR, &optval, sizeof(int));

   err = bind(listen_sd, (SA *) & sa_serv, sizeof(sa_serv));
   SOCKET_ERR(err, "bind");
   err = listen(listen_sd, 1024);
   SOCKET_ERR(err, "listen");

   printf("%s ready. Listening to port '%d'.\n\n", name, PORT);

   client_len = sizeof(sa_cli);
   for (;;) {
      session = initialize_tls_session();

      sd = accept(listen_sd, (SA *) & sa_cli, &client_len);

      printf("- connection from %s, port %d\n",
             inet_ntop(AF_INET, &sa_cli.sin_addr, topbuf,
                       sizeof(topbuf)), ntohs(sa_cli.sin_port));

      gnutls_transport_set_ptr( session, sd);
      ret = gnutls_handshake( session);
      if (ret < 0) {
         close(sd);
         gnutls_deinit(session);
         fprintf(stderr, "*** Handshake has failed (%s)\n\n",
                 gnutls_strerror(ret));
         continue;
      }
      printf("- Handshake was completed\n");

      /* see the Getting peer's information example */
      /* print_info(session); */

      i = 0;
      for (;;) {
         bzero(buffer, MAX_BUF + 1);
         ret = gnutls_record_recv( session, buffer, MAX_BUF);

         if (ret == 0) {
            printf
                ("\n- Peer has closed the GNUTLS connection\n");
            break;
         } else if (ret < 0) {
            fprintf(stderr,
                    "\n*** Received corrupted data(%d). Closing the connection.\n\n",
                    ret);
            break;
         } else if (ret > 0) {
            /* echo data back to the client
             */
            gnutls_record_send( session, buffer,
                         strlen(buffer));
         }
      }
      printf("\n");
      gnutls_bye( session, GNUTLS_SHUT_WR); /* do not wait for
                                 * the peer to close the connection.
                                 */

      close(sd);
      gnutls_deinit(session);

   }
   close(listen_sd);

   gnutls_certificate_free_credentials(x509_cred);

   gnutls_global_deinit();

   return 0;

}

\end{verbatim}


\subsection{Echo Server with X.509 authentication II}
The following example is a server which supports {\bf X.509} authentication.
This server supports the export-grade cipher suites, the DHE ciphersuites
and session resuming.
\begin{verbatim}

#include <stdio.h>
#include <stdlib.h>
#include <errno.h>
#include <sys/types.h>
#include <sys/socket.h>
#include <netinet/in.h>
#include <arpa/inet.h>
#include <string.h>
#include <unistd.h>
#include <gnutls/gnutls.h>

#define KEYFILE "key.pem"
#define CERTFILE "cert.pem"
#define CAFILE "ca.pem"
#define CRLFILE NULL

/* This is a sample TLS 1.0 echo server.
 * Export-grade ciphersuites and session resuming are supported.
 */

#define SA struct sockaddr
#define SOCKET_ERR(err,s) if(err==-1) {perror(s);return(1);}
#define MAX_BUF 1024
#define PORT 5556               /* listen to 5556 port */
#define DH_BITS 1024

/* These are global */
gnutls_certificate_server_credentials x509_cred;

static void wrap_db_init(void);
static void wrap_db_deinit(void);
static int wrap_db_store(void *dbf, gnutls_datum key, gnutls_datum data);
static gnutls_datum wrap_db_fetch(void *dbf, gnutls_datum key);
static int wrap_db_delete(void *dbf, gnutls_datum key);

#define TLS_SESSION_CACHE 50

gnutls_session initialize_tls_session()
{
   gnutls_session session;

   gnutls_init(&session, GNUTLS_SERVER);

   /* Use the default priorities, plus, export cipher suites.
    */
   gnutls_set_default_export_priority(session);

   gnutls_credentials_set(session, GNUTLS_CRD_CERTIFICATE, x509_cred);

   /* request client certificate if any.
    */
   gnutls_certificate_server_set_request(session, GNUTLS_CERT_REQUEST);

   gnutls_dh_set_prime_bits(session, DH_BITS);

   /* some broken clients may require this in order to connect. 
    * This will weaken security though.
    */
   /* gnutls_handshake_set_rsa_pms_check( session, 1); */

   if (TLS_SESSION_CACHE != 0) {
      gnutls_db_set_retrieve_function(session, wrap_db_fetch);
      gnutls_db_set_remove_function(session, wrap_db_delete);
      gnutls_db_set_store_function(session, wrap_db_store);
      gnutls_db_set_ptr(session, NULL);
   }

   return session;
}

gnutls_dh_params dh_params;
/* Export-grade cipher suites require temporary RSA
 * keys.
 */
gnutls_rsa_params rsa_params;

static int generate_dh_params(void)
{
   gnutls_datum prime, generator;

   /* Generate Diffie Hellman parameters - for use with DHE
    * kx algorithms. These should be discarded and regenerated
    * once a day, once a week or once a month. Depends on the
    * security requirements.
    */
   gnutls_dh_params_init(&dh_params);
   gnutls_dh_params_generate(&prime, &generator, DH_BITS);
   gnutls_dh_params_set(dh_params, prime, generator, DH_BITS);

   free(prime.data);
   free(generator.data);
   
   return 0;
}

static int generate_rsa_params(void)
{
   gnutls_datum m, e, d, p, q, u;

   gnutls_rsa_params_init(&rsa_params);

   /* Generate RSA parameters - for use with RSA-export
    * cipher suites. These should be discarded and regenerated
    * once a day, once every 500 transactions etc. Depends on the
    * security requirements.
    */

   gnutls_rsa_params_generate(&m, &e, &d, &p, &q, &u, 512);
   gnutls_rsa_params_set(rsa_params, m, e, d, p, q, u, 512);

   free(m.data);
   free(e.data);
   free(d.data);
   free(p.data);
   free(q.data);
   free(u.data);

   return 0;
}

int main()
{
   int err, listen_sd, i;
   int sd, ret;
   struct sockaddr_in sa_serv;
   struct sockaddr_in sa_cli;
   int client_len;
   char topbuf[512];
   gnutls_session session;
   char buffer[MAX_BUF + 1];
   int optval = 1;
   char name[256];

   strcpy(name, "Echo Server");

   /* this must be called once in the program
    */
   gnutls_global_init();

   gnutls_certificate_allocate_credentials(&x509_cred);

   gnutls_certificate_set_x509_trust_file(x509_cred, CAFILE,
                                          GNUTLS_X509_FMT_PEM);

   gnutls_certificate_set_x509_key_file(x509_cred, CERTFILE, KEYFILE,
                                        GNUTLS_X509_FMT_PEM);

   generate_dh_params();
   generate_rsa_params();

   if (TLS_SESSION_CACHE != 0) {
      wrap_db_init();
   }

   gnutls_certificate_set_dh_params(x509_cred, dh_params);
   gnutls_certificate_set_rsa_params(x509_cred, rsa_params);

   /* Socket operations
    */
   listen_sd = socket(AF_INET, SOCK_STREAM, 0);
   SOCKET_ERR(listen_sd, "socket");

   memset(&sa_serv, '\0', sizeof(sa_serv));
   sa_serv.sin_family = AF_INET;
   sa_serv.sin_addr.s_addr = INADDR_ANY;
   sa_serv.sin_port = htons(PORT);      /* Server Port number */

   setsockopt(listen_sd, SOL_SOCKET, SO_REUSEADDR, &optval, sizeof(int));

   err = bind(listen_sd, (SA *) & sa_serv, sizeof(sa_serv));
   SOCKET_ERR(err, "bind");
   err = listen(listen_sd, 1024);
   SOCKET_ERR(err, "listen");

   printf("%s ready. Listening to port '%d'.\n\n", name, PORT);

   client_len = sizeof(sa_cli);
   for (;;) {
      session = initialize_tls_session();

      sd = accept(listen_sd, (SA *) & sa_cli, &client_len);

      printf("- connection from %s, port %d\n",
             inet_ntop(AF_INET, &sa_cli.sin_addr, topbuf,
                       sizeof(topbuf)), ntohs(sa_cli.sin_port));

      gnutls_transport_set_ptr(session, sd);
      ret = gnutls_handshake(session);
      if (ret < 0) {
         close(sd);
         gnutls_deinit(session);
         fprintf(stderr, "*** Handshake has failed (%s)\n\n",
                 gnutls_strerror(ret));
         continue;
      }
      printf("- Handshake was completed\n");

      /* print_info(session); */

      i = 0;
      for (;;) {
         bzero(buffer, MAX_BUF + 1);
         ret = gnutls_record_recv(session, buffer, MAX_BUF);

         if (ret == 0) {
            printf("\n- Peer has closed the TLS connection\n");
            break;
         } else if (ret < 0) {
            fprintf(stderr,
                    "\n*** Received corrupted data(%d). Closing the connection.\n\n",
                    ret);
            break;
         } else if (ret > 0) {
            /* echo data back to the client
             */
            gnutls_record_send(session, buffer, strlen(buffer));
         }
      }
      printf("\n");
      gnutls_bye(session, GNUTLS_SHUT_WR);      /* do not wait for
                                                   * the peer to close the connection.
                                                 */

      close(sd);
      gnutls_deinit(session);

   }
   close(listen_sd);

   gnutls_certificate_free_credentials(x509_cred);

   gnutls_global_deinit();

   return 0;

}


/* Functions and other stuff needed for session resuming.
 * This is done using a very simple list which holds session ids
 * and session data.
 */

#define MAX_SESSION_ID_SIZE 32
#define MAX_SESSION_DATA_SIZE 512

typedef struct {
   char session_id[MAX_SESSION_ID_SIZE];
   int session_id_size;

   char session_data[MAX_SESSION_DATA_SIZE];
   int session_data_size;
} CACHE;

static CACHE *cache_db;
static int cache_db_ptr = 0;

static void wrap_db_init(void)
{

   /* allocate cache_db */
   cache_db = calloc(1, TLS_SESSION_CACHE * sizeof(CACHE));
}

static void wrap_db_deinit(void)
{
   return;
}

static int wrap_db_store(void *dbf, gnutls_datum key, gnutls_datum data)
{

   if (cache_db == NULL)
      return -1;

   if (key.size > MAX_SESSION_ID_SIZE)
      return -1;
   if (data.size > MAX_SESSION_DATA_SIZE)
      return -1;

   memcpy(cache_db[cache_db_ptr].session_id, key.data, key.size);
   cache_db[cache_db_ptr].session_id_size = key.size;

   memcpy(cache_db[cache_db_ptr].session_data, data.data, data.size);
   cache_db[cache_db_ptr].session_data_size = data.size;

   cache_db_ptr++;
   cache_db_ptr %= TLS_SESSION_CACHE;

   return 0;
}

static gnutls_datum wrap_db_fetch(void *dbf, gnutls_datum key)
{
   gnutls_datum res = { NULL, 0 };
   int i;

   if (cache_db == NULL)
      return res;

   for (i = 0; i < TLS_SESSION_CACHE; i++) {
      if (key.size == cache_db[i].session_id_size &&
          memcmp(key.data, cache_db[i].session_id, key.size) == 0) {


         res.size = cache_db[i].session_data_size;

         res.data = gnutls_malloc(res.size);
         if (res.data == NULL)
            return res;

         memcpy(res.data, cache_db[i].session_data, res.size);

         return res;
      }
   }
   return res;
}

static int wrap_db_delete(void *dbf, gnutls_datum key)
{
   int i;

   if (cache_db == NULL)
      return -1;

   for (i = 0; i < TLS_SESSION_CACHE; i++) {
      if (key.size == cache_db[i].session_id_size &&
          memcmp(key.data, cache_db[i].session_id, key.size) == 0) {

         cache_db[i].session_id_size = 0;
         cache_db[i].session_data_size = 0;

         return 0;
      }
   }

   return -1;

}

\end{verbatim}


\subsection{Echo Server with OpenPGP\index{OpenPGP!Server} authentication}
The following example is an echo server which supports {\bf OpenPGP} key 
authentication. You can easily combine this functionality --that is have
a server that supports both X.509 and OpenPGP certificates-- but we
separated them to keep these examples as simple as possible.
\begin{verbatim}

#include <stdio.h>
#include <stdlib.h>
#include <errno.h>
#include <sys/types.h>
#include <sys/socket.h>
#include <netinet/in.h>
#include <arpa/inet.h>
#include <string.h>
#include <unistd.h>
#include <gnutls/gnutls.h>
/* Must be linked against gnutls-extra.
 */
#include <gnutls/extra.h>

#define KEYFILE "secret.asc"
#define CERTFILE "public.asc"
#define RINGFILE "ring.gpg"

/* This is a sample TLS 1.0-OpenPGP echo server.
 */


#define SA struct sockaddr
#define SOCKET_ERR(err,s) if(err==-1) {perror(s);return(1);}
#define MAX_BUF 1024
#define PORT 5556               /* listen to 5556 port */
#define DH_BITS 1024

/* These are global */
gnutls_certificate_credentials cred;

gnutls_session initialize_tls_session()
{
   gnutls_session session;
   const int cert_type_priority[2] = { GNUTLS_CRT_OPENPGP, 0 };

   gnutls_init(&session, GNUTLS_SERVER);

   /* avoid calling all the priority functions, since the defaults
    * are adequate.
    */
   gnutls_set_default_priority( session);
   gnutls_certificate_type_set_priority(session, cert_type_priority);

   gnutls_credentials_set(session, GNUTLS_CRD_CERTIFICATE, cred);

   /* request client certificate.
    */
   gnutls_certificate_server_set_request( session, GNUTLS_CERT_REQUEST);

   gnutls_dh_set_prime_bits( session, DH_BITS);

   return session;
}

gnutls_dh_params dh_params;

static int generate_dh_params(void) {
gnutls_datum prime, generator;

   /* Generate Diffie Hellman parameters - for use with DHE
    * kx algorithms. These should be discarded and regenerated
    * once a day, once a week or once a month. Depends on the
    * security requirements.
    */
   gnutls_dh_params_init( &dh_params);
   gnutls_dh_params_generate( &prime, &generator, DH_BITS);
   gnutls_dh_params_set( dh_params, prime, generator, DH_BITS);

   gnutls_free( prime.data);
   gnutls_free( generator.data);
   
   return 0;
}

int main()
{
   int err, listen_sd, i;
   int sd, ret;
   struct sockaddr_in sa_serv;
   struct sockaddr_in sa_cli;
   int client_len;
   char topbuf[512];
   gnutls_session session;
   char buffer[MAX_BUF + 1];
   int optval = 1;
   char name[256];

   strcpy(name, "Echo Server");

   /* this must be called once in the program
    */
   gnutls_global_init();

   gnutls_certificate_allocate_credentials( &cred);
   gnutls_certificate_set_openpgp_keyring_file( cred, RINGFILE);

   gnutls_certificate_set_openpgp_key_file( cred, CERTFILE, KEYFILE);

   generate_dh_params();
   
   gnutls_certificate_set_dh_params( cred, dh_params);

   /* Socket operations
    */
   listen_sd = socket(AF_INET, SOCK_STREAM, 0);
   SOCKET_ERR(listen_sd, "socket");

   memset(&sa_serv, '\0', sizeof(sa_serv));
   sa_serv.sin_family = AF_INET;
   sa_serv.sin_addr.s_addr = INADDR_ANY;
   sa_serv.sin_port = htons(PORT);  /* Server Port number */

   setsockopt(listen_sd, SOL_SOCKET, SO_REUSEADDR, &optval, sizeof(int));

   err = bind(listen_sd, (SA *) & sa_serv, sizeof(sa_serv));
   SOCKET_ERR(err, "bind");
   err = listen(listen_sd, 1024);
   SOCKET_ERR(err, "listen");

   printf("%s ready. Listening to port '%d'.\n\n", name, PORT);

   client_len = sizeof(sa_cli);
   for (;;) {
      session = initialize_tls_session();

      sd = accept(listen_sd, (SA *) & sa_cli, &client_len);

      printf("- connection from %s, port %d\n",
             inet_ntop(AF_INET, &sa_cli.sin_addr, topbuf,
                       sizeof(topbuf)), ntohs(sa_cli.sin_port));

      gnutls_transport_set_ptr( session, (gnutls_transport_ptr)sd);
      ret = gnutls_handshake( session);
      if (ret < 0) {
         close(sd);
         gnutls_deinit(session);
         fprintf(stderr, "*** Handshake has failed (%s)\n\n",
                 gnutls_strerror(ret));
         continue;
      }
      printf("- Handshake was completed\n");

      /* see the Getting peer's information example */
      /* print_info(session); */

      i = 0;
      for (;;) {
         bzero(buffer, MAX_BUF + 1);
         ret = gnutls_record_recv( session, buffer, MAX_BUF);

         if (ret == 0) {
            printf
                ("\n- Peer has closed the GNUTLS connection\n");
            break;
         } else if (ret < 0) {
            fprintf(stderr,
                    "\n*** Received corrupted data(%d). Closing the connection.\n\n",
                    ret);
            break;
         } else if (ret > 0) {
            /* echo data back to the client
             */
            gnutls_record_send( session, buffer,
                         strlen(buffer));
         }
      }
      printf("\n");
      gnutls_bye( session, GNUTLS_SHUT_WR); /* do not wait for
                                 * the peer to close the connection.
                                 */

      close(sd);
      gnutls_deinit(session);

   }
   close(listen_sd);

   gnutls_certificate_free_credentials( cred);

   gnutls_global_deinit();

   return 0;

}

\end{verbatim}



\subsection{Echo Server with SRP authentication}
This is a server which supports {\bf SRP} authentication. It is also
possible to combine this functionality with a certificate server. Here it
is separate for simplicity.
\begin{verbatim}

#include <stdio.h>
#include <stdlib.h>
#include <errno.h>
#include <sys/types.h>
#include <sys/socket.h>
#include <netinet/in.h>
#include <arpa/inet.h>
#include <string.h>
#include <unistd.h>
#include <gnutls/gnutls.h>
#include <gnutls/extra.h>

#define SRP_PASSWD "tpasswd"
#define SRP_PASSWD_CONF "tpasswd.conf"

/* This is a sample TLS-SRP echo server.
 */

#define SA struct sockaddr
#define SOCKET_ERR(err,s) if(err==-1) {perror(s);return(1);}
#define MAX_BUF 1024
#define PORT 5556               /* listen to 5556 port */

/* These are global */
gnutls_srp_server_credentials srp_cred;

gnutls_session initialize_tls_session()
{
   gnutls_session session;
   const int protocol_priority[] = { GNUTLS_TLS1, GNUTLS_SSL3, 0 };
   const int kx_priority[] = { GNUTLS_KX_SRP, 0 };
   const int cipher_priority[] = { GNUTLS_CIPHER_RIJNDAEL_CBC, GNUTLS_CIPHER_3DES_CBC, 0};
   const int comp_priority[] = { GNUTLS_COMP_NULL, 0 };
   const int mac_priority[] = { GNUTLS_MAC_SHA, GNUTLS_MAC_MD5, 0 };

   gnutls_init(&session, GNUTLS_SERVER);

   gnutls_protocol_set_priority(session, protocol_priority);
   gnutls_cipher_set_priority(session, cipher_priority);
   gnutls_compression_set_priority(session, comp_priority);
   gnutls_kx_set_priority(session, kx_priority);
   gnutls_mac_set_priority(session, mac_priority);

   gnutls_credentials_set(session, GNUTLS_CRD_SRP, srp_cred);

   /* request client certificate if any.
    */
   gnutls_certificate_server_set_request( session, GNUTLS_CERT_IGNORE);

   return session;
}

int main()
{
   int err, listen_sd, i;
   int sd, ret;
   struct sockaddr_in sa_serv;
   struct sockaddr_in sa_cli;
   int client_len;
   char topbuf[512];
   gnutls_session session;
   char buffer[MAX_BUF + 1];
   int optval = 1;
   char name[256];

   strcpy(name, "Echo Server");

   /* these must be called once in the program
    */
   gnutls_global_init();
   gnutls_global_init_extra(); /* for SRP */

   /* SRP_PASSWD a password file (created with the included srpcrypt utility) 
    */
   gnutls_srp_allocate_server_credentials(&srp_cred);
   gnutls_srp_set_server_credentials_file(srp_cred, SRP_PASSWD, SRP_PASSWD_CONF);


   /* TCP socket operations
    */
   listen_sd = socket(AF_INET, SOCK_STREAM, 0);
   SOCKET_ERR(listen_sd, "socket");

   memset(&sa_serv, '\0', sizeof(sa_serv));
   sa_serv.sin_family = AF_INET;
   sa_serv.sin_addr.s_addr = INADDR_ANY;
   sa_serv.sin_port = htons(PORT);  /* Server Port number */

   setsockopt(listen_sd, SOL_SOCKET, SO_REUSEADDR, &optval, sizeof(int));

   err = bind(listen_sd, (SA *) & sa_serv, sizeof(sa_serv));
   SOCKET_ERR(err, "bind");
   err = listen(listen_sd, 1024);
   SOCKET_ERR(err, "listen");

   printf("%s ready. Listening to port '%d'.\n\n", name, PORT);

   client_len = sizeof(sa_cli);
   for (;;) {
      session = initialize_tls_session();

      sd = accept(listen_sd, (SA *) & sa_cli, &client_len);

      printf("- connection from %s, port %d\n",
             inet_ntop(AF_INET, &sa_cli.sin_addr, topbuf,
                       sizeof(topbuf)), ntohs(sa_cli.sin_port));

      gnutls_transport_set_ptr( session, sd);
      ret = gnutls_handshake( session);
      if (ret < 0) {
         close(sd);
         gnutls_deinit(session);
         fprintf(stderr, "*** Handshake has failed (%s)\n\n",
                 gnutls_strerror(ret));
         continue;
      }
      printf("- Handshake was completed\n");

      /* print_info(session); */

      i = 0;
      for (;;) {
         bzero(buffer, MAX_BUF + 1);
         ret = gnutls_record_recv( session, buffer, MAX_BUF);

         if (ret == 0) {
            printf
                ("\n- Peer has closed the GNUTLS connection\n");
            break;
         } else if (ret > 0) {
            fprintf(stderr,
                    "\n*** Received corrupted data(%d). Closing the connection.\n\n",
                    ret);
            break;
         } else if (ret > 0) {
            /* echo data back to the client
             */
            gnutls_record_send( session, buffer,
                         strlen(buffer));
         }
      }
      printf("\n");
      gnutls_bye( session, GNUTLS_SHUT_WR); /* do not wait for
                                 * the peer to close the connection.
                                 */

      close(sd);
      gnutls_deinit(session);

   }
   close(listen_sd);

   gnutls_srp_free_server_credentials(srp_cred);

   gnutls_global_deinit();

   return 0;

}

\end{verbatim}


\section{Miscellaneous examples}

\subsection{Checking for an alert}
This is a function that checks if an alert has been received
in the current session.
\begin{verbatim}

#include <stdio.h>
#include <stdlib.h>
#include <gnutls/gnutls.h>

/* This function will check whether the given return code from
 * a gnutls function (recv/send), is an alert, and will print
 * that alert.
 */
void check_alert(gnutls_session session, int ret)
{
   int last_alert;

   if (ret == GNUTLS_E_WARNING_ALERT_RECEIVED
       || ret == GNUTLS_E_FATAL_ALERT_RECEIVED) {
      last_alert = gnutls_alert_get(session);

      /* The check for renegotiation is only useful if we are 
       * a server, and we had requested a rehandshake.
       */
      if (last_alert == GNUTLS_A_NO_RENEGOTIATION &&
          ret == GNUTLS_E_WARNING_ALERT_RECEIVED)
         printf("* Received NO_RENEGOTIATION alert. "
                "Client Does not support renegotiation.\n");
      else
         printf("* Received alert '%d': %s.\n", last_alert,
                gnutls_alert_get_name(last_alert));
   }
}

\end{verbatim}


\subsection{X.509 certificate parsing example}
\label{ex:x509-info}
To demonstrate the X.509 parsing capabilities an example program is listed below.
That program reads the peer's certificate, and prints information about it.
\begin{verbatim}

#include <stdio.h>
#include <stdlib.h>
#include <gnutls/gnutls.h>
#include <gnutls/x509.h>

static const char* bin2hex( const void* bin, size_t bin_size)
{
static char printable[110];
unsigned char *_bin = bin;
char* print;

   if (bin_size > 50) bin_size = 50;

   print = printable;
   for (i = 0; i < bin_size; i++) {
      sprintf(print, "%.2x ", _bin[i]);
      print += 2;
   }

   return printable;
}

/* This function will print information about this session's peer
 * certificate. 
 */
static void print_x509_certificate_info(gnutls_session session)
{
   char serial[40];
   char dn[128];
   int i;
   size_t size;
   unsigned int algo, bits;
   time_t expiration_time, activation_time;
   const gnutls_datum *cert_list;
   int cert_list_size = 0;
   gnutls_x509_crt cert;

   /* This function only works for X.509 certificates.
    */
   if (gnutls_certificate_type_get(session) != GNUTLS_CRT_X509)
      return;

   cert_list = gnutls_certificate_get_peers(session, &cert_list_size);

   printf("Peer provided %d certificates.\n", cert_list_size);

   if (cert_list_size > 0) {

      /* we only print information about the first certificate.
       */
      gnutls_x509_crt_init( &cert);

      gnutls_x509_crt_import( cert, &cert_list[0]);

      printf("Certificate info:\n");

      expiration_time = gnutls_x509_crt_get_expiration_time( cert);
      activation_time = gnutls_x509_crt_get_activation_time( cert);

      printf("\tCertificate is valid since: %s", ctime(&activation_time));
      printf("\tCertificate expires: %s", ctime(&expiration_time));

      /* Print the serial number of the certificate.
       */
      size = sizeof(serial);
      gnutls_x509_crt_get_serial(cert, serial, &size);

      size = sizeof( serial);
      printf("\tCertificate serial number: %s\n", 
         bin2hex( serial, size));

      /* Extract some of the public key algorithm's parameters
       */
      algo =
          gnutls_x509_crt_get_pk_algorithm(cert, &bits);

      printf("Certificate public key: ");

      if (algo == GNUTLS_PK_RSA) {
         printf("RSA\n");
         printf(" Modulus: %d bits\n", bits);
      } else if (algo == GNUTLS_PK_DSA) {
         printf("DSA\n");
         printf(" Exponent: %d bits\n", bits);
      } else {
         printf("UNKNOWN\n");
      }

      /* Print the version of the X.509 
       * certificate.
       */
      printf("\tCertificate version: #%d\n",
             gnutls_x509_crt_get_version( cert));

      size = sizeof(dn);
      gnutls_x509_crt_get_dn( cert, dn, &size);
      printf("\tDN: %s\n", dn);

      size = sizeof(dn);
      gnutls_x509_crt_get_issuer_dn( cert, dn, &size);
      printf("\tIssuer's DN: %s\n", dn);

      gnutls_x509_crt_deinit( cert);

   }
}

\end{verbatim}


\subsection{Certificate request generation}
\label{ex:crq}
The following example is about generating a certificate request, and
a private key. A certificate request can be later be processed by a CA,
which should return a signed certificate.

\begin{verbatim}

#include <stdio.h>
#include <stdlib.h>
#include <gnutls/gnutls.h>
#include <gnutls/x509.h>
#include <time.h>

/* This example will generate a private key and a certificate
 * request.
 */

int main()
{
   gnutls_x509_crq_t crq;
   gnutls_x509_privkey_t key;
   unsigned char buffer[10*1024];
   int buffer_size = sizeof(buffer);
   int ret;

   gnutls_global_init();

   /* Initialize an empty certificate request, and
    * an empty private key.
    */
   gnutls_x509_crq_init(&crq);

   gnutls_x509_privkey_init(&key);

   /* Generate a 1024 bit RSA private key.
    */
   gnutls_x509_privkey_generate(key, GNUTLS_PK_RSA, 1024, 0);

   /* Add stuff to the distinguished name
    */
   gnutls_x509_crq_set_dn_by_oid(crq, GNUTLS_OID_X520_COUNTRY_NAME,
				     0, "GR", 2);

   gnutls_x509_crq_set_dn_by_oid(crq, GNUTLS_OID_X520_COMMON_NAME,
				     0, "Nikos", strlen("Nikos"));

   /* Set the request version.
    */
   gnutls_x509_crq_set_version(crq, 1);

   /* Set a challenge password.
    */
   gnutls_x509_crq_set_challenge_password(crq, "something to remember here");

   /* Associate the request with the private key
    */
   gnutls_x509_crq_set_key(crq, key);

   /* Self sign the certificate request.
    */
   gnutls_x509_crq_sign(crq, key);

   /* Export the PEM encoded certificate request, and
    * display it.
    */
   gnutls_x509_crq_export(crq, GNUTLS_X509_FMT_PEM, buffer,
			      &buffer_size);

   printf("Certificate Request: \n%s", buffer);


   /* Export the PEM encoded private key, and
    * display it.
    */
   buffer_size = sizeof(buffer);
   gnutls_x509_privkey_export(key, GNUTLS_X509_FMT_PEM, buffer,
				  &buffer_size);

   printf("\n\nPrivate key: \n%s", buffer);

   gnutls_x509_crq_deinit(crq);
   gnutls_x509_privkey_deinit(key);

   return 0;

}

\end{verbatim}


\subsection{PKCS \#12 structure generation}
\label{ex:pkcs12}
The following example is about generating a PKCS \#12 structure.

\begin{verbatim}

#include <stdio.h>
#include <stdlib.h>
#include <gnutls/gnutls.h>
#include <gnutls/x509.h>

#define OUTFILE "out.p12"

/* This function will write a pkcs12 structure into a file.
 * cert: is a DER encoded certificate
 * pkcs8_key: is a PKCS #8 encrypted key
 * password: is the password used to encrypt the PKCS #12 packet.
 */
int write_pkcs12(const gnutls_datum * cert, const gnutls_datum * pkcs8_key,
                 const char *password)
{
   gnutls_pkcs12 pkcs12;
   int ret, bag_index;
   gnutls_pkcs12_bag bag, key_bag;
   char pkcs12_struct[10 * 1024];
   int pkcs12_struct_size;
   FILE *fd;

   /* A good idea might be to use gnutls_x509_privkey_get_key_id()
    * to obtain a unique ID.
    */
   gnutls_datum key_id = { "\x00\x00\x00\x01", 4 };

   gnutls_global_init();
   gnutls_global_set_log_level(2);

   /* Firstly we create two helper bags, which hold the certificate,
    * and the (encrypted) key.
    */

   gnutls_pkcs12_bag_init(&bag);
   gnutls_pkcs12_bag_init(&key_bag);

   ret = gnutls_pkcs12_bag_set_data(bag, GNUTLS_BAG_CERTIFICATE, cert);
   if (ret < 0) {
      fprintf(stderr, "ret: %s\n", gnutls_strerror(ret));
      exit(1);
   }

   /* ret now holds the bag's index.
    */
   bag_index = ret;

   /* Associate a friendly name with the given certificate. Used
    * by browsers.
    */
   gnutls_pkcs12_bag_set_friendly_name(bag, bag_index, "My name");

   /* Associate the certificate with the key using a unique key
    * ID.
    */
   gnutls_pkcs12_bag_set_key_id(bag, bag_index, &key_id);

   gnutls_pkcs12_bag_encrypt(bag, password, 0);

   /* Now the key.
    */

   ret = gnutls_pkcs12_bag_set_data(key_bag,
                                    GNUTLS_BAG_PKCS8_ENCRYPTED_KEY,
                                    &pkcs8_key);
   if (ret < 0) {
      fprintf(stderr, "ret: %s\n", gnutls_strerror(ret));
      exit(1);
   }

   /* Note that since the PKCS #8 key is encrypted we don't
    * bother encrypting the bag.
    */
   bag_index = ret;

   gnutls_pkcs12_bag_set_friendly_name(key_bag, bag_index, "My name");

   gnutls_pkcs12_bag_set_key_id(key_bag, bag_index, &key_id);


   /* The bags were filled. Now create the PKCS #12 structure.
    */
   gnutls_pkcs12_init(&pkcs12);

   /* Insert the two bags in the PKCS #12 structure.
    */

   gnutls_pkcs12_set_bag(pkcs12, bag);
   gnutls_pkcs12_set_bag(pkcs12, key_bag);


   /* Generate a message authentication code for the PKCS #12
    * structure.
    */
   gnutls_pkcs12_generate_mac(pkcs12, password);

   pkcs12_struct_size = sizeof(pkcs12_struct);
   ret =
       gnutls_pkcs12_export(pkcs12, GNUTLS_X509_FMT_DER, pkcs12_struct,
                            &pkcs12_struct_size);
   if (ret < 0) {
      fprintf(stderr, "ret: %s\n", gnutls_strerror(size));
      exit(1);
   }

   fd = fopen(OUTFILE, "w");
   if (fd == NULL) {
      fprintf(stderr, "cannot open file\n");
      exit(1);
   }
   fwrite(pkcs12_struct, 1, pkcs12_struct_size, fd);
   fclose(fd);

   gnutls_pkcs12_bag_deinit(bag);
   gnutls_pkcs12_bag_deinit(key_bag);
   gnutls_pkcs12_deinit(pkcs12);

}

\end{verbatim}




\section{Compatibility with the OpenSSL\index{OpenSSL} library}

To ease \gnutls{}' integration with existing applications, a compatibility 
layer with the widely used OpenSSL library is included in the \emph{gnutls-openssl}
library. This compatibility layer is not complete and it is not 
intended to completely reimplement the OpenSSL API with \gnutls{}.
It only provides source-level compatibility. There is currently no
attempt to make it binary-compatible with OpenSSL.

Prototypes for the compatibility functions are found in the 
``gnutls/openssl.h'' header file.

Current limitations imposed by the compatibility layer include:

\begin{itemize}

\item Error handling is not thread safe.

\end{itemize}


