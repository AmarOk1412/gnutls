\chapter{How to use GNUTLS\index{Example programs} in applications}
\label{examples}

\section{Client examples}
This section contains examples of \tls{} and \ssl{} clients, using \gnutls{}. 
Note that these examples contain little or no error checking.

\subsection{Simple client example with X.509 certificate support}
Let's assume now that we want to create a client which communicates
with servers using the X.509 authentication schema. The following client
is a very simple \tls{} client, it does not support session resuming nor
any other fancy features.
\begin{verbatim}

#include <stdio.h>
#include <stdlib.h>
#include <sys/types.h>
#include <sys/socket.h>
#include <netinet/in.h>
#include <arpa/inet.h>
#include <unistd.h>
#include <gnutls/gnutls.h>

/* A very basic TLS client.
 */

#define MAX_BUF 1024
#define CRLFILE "crl.pem"
#define CAFILE "ca.pem"
#define SA struct sockaddr
#define MSG "GET / HTTP/1.0\r\n\r\n"

int main()
{
   const char *PORT = "443";
   const char *SERVER = "127.0.0.1";
   int err, ret;
   int sd, ii;
   struct sockaddr_in sa;
   gnutls_session session;
   char buffer[MAX_BUF + 1];
   gnutls_certificate_client_credentials xcred;
   const int protocol_priority[] = { GNUTLS_TLS1, GNUTLS_SSL3, 0 };
   const int kx_priority[] = { GNUTLS_KX_RSA, 0 };
   const int cipher_priority[] = { GNUTLS_CIPHER_3DES_CBC, GNUTLS_CIPHER_ARCFOUR_128, 0};
   const int comp_priority[] = { GNUTLS_COMP_NULL, 0 };
   const int mac_priority[] = { GNUTLS_MAC_SHA, GNUTLS_MAC_MD5, 0 };


   gnutls_global_init();

   /* X509 stuff */
   gnutls_certificate_allocate_credentials(&xcred);

   /* set's the trusted cas file
    */
   gnutls_certificate_set_x509_trust_file(xcred, CAFILE, GNUTLS_X509_FMT_PEM);

   /* connects to server 
    */
   sd = socket(AF_INET, SOCK_STREAM, 0);

   memset(&sa, '\0', sizeof(sa));
   sa.sin_family = AF_INET;
   sa.sin_port = htons(atoi(PORT));
   inet_pton(AF_INET, SERVER, &sa.sin_addr);

   err = connect(sd, (SA *) & sa, sizeof(sa));
   if (err < 0) {
      fprintf(stderr, "Connect error\n");
      exit(1);
   }
   /* Initialize TLS session 
    */
   gnutls_init(&session, GNUTLS_CLIENT);

   /* allow both SSL3 and TLS1
    */
   gnutls_protocol_set_priority(session, protocol_priority);

   /* allow only ARCFOUR and 3DES ciphers
    * (3DES has the highest priority)
    */
   gnutls_cipher_set_priority(session, cipher_priority);

   /* only allow null compression
    */
   gnutls_compression_set_priority(session, comp_priority);

   /* use GNUTLS_KX_RSA
    */
   gnutls_kx_set_priority(session, kx_priority);

   /* allow the usage of both SHA and MD5
    */
   gnutls_mac_set_priority(session, mac_priority);


   /* put the x509 credentials to the current session
    */
   gnutls_credentials_set(session, GNUTLS_CRD_CERTIFICATE, xcred);


   gnutls_transport_set_ptr( session, sd);
   /* Perform the TLS handshake
    */
   ret = gnutls_handshake( session);

   if (ret < 0) {
      fprintf(stderr, "*** Handshake failed\n");
      gnutls_perror(ret);
      goto end;
   } else {
      printf("- Handshake was completed\n");
   }

   gnutls_record_send( session, MSG, strlen(MSG));

   ret = gnutls_record_recv( session, buffer, MAX_BUF);
   if (ret == 0) {
      printf("- Peer has closed the TLS connection\n");
      goto end;
   } else if (ret < 0) {
      fprintf(stderr, "*** Received corrupted data(%d) - server has terminated the connection abnormally\n",
              ret);
      goto end;
   } else if (ret > 0) {
      printf("- Received %d bytes: ", ret);
      for (ii = 0; ii < ret; ii++) {
         fputc(buffer[ii], stdout);
      }
      fputs("\n", stdout);
   }
   gnutls_bye( session, GNUTLS_SHUT_RDWR);

 end:

   shutdown(sd, SHUT_RDWR);     /* no more receptions */
   close(sd);

   gnutls_deinit(session);

   gnutls_certificate_free_credentials(xcred);

   gnutls_global_deinit();

   return 0;
}

\end{verbatim}


\subsection{Verifying peer's certificate}
\par A TLS connection is not secure just after the handshake has finished.
It must be considered secure, after the peer's identity has been
verified. That is, you usually have to verify not only the peer's 
certificate, but also the hostname in the certificate, expiration dates etc. 
After this step you should treat the connection as being a secure one.

\par
The following function is an example on how to verify a certificate.

\index{Verifying certificate paths}
\label{ex:rfc2818}

\begin{verbatim}

#include <gnutls/gnutls.h>

/* This function will try to verify the peer's certificate, and
 * also check if the hostname matches, and the activation, expiration dates.
 */
void verify_certificate( gnutls_session session, const char* hostname)
{
   int status;

   /* This verification function uses the trusted CAs in the credentials
    * structure. So you must have installed one or more CAs.
    */
   status = gnutls_certificate_verify_peers(session);

   if (status == GNUTLS_E_NO_CERTIFICATE_FOUND) {
      printf("No certificate was sent");
      return;
   }

   if (status & GNUTLS_CERT_INVALID || status & GNUTLS_CERT_NOT_TRUSTED
      || status & GNUTLS_CERT_CORRUPTED || status & GNUTLS_CERT_REVOKED) {
      printf("The certificate is not trusted");
      return;
   }

   if ( gnutls_certificate_expiration_time_peers(session) < time(0)) {
      printf("The certificate has expired\n");
      return;
   }

   if ( gnutls_certificate_activation_time_peers(session) > time(0)) {
      printf("The certificate is not yet activated\n");
      return;
   }

   if ( gnutls_certificate_type_get(session) == GNUTLS_CRT_X509) {
      const gnutls_datum* cert_list;
      int cert_list_size;
      
      cert_list = gnutls_certificate_get_peers( session, &cert_list_size);
      if ( cert_list == NULL) {
         printf("No certificate was found!\n");
         return;
      }
      if ( !gnutls_x509_check_certificates_hostname( &cert_list[0], hostname)) {
         printf("The certificate does not match hostname\n");
         return;
      }
   }
   
   printf("The certificate is trusted.\n");
   return;
}

\end{verbatim}


\subsection{Parsing peer's certificate, and obtaining session information}
The following function reads the peer's certificate,
and prints some information about the certificate and the current session.
\par
This function should be called after a successful
\printfunc{gnutls_handshake}{gnutls\_handshake}

\begin{verbatim}

#include <stdio.h>
#include <stdlib.h>
#include <gnutls/gnutls.h>
#include <gnutls/x509.h>

static void print_x509_certificate_info(gnutls_session);

/* This function will print some details of the
 * given session.
 */
int print_info(gnutls_session session)
{
   const char *tmp;
   gnutls_credentials_type cred;
   gnutls_kx_algorithm kx;

   /* print the key exchange's algorithm name
    */
   kx = gnutls_kx_get(session);
   tmp = gnutls_kx_get_name(kx);
   printf("- Key Exchange: %s\n", tmp);

   /* Check the authentication type used and switch
    * to the appropriate.
    */
   cred = gnutls_auth_get_type(session);
   switch (cred) {
   case GNUTLS_CRD_ANON:       /* anonymous authentication */

      printf("- Anonymous DH using prime of %d bits\n",
             gnutls_dh_get_prime_bits(session));
      break;

   case GNUTLS_CRD_CERTIFICATE:        /* certificate authentication */
      
      /* Check if we have been using ephemeral Diffie Hellman.
       */
      if (kx == GNUTLS_KX_DHE_RSA || kx == GNUTLS_KX_DHE_DSS) {
         printf("\n- Ephemeral DH using prime of %d bits\n",
                gnutls_dh_get_prime_bits(session));
      }

      /* if the certificate list is available, then
       * print some information about it.
       */
      print_x509_certificate_info(session);

   } /* switch */

   /* print the protocol's name (ie TLS 1.0) 
    */
   tmp = gnutls_protocol_get_name(gnutls_protocol_get_version(session));
   printf("- Protocol: %s\n", tmp);

   /* print the certificate type of the peer.
    * ie X.509
    */
   tmp = gnutls_certificate_type_get_name(
      gnutls_certificate_type_get(session));

   printf("- Certificate Type: %s\n", tmp);

   /* print the compression algorithm (if any)
    */
   tmp = gnutls_compression_get_name( gnutls_compression_get(session));
   printf("- Compression: %s\n", tmp);

   /* print the name of the cipher used.
    * ie 3DES.
    */
   tmp = gnutls_cipher_get_name(gnutls_cipher_get(session));
   printf("- Cipher: %s\n", tmp);

   /* Print the MAC algorithms name.
    * ie SHA1
    */
   tmp = gnutls_mac_get_name(gnutls_mac_get(session));
   printf("- MAC: %s\n", tmp);

   return 0;
}

/* This function will print information about this session's peer
 * certificate. 
 */
static void print_x509_certificate_info(gnutls_session session)
{
   char digest[20];
   char serial[40];
   int digest_size, i;
   int serial_size;
   char printable[120];
   int printable_size;
   char *print;
   int algo, bits;
   time_t expiration_time, activation_time;
   const gnutls_datum *cert_list;
   int cert_list_size = 0;
   gnutls_x509_crt cert;

   cert_list = gnutls_certificate_get_peers(session, &cert_list_size);

   if (cert_list_size > 0
       && gnutls_certificate_type_get(session) == GNUTLS_CRT_X509) {

      /* no error checking
       */
      gnutls_x509_crt_init( &cert);

      gnutls_x509_crt_import( cert, &cert_list[0]);

      printf(" - Certificate info:\n");

      expiration_time = gnutls_x509_crt_get_expiration_time( cert);
      activation_time = gnutls_x509_crt_get_activation_time( cert);

      printf(" - Certificate is valid since: %s", ctime(&activation_time));
      printf(" - Certificate expires: %s", ctime(&expiration_time));

      /* Print the fingerprint of the certificate
       */
      digest_size = sizeof(digest);
      if (gnutls_x509_fingerprint
          (GNUTLS_DIG_MD5, &cert_list[0], digest, &digest_size) >= 0) {
         print = printable;
         for (i = 0; i < digest_size; i++) {
            sprintf(print, "%.2x ", (unsigned char) digest[i]);
            print += 3;
         }
         printf(" - Certificate fingerprint: %s\n", printable);
      }

      /* Print the serial number of the certificate.
       */
      serial_size = sizeof(serial);
      if (gnutls_x509_crt_get_serial(cert, serial, &serial_size) >= 0) 
      {
         print = printable;
         for (i = 0; i < serial_size; i++) {
            sprintf(print, "%.2x ", (unsigned char) serial[i]);
            print += 3;
         }
         printf(" - Certificate serial number: %s\n", printable);
      }

      /* Extract some of the public key algorithm's parameters
       */
      algo =
          gnutls_x509_crt_get_pk_algorithm(cert, &bits);

      printf("Certificate public key: ");

      if (algo == GNUTLS_PK_RSA) {
         printf("RSA\n");
         printf(" Modulus: %d bits\n", bits);
      } else if (algo == GNUTLS_PK_DSA) {
         printf("DSA\n");
         printf(" Exponent: %d bits\n", bits);
      } else {
         printf("UNKNOWN\n");
      }

      /* Print the version of the X.509 
       * certificate.
       */
      printf(" - Certificate version: #%d\n",
             gnutls_x509_crt_get_version( cert));

      printable_size = sizeof(printable);
      gnutls_x509_crt_get_dn( cert, printable, &printable_size);
      printf(" - DN: %s\n", printable);

      printable_size = sizeof(printable);
      gnutls_x509_crt_get_issuer_dn( cert, printable, &printable_size);
      printf(" - Certificate Issuer's DN: %s\n", printable);

      gnutls_x509_crt_deinit( cert);

   }
}

\end{verbatim}



\subsection{Client with Resume capability example}
\label{resume-example}
This is a modification of the simple client above. Here we added support 
for session resumption.
\begin{verbatim}

#include <stdio.h>
#include <stdlib.h>
#include <gnutls/gnutls.h>

/* Those functions are defined in other examples.
 */
extern void check_alert(gnutls_session session, int ret);
extern int tcp_connect( void);
extern void tcp_close( int sd);

#define MAX_BUF 1024
#define CRLFILE "crl.pem"
#define CAFILE "ca.pem"
#define SA struct sockaddr
#define MSG "GET / HTTP/1.0\r\n\r\n"

int main()
{
   int ret;
   int sd, ii, alert;
   gnutls_session session;
   char buffer[MAX_BUF + 1];
   gnutls_certificate_credentials xcred;

   /* variables used in session resuming 
    */
   int t;
   char *session_data;
   size_t session_data_size;

   gnutls_global_init();

   /* X509 stuff */
   gnutls_certificate_allocate_credentials(&xcred);

   gnutls_certificate_set_x509_trust_file(xcred, CAFILE, GNUTLS_X509_FMT_PEM);

   for (t = 0; t < 2; t++) {    /* connect 2 times to the server */

      sd = tcp_connect();

      gnutls_init(&session, GNUTLS_CLIENT);

      gnutls_set_default_priority(session);

      gnutls_credentials_set(session, GNUTLS_CRD_CERTIFICATE, xcred);

      if (t > 0) { /* if this is not the first time we connect */
         gnutls_session_set_data(session, session_data, session_data_size);
         free(session_data);
      }
      
      gnutls_transport_set_ptr( session, (gnutls_transport_ptr)sd);

      /* Perform the TLS handshake
       */
      ret = gnutls_handshake( session);

      if (ret < 0) {
         fprintf(stderr, "*** Handshake failed\n");
         gnutls_perror(ret);
         goto end;
      } else {
         printf("- Handshake was completed\n");
      }

      if (t == 0) { /* the first time we connect */
         /* get the session data size */
         gnutls_session_get_data(session, NULL, &session_data_size);
         session_data = malloc(session_data_size);

         /* put session data to the session variable */
         gnutls_session_get_data(session, session_data, &session_data_size);

      } else { /* the second time we connect */

         /* check if we actually resumed the previous session */
         if (gnutls_session_is_resumed( session) != 0) {
            printf("- Previous session was resumed\n");
         } else {
            fprintf(stderr, "*** Previous session was NOT resumed\n");
         }
      }

      /* This function was defined in a previous example
       */
      /* print_info(session); */

      gnutls_record_send( session, MSG, strlen(MSG));

      ret = gnutls_record_recv( session, buffer, MAX_BUF);
      if (ret == 0) {
         printf("- Peer has closed the TLS connection\n");
         goto end;
      } else if (ret < 0) {
         fprintf(stderr, "*** Error: %s\n", gnutls_strerror(ret));
         goto end;
      }

      printf("- Received %d bytes: ", ret);
      for (ii = 0; ii < ret; ii++) {
         fputc(buffer[ii], stdout);
      }
      fputs("\n", stdout);

      gnutls_bye( session, GNUTLS_SHUT_RDWR);

    end:

      tcp_close(sd);

      gnutls_deinit(session);

   }  /* for() */

   gnutls_certificate_free_credentials(xcred);

   gnutls_global_deinit();

   return 0;
}

\end{verbatim}


\subsection{Client with Resume capability example II}
\label{resume-example2}
This is also a client with resume capability, but also demonstrates
the use of session IDs.
\begin{verbatim}

#include <stdio.h>
#include <stdlib.h>
#include <sys/types.h>
#include <sys/socket.h>
#include <netinet/in.h>
#include <arpa/inet.h>
#include <unistd.h>
#include <gnutls/gnutls.h>

/* A TLS 1.0 client with session resuming capability.
 */

#define MAX_BUF 1024
#define CRLFILE "crl.pem"
#define CAFILE "ca.pem"
#define SA struct sockaddr
#define MSG "GET / HTTP/1.0\r\n\r\n"

const int protocol_priority[] = { GNUTLS_TLS1, GNUTLS_SSL3, 0 };
const int kx_priority[] = { GNUTLS_KX_RSA, GNUTLS_KX_DHE_RSA, 0 };
const int cipher_priority[] = { GNUTLS_CIPHER_3DES_CBC, 
                                GNUTLS_CIPHER_ARCFOUR_128, 0};
const int comp_priority[] = { GNUTLS_COMP_NULL, 0 };
const int mac_priority[] = { GNUTLS_MAC_SHA, GNUTLS_MAC_MD5, 0 };

int main()
{
   const char *PORT = "443";
   const char *SERVER = "127.0.0.1";
   int err, ret;
   int sd, ii, alert;
   struct sockaddr_in sa;
   gnutls_session session;
   char buffer[MAX_BUF + 1];
   gnutls_certificate_credentials xcred;
   /* variables used in session resuming */
   int t;
   char *session_data;
   char *session_id;
   int session_data_size;
   int session_id_size;
   char *tmp_session_id;
   int tmp_session_id_size;

   gnutls_global_init();

   /* X509 stuff 
    */
   gnutls_certificate_allocate_credentials(&xcred);

   gnutls_certificate_set_x509_trust_file(xcred, CAFILE, GNUTLS_X509_FMT_PEM);

   for (t = 0; t < 2; t++) {    /* connect 2 times to the server */

      sd = socket(AF_INET, SOCK_STREAM, 0);
      memset(&sa, '\0', sizeof(sa));
      sa.sin_family = AF_INET;
      sa.sin_port = htons(atoi(PORT));
      inet_pton(AF_INET, SERVER, &sa.sin_addr);

      err = connect(sd, (SA *) & sa, sizeof(sa));
      if (err < 0) {
         fprintf(stderr, "Connect error");
         exit(1);
      }
      gnutls_init(&session, GNUTLS_CLIENT);

      gnutls_protocol_set_priority(session, protocol_priority);
      gnutls_cipher_set_priority(session, cipher_priority);
      gnutls_compression_set_priority(session, comp_priority);
      gnutls_kx_set_priority(session, kx_priority);
      gnutls_mac_set_priority(session, mac_priority);

      gnutls_credentials_set(session, GNUTLS_CRD_CERTIFICATE, xcred);

      if (t > 0) { /* if this is not the first time we connect */
         gnutls_session_set_data(session, session_data, session_data_size);
         free(session_data);
      }
      
      gnutls_transport_set_ptr( session, (gnutls_transport_ptr)sd);

      /* Perform the TLS handshake
       */
      ret = gnutls_handshake( session);

      if (ret < 0) {
         fprintf(stderr, "*** Handshake failed\n");
         gnutls_perror(ret);
         goto end;
      } else {
         printf("- Handshake was completed\n");
      }

      if (t == 0) { /* the first time we connect */
         /* get the session data size */
         gnutls_session_get_data(session, NULL, &session_data_size);
         session_data = malloc(session_data_size);

         /* put session data to the session variable */
         gnutls_session_get_data(session, session_data, &session_data_size);

         /* keep the current session ID. This is only needed
          * in order to check if the server actually resumed this
          * connection.
          */
         gnutls_session_get_id(session, NULL, &session_id_size);
         session_id = malloc(session_id_size);
         gnutls_session_get_id(session, session_id, &session_id_size);

      } else { /* the second time we connect */

         /* check if we actually resumed the previous session */
         gnutls_session_get_id(session, NULL, &tmp_session_id_size);
         tmp_session_id = malloc(tmp_session_id_size);
         gnutls_session_get_id(session, tmp_session_id, &tmp_session_id_size);

         if (memcmp(tmp_session_id, session_id, session_id_size) == 0) {
            printf("- Previous session was resumed\n");
         } else {
            fprintf(stderr, "*** Previous session was NOT resumed\n");
         }
         free(tmp_session_id);
         free(session_id);
      }

      /* This function was defined in a previous example
       */
      /* print_info(session); */

      gnutls_record_send( session, MSG, strlen(MSG));

      ret = gnutls_record_recv( session, buffer, MAX_BUF);
      if (ret == 0) {
         printf("- Peer has closed the TLS connection\n");
         goto end;
      } else if (ret < 0) {
         fprintf(stderr, "*** Error: %s\n", gnutls_strerror(ret));
         goto end;
      } else if (ret > 0) {
         printf("- Received %d bytes: ", ret);
         for (ii = 0; ii < ret; ii++) {
            fputc(buffer[ii], stdout);
         }
         fputs("\n", stdout);
      }
      gnutls_bye( session, GNUTLS_SHUT_RDWR);

    end:

      shutdown(sd, SHUT_RDWR);  /* no more receptions */
      close(sd);

      gnutls_deinit(session);

   }  /* for() */

   gnutls_certificate_free_credentials(xcred);

   gnutls_global_deinit();

   return 0;
}

\end{verbatim}


\subsection{Simple client example with SRP authentication}
Although {\bf SRP} is not part of the \tls{} standard, \gnutls{} implements
{\it David Taylor's} proposal\footnote{This is work in progress.}  for using the SRP algorithm
within the \tls{} handshake protocol. The following client
is a very simple SRP-TLS client which connects to a server 
and authenticates using {\it username} and {\it password}.

\begin{verbatim}

#include <stdio.h>
#include <stdlib.h>
#include <gnutls/gnutls.h>
#include <gnutls/extra.h>

/* Those functions are defined in other examples.
 */
extern void check_alert(gnutls_session session, int ret);
extern int tcp_connect( void);
void tcp_close( int sd);

#define MAX_BUF 1024
#define USERNAME "user"
#define PASSWORD "pass"
#define SA struct sockaddr
#define MSG "GET / HTTP/1.0\r\n\r\n"

const int kx_priority[] = { GNUTLS_KX_SRP, 0 };

int main()
{
   int ret;
   int sd, ii;
   gnutls_session session;
   char buffer[MAX_BUF + 1];
   gnutls_srp_client_credentials xcred;

   if (gnutls_global_init() < 0) {
      fprintf(stderr, "global state initialization error\n");
      exit(1);
   }

   /* now enable the gnutls-extra library which contains the
    * SRP stuff. */
   if (gnutls_global_init_extra() < 0) {
      fprintf(stderr, "global state initialization error\n");
      exit(1);
   }

   if (gnutls_srp_allocate_client_credentials(&xcred) < 0) {
      fprintf(stderr, "memory error\n");
      exit(1);
   }
   gnutls_srp_set_client_credentials(xcred, USERNAME, PASSWORD);

   /* connects to server 
    */
   sd = tcp_connect();

   /* Initialize TLS session 
    */
   gnutls_init(&session, GNUTLS_CLIENT);


   /* Set the priorities.
    */
   gnutls_set_default_priority(session);
 
   /* use GNUTLS_KX_SRP
    */
   gnutls_kx_set_priority(session, kx_priority);
 

   /* put the SRP credentials to the current session
    */
   gnutls_credentials_set(session, GNUTLS_CRD_SRP, xcred);

   gnutls_transport_set_ptr( session, (gnutls_transport_ptr)sd);

   /* Perform the TLS handshake
    */
   ret = gnutls_handshake( session);

   if (ret < 0) {
      fprintf(stderr, "*** Handshake failed\n");
      gnutls_perror(ret);
      goto end;
   } else {
      printf("- Handshake was completed\n");
   }

   gnutls_record_send( session, MSG, strlen(MSG));

   ret = gnutls_record_recv( session, buffer, MAX_BUF);
   if (gnutls_error_is_fatal(ret) == 1 || ret == 0) {
      if (ret == 0) {
         printf("- Peer has closed the GNUTLS connection\n");
         goto end;
      } else {
         fprintf(stderr, "*** Error: %s\n", gnutls_strerror(ret));
         goto end;
      }
   } else
      check_alert( session, ret);

   if (ret > 0) {
      printf("- Received %d bytes: ", ret);
      for (ii = 0; ii < ret; ii++) {
         fputc(buffer[ii], stdout);
      }
      fputs("\n", stdout);
   }
   gnutls_bye( session, 0);

 end:

   tcp_close( sd);

   gnutls_deinit(session);

   gnutls_srp_free_client_credentials(xcred);

   gnutls_global_deinit();

   return 0;
}

\end{verbatim}


\section{Server examples}
This section contains examples of \tls{} and \ssl{} servers, using \gnutls{}.

\subsection{Echo Server with X.509 authentication}
This example is a server which supports {\bf X.509} authentication.
\begin{verbatim}

#include <stdio.h>
#include <stdlib.h>
#include <errno.h>
#include <sys/types.h>
#include <sys/socket.h>
#include <netinet/in.h>
#include <arpa/inet.h>
#include <string.h>
#include <unistd.h>
#include <gnutls/gnutls.h>

#define KEYFILE "key.pem"
#define CERTFILE "cert.pem"
#define CAFILE "ca.pem"
#define CRLFILE "crl.pem"

/* This is a sample TLS 1.0 echo server.
 */


#define SA struct sockaddr
#define SOCKET_ERR(err,s) if(err==-1) {perror(s);return(1);}
#define MAX_BUF 1024
#define PORT 5556               /* listen to 5556 port */
#define DH_BITS 1024

/* These are global */
gnutls_certificate_credentials x509_cred;

gnutls_session initialize_tls_session()
{
   gnutls_session session;

   gnutls_init(&session, GNUTLS_SERVER);

   /* avoid calling all the priority functions, since the defaults
    * are adequate.
    */
   gnutls_set_default_priority( session);   

   gnutls_credentials_set(session, GNUTLS_CRD_CERTIFICATE, x509_cred);

   /* request client certificate if any.
    */
   gnutls_certificate_server_set_request( session, GNUTLS_CERT_REQUEST);

   gnutls_dh_set_prime_bits( session, DH_BITS);

   /* some broken clients may require this in order to connect. 
    * This may weaken security though.
    */
   /* gnutls_handshake_set_rsa_pms_check( session, 1); */

   
   return session;
}

gnutls_dh_params dh_params;

static int generate_dh_params(void) {

   /* Generate Diffie Hellman parameters - for use with DHE
    * kx algorithms. These should be discarded and regenerated
    * once a day, once a week or once a month. Depends on the
    * security requirements.
    */
   gnutls_dh_params_init( &dh_params);
   gnutls_dh_params_generate2( dh_params, DH_BITS);
   
   return 0;
}

int main()
{
   int err, listen_sd, i;
   int sd, ret;
   struct sockaddr_in sa_serv;
   struct sockaddr_in sa_cli;
   int client_len;
   char topbuf[512];
   gnutls_session session;
   char buffer[MAX_BUF + 1];
   int optval = 1;
   char name[256];

   strcpy(name, "Echo Server");

   /* this must be called once in the program
    */
   gnutls_global_init();

   gnutls_certificate_allocate_credentials(&x509_cred);
   gnutls_certificate_set_x509_trust_file(x509_cred, CAFILE, 
      GNUTLS_X509_FMT_PEM);

   gnutls_certificate_set_x509_crl_file(x509_cred, CRLFILE, 
      GNUTLS_X509_FMT_PEM);

   gnutls_certificate_set_x509_key_file(x509_cred, CERTFILE, KEYFILE, 
      GNUTLS_X509_FMT_PEM);

   generate_dh_params();
   
   gnutls_certificate_set_dh_params( x509_cred, dh_params);

   /* Socket operations
    */
   listen_sd = socket(AF_INET, SOCK_STREAM, 0);
   SOCKET_ERR(listen_sd, "socket");

   memset(&sa_serv, '\0', sizeof(sa_serv));
   sa_serv.sin_family = AF_INET;
   sa_serv.sin_addr.s_addr = INADDR_ANY;
   sa_serv.sin_port = htons(PORT);  /* Server Port number */

   setsockopt(listen_sd, SOL_SOCKET, SO_REUSEADDR, &optval, sizeof(int));

   err = bind(listen_sd, (SA *) & sa_serv, sizeof(sa_serv));
   SOCKET_ERR(err, "bind");
   err = listen(listen_sd, 1024);
   SOCKET_ERR(err, "listen");

   printf("%s ready. Listening to port '%d'.\n\n", name, PORT);

   client_len = sizeof(sa_cli);
   for (;;) {
      session = initialize_tls_session();

      sd = accept(listen_sd, (SA *) & sa_cli, &client_len);

      printf("- connection from %s, port %d\n",
             inet_ntop(AF_INET, &sa_cli.sin_addr, topbuf,
                       sizeof(topbuf)), ntohs(sa_cli.sin_port));

      gnutls_transport_set_ptr( session, sd);
      ret = gnutls_handshake( session);
      if (ret < 0) {
         close(sd);
         gnutls_deinit(session);
         fprintf(stderr, "*** Handshake has failed (%s)\n\n",
                 gnutls_strerror(ret));
         continue;
      }
      printf("- Handshake was completed\n");

      /* see the Getting peer's information example */
      /* print_info(session); */

      i = 0;
      for (;;) {
         bzero(buffer, MAX_BUF + 1);
         ret = gnutls_record_recv( session, buffer, MAX_BUF);

         if (ret == 0) {
            printf
                ("\n- Peer has closed the GNUTLS connection\n");
            break;
         } else if (ret < 0) {
            fprintf(stderr,
                    "\n*** Received corrupted data(%d). Closing the connection.\n\n",
                    ret);
            break;
         } else if (ret > 0) {
            /* echo data back to the client
             */
            gnutls_record_send( session, buffer,
                         strlen(buffer));
         }
      }
      printf("\n");
      gnutls_bye( session, GNUTLS_SHUT_WR); /* do not wait for
                                 * the peer to close the connection.
                                 */

      close(sd);
      gnutls_deinit(session);

   }
   close(listen_sd);

   gnutls_certificate_free_credentials(x509_cred);

   gnutls_global_deinit();

   return 0;

}

\end{verbatim}


\subsection{Echo Server with X.509 authentication II}
The following example is a server which supports {\bf X.509} authentication.
This server also supports export-grade cipher suites and session resuming.
\begin{verbatim}

#include <stdio.h>
#include <stdlib.h>
#include <errno.h>
#include <sys/types.h>
#include <sys/socket.h>
#include <netinet/in.h>
#include <arpa/inet.h>
#include <string.h>
#include <unistd.h>
#include <gnutls/gnutls.h>

#define KEYFILE "key.pem"
#define CERTFILE "cert.pem"
#define CAFILE "ca.pem"
#define CRLFILE NULL

/* This is a sample TLS 1.0 echo server.
 * Export-grade ciphersuites and session resuming are supported.
 */

#define SA struct sockaddr
#define SOCKET_ERR(err,s) if(err==-1) {perror(s);return(1);}
#define MAX_BUF 1024
#define PORT 5556               /* listen to 5556 port */
#define DH_BITS 1024

/* These are global */
gnutls_certificate_server_credentials x509_cred;

static void wrap_db_init(void);
static void wrap_db_deinit(void);
static int wrap_db_store(void *dbf, gnutls_datum key, gnutls_datum data);
static gnutls_datum wrap_db_fetch(void *dbf, gnutls_datum key);
static int wrap_db_delete(void *dbf, gnutls_datum key);

#define TLS_SESSION_CACHE 50

gnutls_session initialize_tls_session()
{
   gnutls_session session;

   gnutls_init(&session, GNUTLS_SERVER);

   /* Use the default priorities, plus, export cipher suites.
    */
   gnutls_set_default_export_priority(session);

   gnutls_credentials_set(session, GNUTLS_CRD_CERTIFICATE, x509_cred);

   /* request client certificate if any.
    */
   gnutls_certificate_server_set_request(session, GNUTLS_CERT_REQUEST);

   gnutls_dh_set_prime_bits(session, DH_BITS);

   /* some broken clients may require this in order to connect. 
    * This will weaken security though.
    */
   /* gnutls_handshake_set_rsa_pms_check( session, 1); */

   if (TLS_SESSION_CACHE != 0) {
      gnutls_db_set_retrieve_function(session, wrap_db_fetch);
      gnutls_db_set_remove_function(session, wrap_db_delete);
      gnutls_db_set_store_function(session, wrap_db_store);
      gnutls_db_set_ptr(session, NULL);
   }

   return session;
}

gnutls_dh_params dh_params;
/* Export-grade cipher suites require temporary RSA
 * keys.
 */
gnutls_rsa_params rsa_params;

static int generate_dh_params(void)
{
   gnutls_datum prime, generator;

   /* Generate Diffie Hellman parameters - for use with DHE
    * kx algorithms. These should be discarded and regenerated
    * once a day, once a week or once a month. Depends on the
    * security requirements.
    */
   gnutls_dh_params_init(&dh_params);
   gnutls_dh_params_generate(&prime, &generator, DH_BITS);
   gnutls_dh_params_set(dh_params, prime, generator, DH_BITS);

   free(prime.data);
   free(generator.data);
   
   return 0;
}

static int generate_rsa_params(void)
{
   gnutls_datum m, e, d, p, q, u;

   gnutls_rsa_params_init(&rsa_params);

   /* Generate RSA parameters - for use with RSA-export
    * cipher suites. These should be discarded and regenerated
    * once a day, once every 500 transactions etc. Depends on the
    * security requirements.
    */

   gnutls_rsa_params_generate(&m, &e, &d, &p, &q, &u, 512);
   gnutls_rsa_params_set(rsa_params, m, e, d, p, q, u, 512);

   free(m.data);
   free(e.data);
   free(d.data);
   free(p.data);
   free(q.data);
   free(u.data);

   return 0;
}

int main()
{
   int err, listen_sd, i;
   int sd, ret;
   struct sockaddr_in sa_serv;
   struct sockaddr_in sa_cli;
   int client_len;
   char topbuf[512];
   gnutls_session session;
   char buffer[MAX_BUF + 1];
   int optval = 1;
   char name[256];

   strcpy(name, "Echo Server");

   /* this must be called once in the program
    */
   gnutls_global_init();

   gnutls_certificate_allocate_credentials(&x509_cred);

   gnutls_certificate_set_x509_trust_file(x509_cred, CAFILE,
                                          GNUTLS_X509_FMT_PEM);

   gnutls_certificate_set_x509_key_file(x509_cred, CERTFILE, KEYFILE,
                                        GNUTLS_X509_FMT_PEM);

   generate_dh_params();
   generate_rsa_params();

   if (TLS_SESSION_CACHE != 0) {
      wrap_db_init();
   }

   gnutls_certificate_set_dh_params(x509_cred, dh_params);
   gnutls_certificate_set_rsa_params(x509_cred, rsa_params);

   /* Socket operations
    */
   listen_sd = socket(AF_INET, SOCK_STREAM, 0);
   SOCKET_ERR(listen_sd, "socket");

   memset(&sa_serv, '\0', sizeof(sa_serv));
   sa_serv.sin_family = AF_INET;
   sa_serv.sin_addr.s_addr = INADDR_ANY;
   sa_serv.sin_port = htons(PORT);      /* Server Port number */

   setsockopt(listen_sd, SOL_SOCKET, SO_REUSEADDR, &optval, sizeof(int));

   err = bind(listen_sd, (SA *) & sa_serv, sizeof(sa_serv));
   SOCKET_ERR(err, "bind");
   err = listen(listen_sd, 1024);
   SOCKET_ERR(err, "listen");

   printf("%s ready. Listening to port '%d'.\n\n", name, PORT);

   client_len = sizeof(sa_cli);
   for (;;) {
      session = initialize_tls_session();

      sd = accept(listen_sd, (SA *) & sa_cli, &client_len);

      printf("- connection from %s, port %d\n",
             inet_ntop(AF_INET, &sa_cli.sin_addr, topbuf,
                       sizeof(topbuf)), ntohs(sa_cli.sin_port));

      gnutls_transport_set_ptr(session, sd);
      ret = gnutls_handshake(session);
      if (ret < 0) {
         close(sd);
         gnutls_deinit(session);
         fprintf(stderr, "*** Handshake has failed (%s)\n\n",
                 gnutls_strerror(ret));
         continue;
      }
      printf("- Handshake was completed\n");

      /* print_info(session); */

      i = 0;
      for (;;) {
         bzero(buffer, MAX_BUF + 1);
         ret = gnutls_record_recv(session, buffer, MAX_BUF);

         if (ret == 0) {
            printf("\n- Peer has closed the TLS connection\n");
            break;
         } else if (ret < 0) {
            fprintf(stderr,
                    "\n*** Received corrupted data(%d). Closing the connection.\n\n",
                    ret);
            break;
         } else if (ret > 0) {
            /* echo data back to the client
             */
            gnutls_record_send(session, buffer, strlen(buffer));
         }
      }
      printf("\n");
      gnutls_bye(session, GNUTLS_SHUT_WR);      /* do not wait for
                                                   * the peer to close the connection.
                                                 */

      close(sd);
      gnutls_deinit(session);

   }
   close(listen_sd);

   gnutls_certificate_free_credentials(x509_cred);

   gnutls_global_deinit();

   return 0;

}


/* Functions and other stuff needed for session resuming.
 * This is done using a very simple list which holds session ids
 * and session data.
 */

#define MAX_SESSION_ID_SIZE 32
#define MAX_SESSION_DATA_SIZE 512

typedef struct {
   char session_id[MAX_SESSION_ID_SIZE];
   int session_id_size;

   char session_data[MAX_SESSION_DATA_SIZE];
   int session_data_size;
} CACHE;

static CACHE *cache_db;
static int cache_db_ptr = 0;

static void wrap_db_init(void)
{

   /* allocate cache_db */
   cache_db = calloc(1, TLS_SESSION_CACHE * sizeof(CACHE));
}

static void wrap_db_deinit(void)
{
   return;
}

static int wrap_db_store(void *dbf, gnutls_datum key, gnutls_datum data)
{

   if (cache_db == NULL)
      return -1;

   if (key.size > MAX_SESSION_ID_SIZE)
      return -1;
   if (data.size > MAX_SESSION_DATA_SIZE)
      return -1;

   memcpy(cache_db[cache_db_ptr].session_id, key.data, key.size);
   cache_db[cache_db_ptr].session_id_size = key.size;

   memcpy(cache_db[cache_db_ptr].session_data, data.data, data.size);
   cache_db[cache_db_ptr].session_data_size = data.size;

   cache_db_ptr++;
   cache_db_ptr %= TLS_SESSION_CACHE;

   return 0;
}

static gnutls_datum wrap_db_fetch(void *dbf, gnutls_datum key)
{
   gnutls_datum res = { NULL, 0 };
   int i;

   if (cache_db == NULL)
      return res;

   for (i = 0; i < TLS_SESSION_CACHE; i++) {
      if (key.size == cache_db[i].session_id_size &&
          memcmp(key.data, cache_db[i].session_id, key.size) == 0) {


         res.size = cache_db[i].session_data_size;

         res.data = gnutls_malloc(res.size);
         if (res.data == NULL)
            return res;

         memcpy(res.data, cache_db[i].session_data, res.size);

         return res;
      }
   }
   return res;
}

static int wrap_db_delete(void *dbf, gnutls_datum key)
{
   int i;

   if (cache_db == NULL)
      return -1;

   for (i = 0; i < TLS_SESSION_CACHE; i++) {
      if (key.size == cache_db[i].session_id_size &&
          memcmp(key.data, cache_db[i].session_id, key.size) == 0) {

         cache_db[i].session_id_size = 0;
         cache_db[i].session_data_size = 0;

         return 0;
      }
   }

   return -1;

}

\end{verbatim}


\subsection{Echo Server with SRP authentication}
This is a server which supports {\bf SRP} authentication.
\begin{verbatim}

#include <stdio.h>
#include <stdlib.h>
#include <errno.h>
#include <sys/types.h>
#include <sys/socket.h>
#include <netinet/in.h>
#include <arpa/inet.h>
#include <string.h>
#include <unistd.h>
#include <gnutls/gnutls.h>
#include <gnutls/extra.h>

#define SRP_PASSWD "tpasswd"
#define SRP_PASSWD_CONF "tpasswd.conf"

/* This is a sample TLS-SRP echo server.
 */

#define SA struct sockaddr
#define SOCKET_ERR(err,s) if(err==-1) {perror(s);return(1);}
#define MAX_BUF 1024
#define PORT 5556               /* listen to 5556 port */

/* These are global */
gnutls_srp_server_credentials srp_cred;

gnutls_session initialize_tls_session()
{
   gnutls_session session;
   const int protocol_priority[] = { GNUTLS_TLS1, GNUTLS_SSL3, 0 };
   const int kx_priority[] = { GNUTLS_KX_SRP, 0 };
   const int cipher_priority[] = { GNUTLS_CIPHER_RIJNDAEL_CBC, GNUTLS_CIPHER_3DES_CBC, 0};
   const int comp_priority[] = { GNUTLS_COMP_NULL, 0 };
   const int mac_priority[] = { GNUTLS_MAC_SHA, GNUTLS_MAC_MD5, 0 };

   gnutls_init(&session, GNUTLS_SERVER);

   gnutls_protocol_set_priority(session, protocol_priority);
   gnutls_cipher_set_priority(session, cipher_priority);
   gnutls_compression_set_priority(session, comp_priority);
   gnutls_kx_set_priority(session, kx_priority);
   gnutls_mac_set_priority(session, mac_priority);

   gnutls_credentials_set(session, GNUTLS_CRD_SRP, srp_cred);

   /* request client certificate if any.
    */
   gnutls_certificate_server_set_request( session, GNUTLS_CERT_IGNORE);

   return session;
}

int main()
{
   int err, listen_sd, i;
   int sd, ret;
   struct sockaddr_in sa_serv;
   struct sockaddr_in sa_cli;
   int client_len;
   char topbuf[512];
   gnutls_session session;
   char buffer[MAX_BUF + 1];
   int optval = 1;
   char name[256];

   strcpy(name, "Echo Server");

   /* these must be called once in the program
    */
   gnutls_global_init();
   gnutls_global_init_extra(); /* for SRP */

   /* SRP_PASSWD a password file (created with the included srpcrypt utility) 
    */
   gnutls_srp_allocate_server_credentials(&srp_cred);
   gnutls_srp_set_server_credentials_file(srp_cred, SRP_PASSWD, SRP_PASSWD_CONF);


   /* TCP socket operations
    */
   listen_sd = socket(AF_INET, SOCK_STREAM, 0);
   SOCKET_ERR(listen_sd, "socket");

   memset(&sa_serv, '\0', sizeof(sa_serv));
   sa_serv.sin_family = AF_INET;
   sa_serv.sin_addr.s_addr = INADDR_ANY;
   sa_serv.sin_port = htons(PORT);  /* Server Port number */

   setsockopt(listen_sd, SOL_SOCKET, SO_REUSEADDR, &optval, sizeof(int));

   err = bind(listen_sd, (SA *) & sa_serv, sizeof(sa_serv));
   SOCKET_ERR(err, "bind");
   err = listen(listen_sd, 1024);
   SOCKET_ERR(err, "listen");

   printf("%s ready. Listening to port '%d'.\n\n", name, PORT);

   client_len = sizeof(sa_cli);
   for (;;) {
      session = initialize_tls_session();

      sd = accept(listen_sd, (SA *) & sa_cli, &client_len);

      printf("- connection from %s, port %d\n",
             inet_ntop(AF_INET, &sa_cli.sin_addr, topbuf,
                       sizeof(topbuf)), ntohs(sa_cli.sin_port));

      gnutls_transport_set_ptr( session, sd);
      ret = gnutls_handshake( session);
      if (ret < 0) {
         close(sd);
         gnutls_deinit(session);
         fprintf(stderr, "*** Handshake has failed (%s)\n\n",
                 gnutls_strerror(ret));
         continue;
      }
      printf("- Handshake was completed\n");

      /* print_info(session); */

      i = 0;
      for (;;) {
         bzero(buffer, MAX_BUF + 1);
         ret = gnutls_record_recv( session, buffer, MAX_BUF);

         if (ret == 0) {
            printf
                ("\n- Peer has closed the GNUTLS connection\n");
            break;
         } else if (ret > 0) {
            fprintf(stderr,
                    "\n*** Received corrupted data(%d). Closing the connection.\n\n",
                    ret);
            break;
         } else if (ret > 0) {
            /* echo data back to the client
             */
            gnutls_record_send( session, buffer,
                         strlen(buffer));
         }
      }
      printf("\n");
      gnutls_bye( session, GNUTLS_SHUT_WR); /* do not wait for
                                 * the peer to close the connection.
                                 */

      close(sd);
      gnutls_deinit(session);

   }
   close(listen_sd);

   gnutls_srp_free_server_credentials(srp_cred);

   gnutls_global_deinit();

   return 0;

}

\end{verbatim}


\subsection{Checking for an alert}
This is a function that checks if an alert has been received
in the current session.
\begin{verbatim}

#include <stdio.h>
#include <stdlib.h>
#include <gnutls/gnutls.h>

/* This function will check whether the given return code from
 * a gnutls function (recv/send), is an alert, and will print
 * that alert.
 */
void check_alert(gnutls_session session, int ret)
{
   int last_alert;

   if (ret == GNUTLS_E_WARNING_ALERT_RECEIVED
       || ret == GNUTLS_E_FATAL_ALERT_RECEIVED) {
      last_alert = gnutls_alert_get(session);

      /* The check for renegotiation is only useful if we are 
       * a server, and we had requested a rehandshake.
       */
      if (last_alert == GNUTLS_A_NO_RENEGOTIATION &&
          ret == GNUTLS_E_WARNING_ALERT_RECEIVED)
         printf("* Received NO_RENEGOTIATION alert. "
                "Client Does not support renegotiation.\n");
      else
         printf("* Received alert '%d': %s.\n", last_alert,
                gnutls_alert_get_name(last_alert));
   }
}

\end{verbatim}


\section{Compatibility with the OpenSSL\index{OpenSSL} library}

To ease \gnutls{}' integration with existing applications, a compatibility 
layer with the widely used OpenSSL library is included in the \emph{gnutls-openssl}
library. This compatibility layer is not complete and it is not 
intended to completely reimplement the OpenSSL API with \gnutls{}.
It only provides source-level compatibility. There is currently no
attempt to make it binary-compatible with OpenSSL.

Prototypes for the compatibility functions are found in the 
``gnutls/openssl.h'' header file.

Current limitations imposed by the compatibility layer include:

\begin{itemize}

\item Error handling is not thread safe.

\end{itemize}


