\begin{verbatim}

#define PRINTX(x,y) if (y[0]!=0) printf(" -   %s %s\n", x, y)
#define PRINT_DN(X) PRINTX( "CN:", X.common_name); \
        PRINTX( "OU:", X.organizational_unit_name); \
        PRINTX( "O:", X.organization); \
        PRINTX( "L:", X.locality_name); \
        PRINTX( "S:", X.state_or_province_name); \
        PRINTX( "C:", X.country); \
        PRINTX( "E:", X.email)

/* This function will print some details of the
 * given state.
 */
int print_info(GNUTLS_STATE state)
{
   const char *tmp;
   GNUTLS_CredType cred;
   gnutls_dn dn;
   const gnutls_datum *cert_list;
   GNUTLS_CertificateStatus status;
   int cert_list_size = 0;
   GNUTLS_KXAlgorithm kx;


   /* print the key exchange's algorithm name
    */
   kx = gnutls_kx_get_algo(state);
   tmp = gnutls_kx_get_name(kx);
   printf("- Key Exchange: %s\n", tmp);

   cred = gnutls_auth_get_type(state);
   switch (cred) {
   case GNUTLS_ANON:
      printf("- Anonymous DH using prime of %d bits\n",
             gnutls_dh_get_bits(state));
      break;
   case GNUTLS_X509PKI:
      /* in case of X509 PKI
       */
      cert_list = gnutls_x509pki_client_get_peer_certificate_list(state, &cert_list_size);
      status = gnutls_x509pki_client_get_peer_certificate_status(state);

      switch (status) {
      case GNUTLS_CERT_NOT_TRUSTED:
         printf("- Peer's X509 Certificate was NOT verified\n");
         break;
      case GNUTLS_CERT_EXPIRED:
         printf("- Peer's X509 Certificate was verified but is expired\n");
         break;
      case GNUTLS_CERT_TRUSTED:
         printf("- Peer's X509 Certificate was verified\n");
         break;
      case GNUTLS_CERT_NONE:
         printf("- Peer did not send any X509 Certificate.\n");
         break;
      case GNUTLS_CERT_INVALID:
         printf("- Peer's X509 Certificate was invalid\n");
         break;
      }

      /* Check if we have been using ephemeral Diffie Hellman.
       */
      if (kx == GNUTLS_KX_DHE_RSA || kx == GNUTLS_KX_DHE_DSS) {
         printf("\n- Ephemeral DH using prime of %d bits\n",
                gnutls_dh_get_bits(state));
      }

      /* if the certificate list is available, then
       * print some information about it.
       */
      if (cert_list_size > 0) {
         char digest[20];
         char serial[40];
         int digest_size = sizeof(digest), i;
         int serial_size = sizeof(serial);
         char printable[120];
         char *print;

         printf(" - Certificate info:\n");

         /* Print the fingerprint of the certificate
          */
         if (gnutls_fingerprint(GNUTLS_DIG_MD5, &cert_list[0], digest, &digest_size) >= 0) {
            print = printable;
            for (i = 0; i < digest_size; i++) {
               sprintf(print, "%.2x ", (unsigned char) digest[i]);
               print += 3;
            }
            printf(" - Certificate fingerprint: %s\n", printable);
         }

         /* Print the serial number of the certificate.
          */
         if (gnutls_x509pki_extract_certificate_serial(&cert_list[0], serial, &serial_size) >= 0) {
            print = printable;
            for (i = 0; i < serial_size; i++) {
               sprintf(print, "%.2x ", (unsigned char) serial[i]);
               print += 3;
            }
            printf(" - Certificate serial number: %s\n", printable);
         }

         /* Print the version of the X.509 
          * certificate.
          */
         printf(" - Certificate version: #%d\n", gnutls_x509pki_extract_certificate_version(&cert_list[0]));

         gnutls_x509pki_extract_certificate_dn(&cert_list[0], &dn);
         PRINT_DN(dn);

         gnutls_x509pki_extract_certificate_issuer_dn(&cert_list[0], &dn);
         printf(" - Certificate Issuer's info:\n");
         PRINT_DN(dn);
      }
   }

   tmp = gnutls_protocol_get_name(gnutls_protocol_get_version(state));
   printf("- Version: %s\n", tmp);

   tmp = gnutls_compression_get_name(gnutls_compression_get_algo(state));
   printf("- Compression: %s\n", tmp);

   tmp = gnutls_cipher_get_name(gnutls_cipher_get_algo(state));
   printf("- Cipher: %s\n", tmp);

   tmp = gnutls_mac_get_name(gnutls_mac_get_algo(state));
   printf("- MAC: %s\n", tmp);

   return 0;
}

\end{verbatim}
