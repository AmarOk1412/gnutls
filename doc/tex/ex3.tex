\begin {verbatim}

int print_info(GNUTLS_STATE state)
{
       const char *tmp;
       const X509PKI_CLIENT_AUTH_INFO *x509_info;

       /* print the key exchange's algorithm name
        */
       tmp = gnutls_kx_get_name(gnutls_get_current_kx(state));
       printf("- Key Exchange: %s\n", tmp);

       /* in case of X509 PKI
        */
       if (gnutls_get_auth_info_type(state) == GNUTLS_X509PKI) {
	      x509_info = gnutls_get_auth_info(state);
	      if (x509_info != NULL) {
		     switch (x509_info->peer_certificate_status) {
		     case GNUTLS_CERT_NOT_TRUSTED:
			    printf("- Peer's X509 Certificate was NOT verified\n");
			    break;
		     case GNUTLS_CERT_EXPIRED:
			    printf("- Peer's X509 Certificate was verified but is expired\n");
			    break;
		     case GNUTLS_CERT_TRUSTED:
			    printf("- Peer's X509 Certificate was verified\n");
			    break;
		     case GNUTLS_CERT_WRONG_CN:
			    /* the server's name
			     * is set by using the gnutls_set_x509_cn() function.
			     */
			    printf("- Peer's X509 Certificate was verified but it does not match the server's name\n");
			    break;
		     case GNUTLS_CERT_INVALID:
		     default:
			    printf("- Peer's X509 Certificate was invalid\n");
			    break;

		     }
	      }
       }
       printf(" - Certificate info:\n");
       printf(" - Certificate version: #%d\n", x509_info->peer_certificate_version);

       PRINT_DN(peer_dn);

       printf(" - Certificate Issuer's info:\n");
       PRINT_DN(issuer_dn);


       tmp = gnutls_version_get_name(gnutls_get_current_version(state));
       printf("- Version: %s\n", tmp);

       tmp = gnutls_compression_get_name(gnutls_get_current_compression_method(state));
       printf("- Compression: %s\n", tmp);

       tmp = gnutls_cipher_get_name(gnutls_get_current_cipher(state));
       printf("- Cipher: %s\n", tmp);

       tmp = gnutls_mac_get_name(gnutls_get_current_mac_algorithm(state));
       printf("- MAC: %s\n", tmp);

       return 0;
}

\end{verbatim}
