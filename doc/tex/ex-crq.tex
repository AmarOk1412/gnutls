\begin{verbatim}

#include <stdio.h>
#include <stdlib.h>
#include <gnutls/gnutls.h>
#include <gnutls/x509.h>
#include <time.h>

/* This example will generate a private key and a certificate
 * request.
 */

int main()
{
   gnutls_x509_crq crq;
   gnutls_x509_privkey key;
   unsigned char buffer[10*1024];
   int buffer_size = sizeof(buffer);
   int ret;

   gnutls_global_init();

   /* Initialize an empty certificate request, and
    * an empty private key.
    */
   gnutls_x509_crq_init(&crq);

   gnutls_x509_privkey_init(&key);

   /* Generate a 1024 bit RSA private key.
    */
   gnutls_x509_privkey_generate(key, GNUTLS_PK_RSA, 1024, 0);

   /* Add stuff to the distinguished name
    */
   gnutls_x509_crq_set_dn_by_oid(crq, GNUTLS_OID_X520_COUNTRY_NAME,
				     0, "GR", 2);

   gnutls_x509_crq_set_dn_by_oid(crq, GNUTLS_OID_X520_COMMON_NAME,
				     0, "Nikos", strlen("Nikos"));

   /* Set the request version.
    */
   gnutls_x509_crq_set_version(crq, 1);

   /* Set a challenge password.
    */
   gnutls_x509_crq_set_challenge_password(crq, "something to remember here");

   /* Associate the request with the private key
    */
   gnutls_x509_crq_set_key(crq, key);

   /* Self sign the certificate request.
    */
   gnutls_x509_crq_sign(crq, key);

   /* Export the PEM encoded certificate request, and
    * display it.
    */
   gnutls_x509_crq_export(crq, GNUTLS_X509_FMT_PEM, buffer,
			      &buffer_size);

   printf("Certificate Request: \n%s", buffer);


   /* Export the PEM encoded private key, and
    * display it.
    */
   buffer_size = sizeof(buffer);
   gnutls_x509_privkey_export(key, GNUTLS_X509_FMT_PEM, buffer,
				  &buffer_size);

   printf("\n\nPrivate key: \n%s", buffer);

   gnutls_x509_crq_deinit(crq);
   gnutls_x509_privkey_deinit(key);

   return 0;

}

\end{verbatim}
