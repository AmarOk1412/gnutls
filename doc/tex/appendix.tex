\chapter{Patents}
\label{ap:patents}
Patents in algorithms is a disputable topic, although some countries have 
decided to approve such patents. In order for \gnutls{} to be free in all
countries, we try not to include patented algorithms, which could turn the 
library being non free.

A patent which we couln't avoid was a patent by Netscape Communications 
Corporation on the Secure Sockets Layer (\ssl{}) work that \tlsI{} is based on.
Fortunately Netscape has provided a statement that allows royalty free 
adoption and use of the \ssl{} protocol. Below is the quote of Netscape 
Communications' statement in RFC2246.

\begin{verbatim}
    Intellectual Property Rights

    Secure Sockets Layer

    The United States Patent and Trademark Office ("the PTO")
    recently issued U.S. Patent No. 5,657,390 ("the SSL Patent")  to
    Netscape for inventions described as Secure Sockets Layers
    ("SSL"). The IETF is currently considering adopting SSL as a
    transport protocol with security features.  Netscape encourages
    the royalty-free adoption and use of the SSL protocol upon the
    following terms and conditions:

      * If you already have a valid SSL Ref license today which
        includes source code from Netscape, an additional patent
        license under the SSL patent is not required.

      * If you don't have an SSL Ref license, you may have a royalty
        free license to build implementations covered by the SSL
        Patent Claims or the IETF TLS specification provided that you
        do not to assert any patent rights against Netscape or other
        companies for the implementation of SSL or the IETF TLS
        recommendation.
\end{verbatim}


\chapter{Certificate to XML\index{Certificate to XML convertion} convertion functions}

\label{ap:xml}

This appendix contains some example output of the XML convertion
functions:
\begin{itemize}
\item \printfunc{gnutls_x509_certificate_to_xml}{gnutls\_x509\_certificate\_to\_xml}
\item \printfunc{gnutls_openpgp_key_to_xml}{gnutls\_openpgp\_key\_to\_xml}
\end{itemize}

\section{An X.509 certificate}
\begin{verbatim}

<?xml version="1.0" encoding="UTF-8"?>
<certificate type="SEQUENCE">
  <tbsCertificate type="SEQUENCE">
    <version type="INTEGER" encoding="HEX">02</version>
    <serialNumber type="INTEGER" encoding="HEX">01</serialNumber>
    <signature type="SEQUENCE">
      <algorithm type="OBJECT ID">1 2 840 113549 1 1 4</algorithm>
      <parameters type="ANY">
        <md5WithRSAEncryption encoding="HEX">0500</md5WithRSAEncryption>
      </parameters>
    </signature>
    <issuer type="CHOICE">
      <rdnSequence type="SEQUENCE OF">
        <unnamed1 type="SET OF">
          <unnamed1 type="SEQUENCE">
            <type type="OBJECT ID">2 5 4 6</type>
            <value type="ANY">
              <X520countryName>GR</X520countryName>
            </value>
          </unnamed1>
        </unnamed1>
        <unnamed2 type="SET OF">
          <unnamed1 type="SEQUENCE">
            <type type="OBJECT ID">2 5 4 8</type>
            <value type="ANY">
              <X520StateOrProvinceName>Attiki</X520StateOrProvinceName>
            </value>
          </unnamed1>
        </unnamed2>
        <unnamed3 type="SET OF">
          <unnamed1 type="SEQUENCE">
            <type type="OBJECT ID">2 5 4 7</type>
            <value type="ANY">
              <X520LocalityName>Athina</X520LocalityName>
            </value>
          </unnamed1>
        </unnamed3>
        <unnamed4 type="SET OF">
          <unnamed1 type="SEQUENCE">
            <type type="OBJECT ID">2 5 4 10</type>
            <value type="ANY">
              <X520OrganizationName>GNUTLS</X520OrganizationName>
            </value>
          </unnamed1>
        </unnamed4>
        <unnamed5 type="SET OF">
          <unnamed1 type="SEQUENCE">
            <type type="OBJECT ID">2 5 4 11</type>
            <value type="ANY">
              <X520OrganizationalUnitName>GNUTLS dev.</X520OrganizationalUnitName>
            </value>
          </unnamed1>
        </unnamed5>
        <unnamed6 type="SET OF">
          <unnamed1 type="SEQUENCE">
            <type type="OBJECT ID">2 5 4 3</type>
            <value type="ANY">
              <X520CommonName>GNUTLS TEST CA</X520CommonName>
            </value>
          </unnamed1>
        </unnamed6>
        <unnamed7 type="SET OF">
          <unnamed1 type="SEQUENCE">
            <type type="OBJECT ID">1 2 840 113549 1 9 1</type>
            <value type="ANY">
              <Pkcs9email>gnutls-dev@gnupg.org</Pkcs9email>
            </value>
          </unnamed1>
        </unnamed7>
      </rdnSequence>
    </issuer>
    <validity type="SEQUENCE">
      <notBefore type="CHOICE">
        <utcTime type="TIME">010707101845Z</utcTime>
      </notBefore>
      <notAfter type="CHOICE">
        <utcTime type="TIME">020707101845Z</utcTime>
      </notAfter>
    </validity>
    <subject type="CHOICE">
      <rdnSequence type="SEQUENCE OF">
        <unnamed1 type="SET OF">
          <unnamed1 type="SEQUENCE">
            <type type="OBJECT ID">2 5 4 6</type>
            <value type="ANY">
              <X520countryName>GR</X520countryName>
            </value>
          </unnamed1>
        </unnamed1>
        <unnamed2 type="SET OF">
          <unnamed1 type="SEQUENCE">
            <type type="OBJECT ID">2 5 4 8</type>
            <value type="ANY">
              <X520StateOrProvinceName>Attiki</X520StateOrProvinceName>
            </value>
          </unnamed1>
        </unnamed2>
        <unnamed3 type="SET OF">
          <unnamed1 type="SEQUENCE">
            <type type="OBJECT ID">2 5 4 7</type>
            <value type="ANY">
              <X520LocalityName>Athina</X520LocalityName>
            </value>
          </unnamed1>
        </unnamed3>
        <unnamed4 type="SET OF">
          <unnamed1 type="SEQUENCE">
            <type type="OBJECT ID">2 5 4 10</type>
            <value type="ANY">
              <X520OrganizationName>GNUTLS</X520OrganizationName>
            </value>
          </unnamed1>
        </unnamed4>
        <unnamed5 type="SET OF">
          <unnamed1 type="SEQUENCE">
            <type type="OBJECT ID">2 5 4 11</type>
            <value type="ANY">
              <X520OrganizationalUnitName>GNUTLS dev.</X520OrganizationalUnitName>
            </value>
          </unnamed1>
        </unnamed5>
        <unnamed6 type="SET OF">
          <unnamed1 type="SEQUENCE">
            <type type="OBJECT ID">2 5 4 3</type>
            <value type="ANY">
              <X520CommonName>localhost</X520CommonName>
            </value>
          </unnamed1>
        </unnamed6>
        <unnamed7 type="SET OF">
          <unnamed1 type="SEQUENCE">
            <type type="OBJECT ID">1 2 840 113549 1 9 1</type>
            <value type="ANY">
              <Pkcs9email>root@localhost</Pkcs9email>
            </value>
          </unnamed1>
        </unnamed7>
      </rdnSequence>
    </subject>
    <subjectPublicKeyInfo type="SEQUENCE">
      <algorithm type="SEQUENCE">
        <algorithm type="OBJECT ID">1 2 840 113549 1 1 1</algorithm>
        <parameters type="ANY">
          <rsaEncryption encoding="HEX">0500</rsaEncryption>
        </parameters>
      </algorithm>
      <subjectPublicKey type="BIT STRING" encoding="HEX" length="1120">30818902818100D00B49EBB226D951F5CC57072199DDF287683D2DA1A0EFCC96BFF73164777C78C3991E92EDA66584E7B97BAB4BE68D595D225557E01E7E57B5C35C04B491948C5C427AD588D8C6989764996D6D44E17B65CCFC86F3B4842DE559B730C1DE3AEF1CE1A328AFF8A357EBA911E1F7E8FC1598E21E4BF721748C587F50CF46157D950203010001</subjectPublicKey>
    </subjectPublicKeyInfo>
    <extensions type="SEQUENCE OF">
      <unnamed1 type="SEQUENCE">
        <extnID type="OBJECT ID">2 5 29 35</extnID>
        <critical type="BOOLEAN">FALSE</critical>
        <extnValue type="SEQUENCE">
          <keyIdentifier type="OCTET STRING" encoding="HEX">EFEE94ABC8CA577F5313DB76DC1A950093BAF3C9</keyIdentifier>
        </extnValue>
      </unnamed1>
      <unnamed2 type="SEQUENCE">
        <extnID type="OBJECT ID">2 5 29 37</extnID>
        <critical type="BOOLEAN">FALSE</critical>
        <extnValue type="SEQUENCE OF">
          <unnamed1 type="OBJECT ID">1 3 6 1 5 5 7 3 1</unnamed1>
          <unnamed2 type="OBJECT ID">1 3 6 1 5 5 7 3 2</unnamed2>
          <unnamed3 type="OBJECT ID">1 3 6 1 4 1 311 10 3 3</unnamed3>
          <unnamed4 type="OBJECT ID">2 16 840 1 113730 4 1</unnamed4>
        </extnValue>
      </unnamed2>
      <unnamed3 type="SEQUENCE">
        <extnID type="OBJECT ID">2 5 29 19</extnID>
        <critical type="BOOLEAN">TRUE</critical>
        <extnValue type="SEQUENCE">
          <cA type="BOOLEAN">FALSE</cA>
        </extnValue>
      </unnamed3>
    </extensions>
  </tbsCertificate>
  <signatureAlgorithm type="SEQUENCE">
    <algorithm type="OBJECT ID">1 2 840 113549 1 1 4</algorithm>
    <parameters type="ANY">
      <md5WithRSAEncryption encoding="HEX">0500</md5WithRSAEncryption>
    </parameters>
  </signatureAlgorithm>
  <signature type="BIT STRING" encoding="HEX" length="1024">B73945273AF2A395EC54BF5DC669D953885A9D811A3B92909D24792D36A44EC27E1C463AF8738BEFD29B311CCE8C6D9661BEC30911DAABB39B8813382B32D2E259581EBCD26C495C083984763966FF35D1DEFE432891E610C85072578DA7423244A8F5997B41A1F44E61F4F22C94375775055A5E72F25D5E4557467A91BD4251</signature>
</certificate>

\end{verbatim}


\section{An OpenPGP key}
\begin{verbatim}

<?xml version="1.0"?>

<gnutls:openpgp:key version="1.0">
 <OPENPGPKEY>
  <MAINKEY>
    <KEYID>BD572CDCCCC07C3</KEYID>
    <FINGERPRINT>BE615E88D6CFF27225B8A2E7BD572CDCCCC07C35</FINGERPRINT>
    <PKALGO>DSA</PKALGO>
    <KEYLEN>1024</KEYLEN>
    <CREATED>1011533164</CREATED>
    <REVOKED>0</REVOKED>
    <KEY ENCODING="HEX"/>
    <DSA-P>0400E72E76B62EEFA9A3BD594093292418050C02D7029D6CA2066EFC34C86038627C643EB1A652A7AF1D37CF46FC505AC1E0C699B37895B4BCB3E53541FFDA4766D6168C2B8AAFD6AB22466D06D18034D5DAC698E6993BA5B350FF822E1CD8702A75114E8B73A6B09CB3B93CE44DBB516C9BB5F95BB666188602A0A1447236C0658F</DSA-P>
    <DSA-Q>00A08F5B5E78D85F792CC2072F9474645726FB4D9373</DSA-Q>
    <DSA-G>03FE3578D689D6606E9118E9F9A7042B963CF23F3D8F1377A273C0F0974DBF44B3CABCBE14DD64412555863E39A9C627662D77AC36662AE449792C3262D3F12E9832A7565309D67BA0AE4DF25F5EDA0937056AD5BE89F4069EBD7EC76CE432441DF5D52FFFD06D39E5F61E36947B698A77CB62AB81E4A4122BF9050671D9946C865E</DSA-G>
    <DSA-Y>0400D061437A964DDE318818C2B24DE008E60096B60DB8A684B85A838D119FC930311889AD57A3B927F448F84EB253C623EDA73B42FF78BCE63A6A531D75A64CE8540513808E9F5B10CE075D3417B801164918B131D3544C8765A8ECB9971F61A09FC73D509806106B5977D211CB0E1D04D0ED96BCE89BAE8F73D800B052139CBF8D</DSA-Y>
  </MAINKEY>
  <USERID>
    <NAME>OpenCDK test key (Only intended for test purposes!)</NAME>
    <EMAIL>opencdk@foo-bar.org</EMAIL>
    <PRIMARY>0</PRIMARY>
    <REVOKED>0</REVOKED>
  </USERID>
  <SIGNATURE>
    <VERSION>4</VERSION>
    <SIGCLASS>19</SIGCLASS>
    <EXPIRED>0</EXPIRED>
    <PKALGO>DSA</PKALGO>
    <MDALGO>SHA1</MDALGO>
    <CREATED>1011533164</CREATED>
    <KEYID>BD572CDCCCC07C3</KEYID>
  </SIGNATURE>
  <SUBKEY>
    <KEYID>FCB0CF3A5261E06</KEYID>
    <FINGERPRINT>297B48ACC09C0FF683CA1ED1FCB0CF3A5261E067</FINGERPRINT>
    <PKALGO>ELG</PKALGO>
    <KEYLEN>1024</KEYLEN>
    <CREATED>1011533167</CREATED>
    <REVOKED>0</REVOKED>
    <KEY ENCODING="HEX"/>
    <ELG-P>0400E20156526069D067D24F4D71E6D38658E08BE3BF246C1ADCE08DB69CD8D459C1ED335738410798755AFDB79F1797CF022E70C7960F12CA6896D27CFD24A11CD316DDE1FBCC1EA615C5C31FEC656E467078C875FC509B1ECB99C8B56C2D875C50E2018B5B0FA378606EB6425A2533830F55FD21D649015615D49A1D09E9510F5F</ELG-P>
    <ELG-G>000305</ELG-G>
    <ELG-Y>0400D0BDADE40432758675C87D0730C360981467BAE1BEB6CC105A3C1F366BFDBEA12E378456513238B8AD414E52A2A9661D1DF1DB6BB5F33F6906166107556C813224330B30932DB7C8CC8225672D7AE24AF2469750E539B661EA6475D2E03CD8D3838DC4A8AC4AFD213536FE3E96EC9D0AEA65164B576E01B37A8DCA89F2B257D0</ELG-Y>
  </SUBKEY>
  <SIGNATURE>
    <VERSION>4</VERSION>
    <SIGCLASS>24</SIGCLASS>
    <EXPIRED>0</EXPIRED>
    <PKALGO>DSA</PKALGO>
    <MDALGO>SHA1</MDALGO>
    <CREATED>1011533167</CREATED>
    <KEYID>BD572CDCCCC07C3</KEYID>
  </SIGNATURE>
 </OPENPGPKEY>
</gnutls:openpgp:key>

\end{verbatim}


\input{error_codes}
