\begin{verbatim}

#include <stdio.h>
#include <stdlib.h>
#include <sys/types.h>
#include <sys/socket.h>
#include <netinet/in.h>
#include <arpa/inet.h>
#include <unistd.h>
#include <gnutls/gnutls.h>

/* A very basic TLS client.
 */

#define MAX_BUF 1024
#define CRLFILE "crl.pem"
#define CAFILE "ca.pem"
#define SA struct sockaddr
#define MSG "GET / HTTP/1.0\r\n\r\n"

int main()
{
   const char *PORT = "443";
   const char *SERVER = "127.0.0.1";
   int err, ret;
   int sd, ii;
   struct sockaddr_in sa;
   gnutls_session session;
   char buffer[MAX_BUF + 1];
   gnutls_certificate_client_credentials xcred;
   const int protocol_priority[] = { GNUTLS_TLS1, GNUTLS_SSL3, 0 };
   const int kx_priority[] = { GNUTLS_KX_RSA, 0 };
   const int cipher_priority[] = { GNUTLS_CIPHER_3DES_CBC, GNUTLS_CIPHER_ARCFOUR_128, 0};
   const int comp_priority[] = { GNUTLS_COMP_NULL, 0 };
   const int mac_priority[] = { GNUTLS_MAC_SHA, GNUTLS_MAC_MD5, 0 };


   gnutls_global_init();

   /* X509 stuff */
   gnutls_certificate_allocate_credentials(&xcred);

   /* set's the trusted cas file
    */
   gnutls_certificate_set_x509_trust_file(xcred, CAFILE, GNUTLS_X509_FMT_PEM);

   /* connects to server 
    */
   sd = socket(AF_INET, SOCK_STREAM, 0);

   memset(&sa, '\0', sizeof(sa));
   sa.sin_family = AF_INET;
   sa.sin_port = htons(atoi(PORT));
   inet_pton(AF_INET, SERVER, &sa.sin_addr);

   err = connect(sd, (SA *) & sa, sizeof(sa));
   if (err < 0) {
      fprintf(stderr, "Connect error\n");
      exit(1);
   }
   /* Initialize TLS session 
    */
   gnutls_init(&session, GNUTLS_CLIENT);

   /* allow both SSL3 and TLS1
    */
   gnutls_protocol_set_priority(session, protocol_priority);

   /* allow only ARCFOUR and 3DES ciphers
    * (3DES has the highest priority)
    */
   gnutls_cipher_set_priority(session, cipher_priority);

   /* only allow null compression
    */
   gnutls_compression_set_priority(session, comp_priority);

   /* use GNUTLS_KX_RSA
    */
   gnutls_kx_set_priority(session, kx_priority);

   /* allow the usage of both SHA and MD5
    */
   gnutls_mac_set_priority(session, mac_priority);


   /* put the x509 credentials to the current session
    */
   gnutls_credentials_set(session, GNUTLS_CRD_CERTIFICATE, xcred);


   gnutls_transport_set_ptr( session, sd);
   /* Perform the TLS handshake
    */
   ret = gnutls_handshake( session);

   if (ret < 0) {
      fprintf(stderr, "*** Handshake failed\n");
      gnutls_perror(ret);
      goto end;
   } else {
      printf("- Handshake was completed\n");
   }

   gnutls_record_send( session, MSG, strlen(MSG));

   ret = gnutls_record_recv( session, buffer, MAX_BUF);
   if (gnutls_error_is_fatal(ret) == 1 || ret == 0) {
      if (ret == 0) {
         printf("- Peer has closed the GNUTLS connection\n");
         goto end;
      } else {
         fprintf(stderr, "*** Received corrupted data(%d) - server has terminated the connection abnormally\n",
                 ret);
         goto end;
      }
   } else {
      if (ret == GNUTLS_E_WARNING_ALERT_RECEIVED || ret == GNUTLS_E_FATAL_ALERT_RECEIVED)
         printf("* Received alert [%d]\n", gnutls_alert_get(session));
      if (ret == GNUTLS_E_REHANDSHAKE)
         printf("* Received HelloRequest message (server asked to rehandshake)\n");
         gnutls_alert_send_appropriate( session, ret); /* we don't want rehandshake */
   }

   if (ret > 0) {
      printf("- Received %d bytes: ", ret);
      for (ii = 0; ii < ret; ii++) {
         fputc(buffer[ii], stdout);
      }
      fputs("\n", stdout);
   }
   gnutls_bye( session, GNUTLS_SHUT_RDWR);

 end:

   shutdown(sd, SHUT_RDWR);     /* no more receptions */
   close(sd);

   gnutls_deinit(session);

   gnutls_certificate_free_credentials(xcred);

   gnutls_global_deinit();

   return 0;
}

\end{verbatim}
